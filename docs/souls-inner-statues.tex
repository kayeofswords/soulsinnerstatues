% Options for packages loaded elsewhere
\PassOptionsToPackage{unicode}{hyperref}
\PassOptionsToPackage{hyphens}{url}
%
\documentclass[
]{book}
\usepackage{amsmath,amssymb}
\usepackage{lmodern}
\usepackage{iftex}
\ifPDFTeX
  \usepackage[T1]{fontenc}
  \usepackage[utf8]{inputenc}
  \usepackage{textcomp} % provide euro and other symbols
\else % if luatex or xetex
  \usepackage{unicode-math}
  \defaultfontfeatures{Scale=MatchLowercase}
  \defaultfontfeatures[\rmfamily]{Ligatures=TeX,Scale=1}
\fi
% Use upquote if available, for straight quotes in verbatim environments
\IfFileExists{upquote.sty}{\usepackage{upquote}}{}
\IfFileExists{microtype.sty}{% use microtype if available
  \usepackage[]{microtype}
  \UseMicrotypeSet[protrusion]{basicmath} % disable protrusion for tt fonts
}{}
\makeatletter
\@ifundefined{KOMAClassName}{% if non-KOMA class
  \IfFileExists{parskip.sty}{%
    \usepackage{parskip}
  }{% else
    \setlength{\parindent}{0pt}
    \setlength{\parskip}{6pt plus 2pt minus 1pt}}
}{% if KOMA class
  \KOMAoptions{parskip=half}}
\makeatother
\usepackage{xcolor}
\IfFileExists{xurl.sty}{\usepackage{xurl}}{} % add URL line breaks if available
\IfFileExists{bookmark.sty}{\usepackage{bookmark}}{\usepackage{hyperref}}
\hypersetup{
  pdftitle={The Soul's Inner Statues},
  pdfauthor={Kaye Boesme},
  hidelinks,
  pdfcreator={LaTeX via pandoc}}
\urlstyle{same} % disable monospaced font for URLs
\usepackage{longtable,booktabs,array}
\usepackage{calc} % for calculating minipage widths
% Correct order of tables after \paragraph or \subparagraph
\usepackage{etoolbox}
\makeatletter
\patchcmd\longtable{\par}{\if@noskipsec\mbox{}\fi\par}{}{}
\makeatother
% Allow footnotes in longtable head/foot
\IfFileExists{footnotehyper.sty}{\usepackage{footnotehyper}}{\usepackage{footnote}}
\makesavenoteenv{longtable}
\usepackage{graphicx}
\makeatletter
\def\maxwidth{\ifdim\Gin@nat@width>\linewidth\linewidth\else\Gin@nat@width\fi}
\def\maxheight{\ifdim\Gin@nat@height>\textheight\textheight\else\Gin@nat@height\fi}
\makeatother
% Scale images if necessary, so that they will not overflow the page
% margins by default, and it is still possible to overwrite the defaults
% using explicit options in \includegraphics[width, height, ...]{}
\setkeys{Gin}{width=\maxwidth,height=\maxheight,keepaspectratio}
% Set default figure placement to htbp
\makeatletter
\def\fps@figure{htbp}
\makeatother
\setlength{\emergencystretch}{3em} % prevent overfull lines
\providecommand{\tightlist}{%
  \setlength{\itemsep}{0pt}\setlength{\parskip}{0pt}}
\setcounter{secnumdepth}{5}
\usepackage{booktabs}
\usepackage{amsthm}
%\usepackage[english,greek]{babel}
\usepackage[utf8]{inputenc}
\makeatletter
\def\thm@space@setup{%
  \thm@preskip=8pt plus 2pt minus 4pt
  \thm@postskip=\thm@preskip
}
\makeatother
\ifLuaTeX
  \usepackage{selnolig}  % disable illegal ligatures
\fi
\usepackage[]{natbib}
\bibliographystyle{plainnat}

\title{The Soul's Inner Statues}
\author{Kaye Boesme}
\date{2022-03-09}

\begin{document}
\maketitle

{
\setcounter{tocdepth}{1}
\tableofcontents
}
\hypertarget{welcome}{%
\chapter{Welcome!}\label{welcome}}

\emph{The Soul's Inner Statues} is a guidebook for anyone who wants to get started worshipping many Gods, written from a slightly Platonizing framework. It is designed to be broadly relevant.

The book is based on blog posts, with additional information and contexts, on \href{https://kallisti.blog}{KALLISTI}, and on a blog of the same name that preexisted it. Most content was sourced from posts dating between 2009-2022. Currently, as of 9 March 2022, this project is in a \textbf{public beta} phase, with revisions anticipated over the summer. Feedback may be directed to kallisti {[}@{]} fea {[}.{]} st. You can also \href{https://github.com/kayeofswords/soulsinnerstatues/}{log an Issue in GitHub}.

\href{souls-inner-statues.epub}{You can download an ePub of this book (compatible with most eReading apps, including Google Play Books and Apple iBooks) here.}

There is an in-progress PDF located (in the GitHub codebase){[}\url{https://github.com/kayeofswords/soulsinnerstatues/tree/master/docs}{]}.

{The Soul's Inner Statues} by Kaye Boesme is licensed under a Creative Commons Attribution-NonCommercial-ShareAlike 4.0 International License. This means that anyone can take this book (recommended: use the GitHub source code) and create a new book based on it as long as you share in kind. If you decide to do this, I recommend replacing first-person anecdotes in the book with your own or referring to me in the third person (i.e., ``Kaye Boesme says that she \ldots{}'').

\hypertarget{intro}{%
\chapter{Introduction}\label{intro}}

An image that has always struck me in Plato's \emph{Symposium} is that of a hollow woodland spirit's statue: before it is closed up by its creator, it is filled with small figurine icons of Gods. One of Socrates' lapsed students, Alcibiades --- a person who had strayed from the path of self-discovery and the love of wisdom, choosing instead to pursue his love of politics --- describes the philosopher in this way during a passionate speech about how much Socrates drives him to do better and evaluate his actions. Like Alcibiades, we are all divided into many pieces, from work to family to society, our awareness cast about from thing to thing like a tiny vessel trying not to capsize on a tumultuous ocean. We struggle to know \emph{where} we are, let alone \emph{who} we are; forgetting our intrinsic wholeness, we seek to be whole. Coming to see the statues hidden within ourselves and the beings they represent --- and anchoring ourselves to them in the quest to flourish as human beings --- can go a long way to securing our inner calm, even in the face of grueling challenges, calamities, or the daily grind of work and life.

This book aims at being a small guidebook: a collection of tips and practices that can withstand the sprays of water, the stinging salt, and the passage of time. It aims to be a systemic synthesis useful to those of us in modern times, especially we who live in cultural melting pots like the United States, as we ponder what allures us about the dazzling statues of Gods and the strange stories hidden beneath the husk of commercialization and popular entertainment.

Finally, it is a hymn, both to the Gods and the underlying unity of all things that cannot be named, let alone conceived --- to the ones whose images are nestled within our souls like jewels awaiting discovery and reunion with our everyday selves. ``All things are full of Gods,'' the ancient Greek statesman Thales once said. Iamblichus, a Syrian philosopher writing in the early common era, wrote that conceding the Gods' existence was ``not the right way to put it''; he continued that ``an innate knowledge about the gods is coexistent with our nature, and is superior to all judgment and choice, reasoning and proof'' and ``{[}t{]}his knowledge is united from the outset with its own cause and exists in tandem with the essential striving of the soul towards the Good'' (\emph{De Mysteriis}, trans. Clarke et al., 2013, I.3). Unpacking what Iamblichus said --- the assumptions we may make on first reading it, the necessity of casting off misguided assumptions, and the joyous reality that unfolds as we open ourselves to the constellations of statues and the worlds within ourselves --- can guide us to the inner calm of correct opinion.

\hypertarget{the-basics}{%
\section{The Basics}\label{the-basics}}

This is a book about honoring Gods, and it is designed to be a practical work grounded in theory. While writing it, I had in mind an individual --- perhaps spiritual-but-not-religious, maybe religious-but-looking-for-more, even not religious at all --- who wants a connection to something greater than themselves and who is curious about Gods writ large. This person is not planning to join a group right now, nor are they committed to a specific cultural tradition that worships Gods. Maybe they grew up in a monotheistic faith or atheism and are trying things out, or maybe they grew up in a continuous, Gods-venerating tradition that no longer fits them right now. Maybe they are somewhere in between. It may also be useful for anyone who is looking to refresh a preexisting practice.

This primer will give you generic resources and tools you need to get started with prayer, and it prioritizes what I believe will help do that in a healthy way. The word choice I have opted for in this book is intentionally uplifting and positive, as deciding to do something about that desire for spiritual wholeness means that any reader has likely already encountered an overwhelming amount of information --- some good, some misinformed, some wrong. I believe that anyone, no matter what your background may be, can develop a fulfilling private spiritual practice, and that is why this primer exists. We will refer to this fulfilling practice as \emph{ritual}, and we will direct that ritual focus at the Gods.

Much as exercise is good for the body and meditation is good for the mind, revering the Gods is good for the soul. As with activities to strengthen the body and mind, establishing habitual ritual is good for its own sake --- you do not need to pursue initiation, become a theologian, or seek leadership. While a person may occasionally pray and ask for something specific, the primary purpose of venerating and praying to a higher power is to establish and deepen a relationship with them, a sacred dance between you and the holy.

Ritual practice and contemplation are simple: you set aside time, and ideally a small amount of your space, to take pause and acknowledge the divine. A ritual can be something as simple as lighting an electric or flame-based candle before murmuring a prayer of gratitude for another day. It could be pouring cool water into a bowl or sharing a few sips of one's morning coffee on the way out the door. It could be taking a few grounding breaths in front of an image of a God, cultivating fond feelings for what that God is known for and a desire to live up to that, in quiet meditation.

For the purposes of this book, the divine consists of divine persons --- Gods --- who are each unique individuals. Often, when the average person thinks about Gods, the term ``unique'' becomes a proxy for ``segmented into distinct offices'' like Tupperware in the fridge after a big meal prep. Commonly-used resources online will tell you that we have a God for luck, a God for love, a God for healing, and so on. It is only after plunging into worship that we learn that many of these labels are inaccurate and overly simplistic. Every God is very complicated, and as described by the philosophers and theologians, each of them is a boundary-less, unique window on the universe --- completely simple, each is an inner sanctum, a God who proceeds forth from that absolute unity. I pray to Apollon, for example, whose offices ultimately express that this God's perspective is one related to harmony. The entire universe can be viewed through that lens, and the God can encompass everything. A devotee of another God may say the same thing about the one they follow, and our perspectives can coexist in mutual acknowledgment.

It is the dizzyingly complex interpolation of a myriad Gods into the receptive beauty of matter that causes everything around us. The Gods flow forth with abundance. Like floodwaters bringing life to a desert, they give rise to intermediary spirits whose identities are often preserved in folklore, folk practices, and habitual rituals.

Coming into harmony with the Gods, as Iamblichus taught, is the foundation of human happiness. This connection is something that cannot be taken away from any of us because it is innate. Like the image of Socrates as an artisinal work filled with Gods' statues, each of us contains a unique window on divine reality that can anchor us. We can adapt how we live our lives through ritual, contemplation, and mindfully tuning into the seasons of the year and our lives. Becoming conscious of our connection to the Gods is a lifelong process, a dance around that anchor, and regular practice enriches our awareness of it no matter how much the winds howl.

Despite calling a spiritual practice \emph{simple}, I know that it can be very daunting to start one. Speaking as an American, many of us are taught that spiritual practice is something you get by paying money to attend a yoga class in a chic studio or filling a seat in a church beside other people who may or may not actually want to be there or declaring yourself to \emph{be something}. Doing a prayer or small ritual every day can be viewed as strange or over-the-top to some. However, it is far from uncommon among spiritual people. Zen practitioners, if they want to move from casual to proficient, are expected to do at least two hours of zazen practice each day as a \emph{layperson} (see note 16 at the end of \emph{Hidden Zen} by Meido Moore). Muslims pray five times each day, and it is expected of everyone who is capable of it. In Shinto, those who keep kamidana (a type of household shrine for Kami, a word that describes a wide range of spiritual beings) give offerings daily, at least aspiratonally.

Taking two to ten minutes a day to devote to one's spiritual practice is achievable to most people.

This book will take that statement --- again, \emph{two to ten minutes is achievable} --- and provide guidance to \emph{make} it achievable.

\hypertarget{recommended-supplies}{%
\section{Recommended Supplies}\label{recommended-supplies}}

Each of these chapters contains small exercises designed to connect you to an aspect of the topic under investigation. Sometimes, you will need materials. I recommend having a small bowl on hand, a vessel (like a jar) that contains water, and a slip of paper. If you can clear off a part of one shelf, that is ideal. If not, just use a clean surface and store your prayer materials in a clean box when you're not using them.

You can, of course, be more elaborate. Once we get going with the exercises, you will have opportunities to offer items like incense. When we move from talking about many Gods in the generic to actually choosing who to honor, you may want to purchase icons of those Gods or do research on Etsy to save up for one. This book is open access for a reason: I am aware that we have a variety of economic situations. Cultivating your inner light --- and connecting to the Gods --- is something that is accessible no matter how simple your prayer space is. A set of small bowls and a pitcher for pouring will likely cost \$3-5 at Goodwill; used sauce jars will cost you nothing more than the time in cleaning them out.

Many people end up engaging in spiritual consumerism and filling their prayer spaces up with material items to make themselves feel more spiritual. I err on the functional side. Everything you put in this space is a tool, or a vessel, to help you connect to God(s). Some parts of these spaces --- particularly the images of Gods --- are extremely sacred. If a fire or other disaster came, how many of these precious images could we put in our go bags? Probably not many. Be reasonable about your space and ask yourself questions about ergonomics, durability, and maintenance whenever you consider adding something.

\hypertarget{starting-prayer}{%
\section{Starting Prayer}\label{starting-prayer}}

Let's take a quick pause.

Be in a space where you will not be disturbed for two minutes. (It's OK if this is your car, but the bathroom is not appropriate.) Close your eyes and take a few deep breaths. Silently count to four while breathing in, and four while breathing out.

After a few cycles of breath, hold out your hands, with bent elbows, palms up. Your palms may either be together or apart.

At this point, you may either pray from the heart or use the words I am providing below. It is okay to open your eyes unless that is distracting for you.

\begin{quote}
I honor and acknowledge the Gods I know, the Gods I do not know, and the good divinities willing to guide me. Please grant me what I need, and as I embark on a new beginning to study how best to approach you, may the actions I undertake guide me serendipitously to the place where I can be steady, happy, and truly free.
\end{quote}

\hypertarget{dedication}{%
\section{Dedication}\label{dedication}}

As one seeker of truth to another, I hope that the words in this book will be useful to your own voyage through life.

I will end the introduction with part of a prayer that I give every morning:

Please, Apollon, let me find the still heart of truth at the core of all things. Please cultivate within me skill --- in my poetry, in my prose, and in all of my actions, let all that I do flow forth from you.

I will add to this a prayer to the Goddess Aletheia:

May you, Aletheia, guide the reader towards truth and grant them a sip from your compassionate, abiding nectar, and may my words in this open access book be good enough to give them what they need for their journey.

This book is dedicated to Apollon and Aletheia.

\hypertarget{foundations}{%
\chapter{Foundations}\label{foundations}}

This chapter is titled \emph{Foundations} because we will dive a bit deeper into thinking about Gods now, at the outset. It is also titled \emph{Foundations} because the main practice at the end of this chapter involves the cleared-off section of shelf and small bowl I mentioned in the introduction.

Specifically, we will look at what Gods actually are, and we will use Plato's \emph{Laws} to unpack some common pitfalls that happen when we think about them --- mindsets that most of us may have slipped into at some point or other in our lives. We will then consider the foundations of contemplative practices, ritual practices, and household observances. Finally, you will have an opportunity to do some brief veneration on your own.

While explaining these things, I will ask you questions.

One option when encountering these questions? Write out the answers, either longhand or in a text editor.

Other options? If you are a parent, beholden to one or more demanding jobs with no work-life balance, or writing is just not accessible right now, that's okay. Think about these questions while you are in the shower, during your commute, or when completing tasks that don't require a heavy cognitive load. If something comes up that you want to keep hold of for later, use a note-taking app on your phone or create a draft email that you will send to yourself.

\hypertarget{what-is-a-god-anyway}{%
\section{What is a God, Anyway?}\label{what-is-a-god-anyway}}

In the Introduction, I quoted from two philosophers who were operating in a similar cultural context to each other. Thales, a Greek philosopher who was active before Socrates' lifetime (and is thus called a ``Presocratic''), said that the world is full of Gods; Iamblichus, a Syrian fluent in both his native culture and Greek culture of the late third century of our common era, described a God's existence as a given, but strangely so.

Not once did I bring up the predomiant understanding of the word \emph{God} in the modern era, which is often regared as a \emph{specific} someone --- a singular omnipotent, omnipresent, omniscient being who is often tied to the initiatory rites of revealed religions, a being only accessible when someone adheres to the correct holy book and doctrines. Sometimes in spiritual circles, people pray to the Source, Spirit, or the One, as a way of being ``generic'' and ``accessible,'' referring to a single God without as much salvific baggage. My reluctance to discuss this broader cultural material is partly related to my own background --- my outlook on the Gods is Platonizing, which means that I do not believe that the first principle is capable of being described in that way. Even before my outlook was Platonizing, I grew up Neopagan and had significant misgivings about the theology I was taught in the late 90s and early 00s related to how we collapsed Gods into one God and Goddess, ultimately Spirit (and that's what led me to Plato). This book is grounded in my perspective even though my goal is to provide something practical and useful for beginners. My second reason for \emph{not} bringing it up in the introduction was that doing so would set an inappropriate tone for this work --- this work is not a defense of Gods, but in praise of them. My third reason is that I want to push back against the totalizing understanding of words like \emph{god}, \emph{deity}, or \emph{divinity}. Many of us assume what these words mean based on tacit things we learned as children, not on anything systematic.

\emph{God} is used as a class term for a wide range of divine persons, sometimes accurately, sometimes imprecisely. Often, the term is used when translating indigenous terms for divine beings. The word \emph{spirit} is used just as often. The Shinto term \emph{Kami}, for example, describes a range of beings, with many of them more reminiscent of divinities like nymphs, river deities, or house wights. The words \emph{theos} and \emph{daimon} in Greek are equally problematic. A daimon is an intermediary spirit that carries out specific functions, but sometimes, a God is called a daimon in surviving writings from Ancient Greece! River divinities are usually referred to with \emph{theos}, but are they a \emph{theos} in the same way the Goddess Athene is? Are planetary Gods similar to other Gods even though most planetary bodies will perish when their stars go nova, or are they more like extremely long-lived nature spirits? Which, if any, beings from another language receive the term ``God''? ``Spirit''? ``Wight''? It depends on the biases of who is translating and if they have a specific set of beings in mind when they choose a word. This is one reason why Wikipedia is often not as good as going to a religious organization's website (or, for decentralized religions, a few sites put together by different practitioners, ideally ones who are not in the same communication bubbles) to learn what its adherents believe.

Theological and philosophical exercises can help us narrow down what these categories mean. In Platonism, these elastic terms have been refined into four classes: gods, angels, daimones, and heroes. The final three classes can be grouped under ``daimones.'' Each of the daimonic classes has less contact with the ultimate divine reality and more contact with variability and change. Divinities of the natural world and household inhabit the spaces ``closest'' to us. You and I are located at the level just below heroes --- we are souls that incarnate in material bodies according to a long, regular cycle; a small sliver of us is permanent and godlike like the beings closer to godhood than us, and that is what we want to cultivate as \emph{I} in a spiritual practice. Other philosophical schools, like Yogic philosophy, Stoicism, or Daoist philosophy, have different takes on this. Generally speaking, though, we are all trying to become as Godlike as possible, to pull from Plato's \emph{Theaetetus} (176a-b).

The \emph{strict} definition of God I will take --- and the one I mean when I discuss our inner core of happiness and the statues within us --- are those individual divine persons who ground the entirety of reality. They slip and slide into one another, boundary-free because they preexist both limit and unlimitedness. They are the ones who we ultimately uncover in our practices. What is visible in breathtaking photographs from a space telescope or witnessed in the powerful fury of the world around us, or even the small actions of our daily lives, are all ultimately grounded in \emph{their} astonishing foundation. It is no wonder that Iamblichus wrote that ``an innate knowledge about the gods is coexistent with our nature'' and that they just \emph{are}, but strangely so. The \emph{loose} way in which I use the term ``God'' is when I am referring to any divine being for prose flow. All of the beings following from the Gods are nourished by their divinity, including the material world.

Most of the time, I will refer to other categories of divine beings as spirits or divinities. I will also occasionally refer to ancestors. There may even be a triad of \emph{Gods, spirits, and ancestors} in some of these materials, although ancestor worship is not the primary focus of this book.

\begin{itemize}
\item
  What are the definitions you have heard for the term \emph{God}? How do they differ from what I have said?
\item
  Try looking up the divine beings in a variety of traditions. What are the similarities? Differences? How well does the term \emph{God} fit what you see?
\end{itemize}

\hypertarget{mindset-lessons-from-plato}{%
\section{Mindset Lessons from Plato}\label{mindset-lessons-from-plato}}

One passage from Plato's \emph{Laws} can ground our understanding of the Gods even more. Plato's \emph{Laws} is a long city-soul analogy that describes how parts of the soul function in an embodied context, and it was questionably complete when Plato died --- some parts of it seem rushed, and some analogies that are stridently hammered through by the speakers in later books are not as deftly scaffolded in as they are in Plato's other works, especially when compared to the care he took in his other long city-soul analogy, the \emph{Republic}. The \emph{Laws} is said to have still been ``on the wax'' when he died, meaning that it had not been transferred to a more lasting medium like his finished pieces.

Plato's works (apart from his letters) take the form of dialogues --- dramatic vignettes in which speakers encounter one another and ask questions, usually featuring Socrates. The \emph{Laws} does not include Socrates as a speaker, but an unidentified old man named the Athenian Stranger, who is doing a pilgrimage hike with two other elderly men. Like the \emph{Republic}, it is long and broken up into many sections. Book 10 of the \emph{Laws} contains a conversation on atheism and piety. The Athenian Stranger says:

\begin{quote}
``People {[}commit offenses{]} in one of three frames of mind: either lacking the belief {[}in the Gods{]} I mentioned; or second, believing that there \emph{are} gods, but that they care nothing for human beings; or third, that they are easily won over by inducements in the form of sacrifice and prayer.'' Laws 10, 885b
\end{quote}

In Platonism, an embodied rational soul has three parts: our thinking part, which can connect to the Gods and which survives death; and an irrational soul divided in two, the emotional and appetitive parts. The irrational soul perishes because our feelings and appetites are contingent on our specific embodiment (although depending on the specific person one asks, some aspects of the irrational soul can persist across several lifetimes). Our consciousness, and our self, is lasting. The three types of atheism correspond roughly to the three parts of the soul. A lack of belief in Gods is an intellectual atheism. Believing the Gods do not care or influence us is an atheism of care, an alienation from our own emotionality. Believing they can be swayed by offerings is an atheism of appetite. Proper piety requires unlearning these three traps so we can get out of our own way.

In the \emph{Laws}, the Athenian Stranger discusses the lack of belief in Gods by referencing the planetary bodies and other types of natural phenomena as a way to jolt someone into reverence. Writing in 2022, as someone with a love of modern astronomy, it was daunting for me at first to read the discussion. At face value, his argument relies on outmoded ways of looking at the cosmos. We do not believe that the Earth is at the center --- the planets orbit the Sun, and the Sun orbits the center of the Milky Way, which inhabits a local group of galaxies that is in a region of galaxies called the Virgo Supercluster. Our robots have been to the surface of Mars, and they probe the secrets of Venus. We may even have dedicated scientific probes orbiting Europa soon.

What daunted me is a common problem among many of us in the United States, and it traces itself to culturally-specific baggage. The dominant form of religion here has a strong anti-science contingent. When we think about Gods, especially in America, we are living in the shadow of the way Christian religious rhetoric operates in the public sphere. We are living in the wake of that religion's Great Awakenings, most recently in the charismatic and evangelical waves of the mid-20th century with their prosperity gospel, cultivated distrust of science, and weaponized piety. My paternal grandparents gave money to the scam that sent Jim Bakker to jail. We are encouraged to view jobs as \emph{callings}, a jargon word from Christianity for divine service, to prevent burnout when work gets rough. Often, our reluctance to engage with the divine is based on an inner fear that we will end up exactly like those toxic people, organizations, and cultural zeitgeists, or we are driven by the desire for social approval to just not investigate Gods at all because ``there are many Gods'' is not a common position for someone to have in the United States right now.

I was born in the late 1980s and experienced American public schooling in the 1990s and early 2000s. When I was in 9th grade, our biology teacher told the class --- a rural school filled with students who had been taught that evolution was hubristic sin against their sacred texts --- that he didn't believe in evolution, but he was obligated to teach it, and we had better fill out the state exams in a way that the graders of the exams liked. I had been taught a non-Christian form of intelligent design --- the divine created the cosmos, but science describes how the cosmos actually operates. In 2001, intelligent design was \emph{also} a hubristic sin to many Christians. It wasn't until a few years later that it became a tool for ultra-conservative Christians to sneak Biblical creationism into school curricula. For a few years, it was very possible to believe in a divine origin for the cosmos \emph{and} to love and appreciate science, and to do so publicly in intellectual circles, without anyone caring.

I choose to read Plato's Athenian Stranger as trying to jolt people into contextualizing their feelings of awe at the natural world. A deep, connected feeling in prayer is very similar to the feeling one gets when seeing the night sky in a place with very little light pollution or when we witness sunlight fall against hills in a breathtaking way. When it rains, I sometimes think of the diamond rain on Jupiter and fall into silence --- it is that smallness, that sense of being tuned into what is common between the worlds, that makes me feel connected to the cosmos. The same patterns exist everywhere. Nothing is exotic.

Let's look at the Iamblichus I quoted in the introduction again. Conceding the Gods' existence is ``not the right way to put it'' --- but why? He continues with ``an innate knowledge about the gods is coexistent with our nature, and is superior to all judgment and choice, reasoning and proof'' and ``{[}t{]}his knowledge is united from the outset with its own cause and exists in tandem with the essential striving of the soul towards the Good'' (\emph{De Mysteriis}, trans. Clarke et al., 2013, I.3) --- but how? And do all of us really feel that way, deep down? If we become quiet, if we progress from the turbulence of our daily lives to rest in that stillness, will knowledge of Gods just flow forth?

The Gods are not their myths, and they crown the totality of the cosmos, not just Earth. They are the preface to existence, and what they have lain down forms the scaffolding of everything we see in the universe. The universe itself is a Goddess. She expresses her order in a beautiful array of mathematical brilliance, in both the mundane everyday of our lives and the half-grasped dreams of theoretical physicists. Looking to a physical object like the Moon or a planet (or contemplating a story about diamond expolanets) anchors us in a distant, still-corporeal thing --- unless we use that knowledge of other worlds to jostle ourselves out of our subjective experience. It is that intellectual flight that we want to cultivate.

Beyond that discussion of how contemplating the natural world can draw us up to contemplating the Gods, the logic of belief and the Gods depends on several factors, ranging from philosophical school doctrines to deep, indescribable personal experiences. Proclus' \emph{Platonic Theology}, Book I, explains how the Gods fit into the emanative system of Platonism and how they transcend existence itself, and he uses language that is most accessible to those who have already studied Platonic dialogues and doctrines. Similar arguments occur in Stoic, yogic, and other works using the schema of each system. What they all come down to, though, is that looking for a ``god of the gaps'' or saying ``god works in mysterious ways'' are not correct statements. Setting the Gods as the \emph{base layer} of reality that gives rise to our material heterogeneity, regardless of the details of the system, harmonizes all of existence with our unique personal experiences, or lack thereof; in each system, we are trying to reach a place where we can see both the forest and the trees without dissociating from either perspective. There are no gaps. Everything around us is produced by the Gods' activity. Learning about the Gods may be complicated, and exploring how their beautiful unity unfolds is a daunting and frustrating task at times. What we uncover through deep thought and experience is ineffable, an understanding that defies reduction into words. This is not mysterious --- just challenging to communicate.

Moving on to the second type of error, believing that the Gods do not care comes from several roots. It is a way of cutting off our feelings and our emotional investment --- insulating ourselves from everyday disappointments and coping with the seemingly random horrors that come to pass in our lives. Sometimes, we cultivate this mindset because we believe it is the best way to be realistic and logical. It is true that the universe is enormous. We are each very small in comparison.

Plato's Athenian Stranger said:

\begin{quote}
``The gods are on our side --- as also are the guardian spirits --- and we in turn are the property of the gods and guardian spirits. What is fatal for us is injustice, and arrogance allied to folly; our salvation is justice, and self-control allied to wisdom, and these are to be found dwelling in the living powers of the gods --- though they can also be seen dwelling in us, just a bit --- or something very like them.'' Laws, Book 10, 906b
\end{quote}

In the system that Plato built, we are part of the organizational structure of the cosmos. We may be small, but we are each capable of developing agency and learning how to be as good and godlike as possible. Everything we see and touch, including ourselves, is part of the universe and the Gods' collaborative creation of it. Gods, intermediary spirits, and our guardian spirits, look after the whole and the parts within the whole. It is not conceited for a rose in a garden to know that its caretaker is checking its roots, pruning its leaves, and ensuring it grows well.

Sometimes it's hard to think about the whole. We are living on a small planet in a vast universe. The climate crisis touches each of us. We learn of new horrors every day. Many of us live in less than perfect circumstances. One challenge of the material world is that it has spatial and temporal extent, unlike the levels of reality that ground everything. The cosmos must unfold in time, and the way the pre-cosmic levels --- whose components overlap and interpenetrate without spatiality or conflict --- ``freeze'' into the material world introduces interference patterns. The evils we experience, the conflicts that devastate us, and the imperfections of the material world are all produced by this. When we incarnate, we make the best choice of life we can based on the options available to us and our disposition --- and, like deciding between a tooth extraction and a root canal, sometimes having a healthy tooth is just not on the table. We can possess realistic awareness of our material surroundings while holding the Gods as good. Once embodied, we do possess some level of agency. We can make the options better or worse for the souls choosing lives on this planet in the future, including our own, through deliberate effort to tackle challenges and create the societies we want.

Now, let's talk about appetitive atheism, or the belief we can sway the Gods with offerings. Many of us who have looked at \emph{National Geographic} or who have taken any history class know that offerings to Gods are a huge component of many societies. When I originally learned about Ancient Greece and Rome, for example, it was explained that the elaborate systems of offerings were used to goad the Gods into granting us favor --- and this idea was backed up in plays, epics, and other written pop culture works from the ancient Mediterranean. There are people who attempt to bribe the Gods to ``offset'' immoral behavior, like lies and theft, to this day. Believing this is possible means believing that the Gods are \emph{not} good or stable. It means believing that they are vulnerable to appetites. Anything that has an appetite lacks something, which means it is not complete or perfect.

This apparent paradox is most apparent when we look at myth. In myths, Gods can have very strong appetites, with good or disastrous impacts. Many of these stories are violent. The stories force us to look deeper at what the myth is actually saying about how a God impacts the world. I will only touch on this briefly here, but I recommend reading Sallust/Sallustius' ``On the Gods and the World,'' Chapter IV. In that chapter, Sallust discusses how myths operate and how we interpret them.

For anyone who has done literary criticism, the practice of mythic exegesis is very similar. For example, in one myth, Medusa is transformed and exiled by Athene after being raped by Poseidon in Athene's temple. Athene then has a champion hero, Perseus, go to the lair of Medusa to kill her and bring back the head. Athene then uses this head to adorn her body. Athene is an intellectual Goddess --- philosophical and wisdom-filled, virginal and separated from generative actions like sex. In the Platonic tradition, the soul begins its journey in contact with the wisdom, beauty, truth, and love in the highest places our souls can access. At some point, we lose our stability and plunge into generation, the term for the material cosmos of ``coming-to-be,'' and we lose access to that nourishment in the discordant fall. The descent into generation was a descent into what is alien to us, a place where we can never exist without violence or pain. Poseidon is the Lord of the World of Becoming and is the ruler of generation --- his violence is a symbol of how our soul experiences our alienating, self-violating descent. Medusa here represents our souls. She is depicted with wings, reminiscent of the metaphorical wings we lost during our descent. The serpents represent renewal, the many incarnations the soul must go through, all attached to her reasoning core. Her petrifying gaze renders every living thing around her stone, much as we all fail, in some way or other, to truly see the living beings around us, instead relying on ossified mental images and conceptions. Perseus, as a heroic spiritual intermediary, liberates Medusa from the body and returns the immortal part of her to Athene, her proper guardian, completing the soul's cycle. Medusa becomes harmonious with the Goddess once again through directing her activity in tandem with the warlike Goddess' motions.

This interpretative analysis has several further implications that I could ponder: What does it mean to view the descent into generation as negative, given that the above interpretation includes elements of violence and loss? How do we view Medusa's older sisters, who are both immortal in some stories? Does it preserve or present problems to the interpretation? If every God is good, how do we interpret the violence of descent, as a God (Poseidon) presides over that violence?

Contemplating and wrestling with what myths mean is something that anyone can do, even without a ritual space. It becomes easier to think through these things with time and attentiveness, and depending on which Gods one worships, one may have intricate puzzles to work through. It is rewarding, though, for overcoming a shallow view of myths and coming to a better understanding of how appetites are not what they seem in these stories.

Why do we pray to the Gods, then, if they cannot be swayed and if the stories \emph{about} them being swayed are all to be interpreted allegorically? The Gods still care about everything within the realm of generation. While writing about the ancient world, Andrej and Ivana Petrovic, in \emph{Inner Purity and Pollution in Greek Religion}, quote beliefs of Pythagoras about prayer as recorded by Diodorus Siculus:

\begin{quote}
\textbf{For he himself {[}Pythagoras{]} disclosed that wise men should pray to the gods for the good things for the benefit of the unwise, since the unwise are incapable of understanding what in life is truly good.} (10.9.8) He used to say that it was necessary in prayers to pray simply for the good things, and not to name them individually, such as for instance to pray for power, beauty, wealth, and other similar things. For often each of these things, when those who desired them acquire them, turning against them, totally ruins them.

p.~63, emphasis added
\end{quote}

Pythagoras' mindset still rings true. Because the Gods care for us, and because they are more expansive than us, they have a more secure knowledge of what is good for us. I could pray to the Gods for money, power, and fame, but if it were granted, would it actually be the best thing for me? Probably not. Studies have shown that there are hard limits on the amount of happiness and fulfillment we can achieve through material and social wealth. Many millionaires, celebrities, and rulers are absolutely miserable, moreso than the general population.

In another dialogue, the \emph{Phaedrus}, Socrates and Phaedrus pray at the end for good things --- a generic prayer that leaves the specifics of what happens up to the God, one that expresses humility about our human knowledge. The generalized prayer accounts for our lack of knowledge about the future.

You will find that many theological and philosophical systems, when they are coupled with the worship of Gods, have similar features --- (often) multi-part souls, a complex relationship with mythic corpora, and a belief in the goodness of the Gods. If you have ideas about who you want to worship and which philosophical systems draw you, I encourage you to see how they handle these things.

\hypertarget{contemplative-practices}{%
\section{Contemplative Practices}\label{contemplative-practices}}

Even without setting aside physical space for spiritual practices, we can ground ourselves in the Gods through contemplative techniques. The title of this primer, \emph{The Soul's Inner Statues}, likely appeals to many who have tried some form of contemplation or breathwork before because the title names the soul and evokes some kind of interior spiritual experience. You may have tried a form of meditation using the Headspace or Calm apps, pranayama techniques in a yoga class, or the practice of taking a few deep breaths before or after engaging with something difficult.

\hypertarget{foundations-1}{%
\subsection{Foundations}\label{foundations-1}}

If you haven't tried meditation techniques, or if your experiences have not been what you wanted, here is some good news: there are many types of contemplative techniques, not all of which require sitting down and focusing on the breath, and online instructional materials are widely available for many types of practices. Breathwork techniques, chanting guidance, and guided meditations on focal-point-based mindfulness are some of the most widespread options.

Here are a few resources to get started:

\begin{itemize}
\tightlist
\item
  Anusha Wijeyakumar's \emph{Meditation with Intention: Quick and Easy Ways to Create Lasting Peace} seeks to provide useful foundations for anyone, including those who have experienced challenges when meditating. I took note of her book when I saw praise from people who expressed difficulty meditating.
\item
  Headspace, Calm, Healthy Minds Program, or another meditation app. Calm and Headspace offer free trials, and Healthy Minds is free. Core features of these apps center on mindfulness meditation. You may also have access to mental wellness apps like Sanvello (which includes meditation timers) through your workplace or insurance benefits.
\item
  For people interested in Stoicism, try out \href{https://donaldrobertson.name/2018/03/27/four-stoic-meditation-exercises/}{Donald Robinson's four Stoic meditation exercises} \href{https://web.archive.org/web/2021*/https://donaldrobertson.name/2018/03/27/four-stoic-meditation-exercises/}{(and here's an archival link)} or \href{https://stoameditation.com/}{the meditation app Stoa}.
\item
  If you're curious about contemplative activities in Platonism (again, this is my bias), I recommend Mindy Mandell's \emph{Discovering the Beauty of Wisdom}, which contains some contemplative techniques.
\item
  David Nowakowski has made \href{https://davidnowakowski.net/meditation/}{a primer on techniques available on his website}, and I recommend reading those resources. \href{https://web.archive.org/web/2021*/https://davidnowakowski.net/meditation/}{(Here's an archival link.)}
\end{itemize}

To time a meditation, you can use a dedicated phone app, the timer on your a watch or fitness tracker, the timer function in your clock app (with a non-jarring timer noise), or even a classic physical timer. Headspace and other paid apps have trials that you can use if you want some guided direction when just starting out.

When starting a contemplative practice, I recommend having a plan: Know when you will do it, which tool you will use to time it, and the technique(s) you will use during the brief period.

\hypertarget{contemplating-the-divine}{%
\subsection{Contemplating the Divine}\label{contemplating-the-divine}}

I do daily meditation as basic mental hygiene, as do many people --- ten-minute shower, ten minute-meditation, and my body and mind are ordered and ready to face the world. If we want to reach inside and open the gate to the statues within, and rest in the Gods who wait just beyond our breath, we need a divine set of focal points instead.

\hypertarget{contemplating-the-divine-being}{%
\subsubsection{Contemplating the Divine Being}\label{contemplating-the-divine-being}}

Instead of focusing on your breath, you might focus on a deity, household spirit, or ancestor you want to connect with --- and don't worry if you don't know \emph{who} quite yet.

This can be made a bit easier by using an image. If you have a smartphone, put your phone in airplane mode and make sure you have a few images saved in your photos. Create a dedicated folder so you can find them easily. My Android phone has a feature that keeps the phone on when I'm looking at it, and if yours can do similar, all you have to do is pull up one of those photos and make sure your phone can see you. You could also purchase a bookmark or postcard featuring the deity, use a printout image, or invest in artisinal divine images. (Do not go overboard on Etsy, especially if you are just starting out and don't know which deities you want to worship. That said, I have found some excellent wood bookmarks of Gods --- both Norse and Hellenic, and a few shops have excellent wood icons available, too.) It works well if the deity is looking out of the image towards you because you have the illusion of eye contact.

While focusing on the deity, just breathe. You can close your eyes if you like while holding the image in your mind. Using your meditation or phone timer is useful here.

Often, when I meditate on a God, I start with prayer beads. Because I do not craft, I have usually purchased these beads, often from sellers on Etsy or another site for small businesses and artisans. The beads that work best for me for most contemplations are short strings with between nine and twenty beads. I have selected short phrases from content I have come across for various Gods, and I have pored through other texts to find short snippets that I want to use in some other cases. A few sets of prayer beads I purchased came with prayers, and I use those prayers (with modifications). After chanting for several minutes, I come into stillness while maintaining mental focus on the God.

Sometimes, I don't start with prayer beads. I start by reading a poem (often, a hymn translation) for the God or thinking about aspects of the Gods in epithets and culture before I drop into stillness. I find that this is grounding --- like the image, it improves my focus. I will also usually light incense or give the God a libation.

\hypertarget{contemplating-a-text}{%
\subsubsection{Contemplating a Text}\label{contemplating-a-text}}

You could contemplate a myth or other text. This could be done in a traditional meditation, with a pen and paper at a desk, or at a computer --- just airplane your phone and ensure that you are difficult to contact while you're working through your thoughts. It's also something you can do on the go. I usually think about passages from what I'm reading while doing dishes or commuting. You know if you are a restless thinker who needs to be doing something tactile while your mind is working.

Optionally, say a small, heartfelt prayer to the God(s) in the myth. Read the passage you want to work with and jot down what comes to mind. Highlight any passages you want to spend extra attention on, and pause whenever inspiration strikes. You can come back to the same story many times. You can use specific passages from your notes as seeds for a seated meditation. Sometimes, if you alternate between taking notes on myths and contemplating a divine image, elements of the myth will arise spontaneously in the mind while you are doing the seated meditation on the God.

\hypertarget{breathwork-and-adapted-meditation-techniques}{%
\subsubsection{Breathwork and Adapted Meditation Techniques}\label{breathwork-and-adapted-meditation-techniques}}

For those who have exposure and practice with other forms of meditation, doing pranayama or using traditional meditation techniques can sometimes be adapted to a contemplative spiritual practice like this.

Sometimes, I do breath of fire (kapalbhati) before contemplating a God. I do meditative visualizations in which sunlight, starlight, or moonlight are filling the body, depending on the context --- honoring the solstice or full moon, for example.

There are also techniques called ``grounding and centering'' that I learned growing up. A person envisions that they are rooted into the ground like a tree, drawing up nourishment from the Earth, and drawing down energy from the sun or sun and stars as the centerpoint of this structure. It can be beneficial to do techniques like that if one is feeling scattered and distracted. Grounding and centering techniques are used in a variety of modern spiritual and religious traditions because they work well. You can find guided exercises made by people from a variety of backgrounds on YouTube.

\hypertarget{ritual-practices}{%
\section{Ritual Practices}\label{ritual-practices}}

Ritual practices are everywhere in contemporary culture --- glossy checkout magazines, online-only publications offering the gamut of self-care options, pop culture spirituality blogs, mainstream occult practices, corporate cultish bonding exercises, and so on. The app I currently use for personal development (the Fabulous App) sometimes calls my morning, workday, and evening routines ``rituals'' --- and there wasn't even a preexisting option for adding ``prayer'' to a routine.

In \emph{The Soul's Inner Statues}, we will use ``ritual'' to mean the activities we do to connect with Gods, divinities, and those who have come before us. A ritual in this context is a set of routine practices we use when engaging in this type of activity.

It is possible to pray without something being a ritual. Sometimes, when I am outside after work in the winter, I see the rising full moon, and I often murmur a quick prayer. There is no ritual involved in that spontaneous reverence.

\hypertarget{hearth-home-and-ancestors}{%
\section{Hearth, home, and ancestors}\label{hearth-home-and-ancestors}}

Who you pray to out of affinity --- this God, that Goddess, someone else --- are often like friendships that change and fade over time. A few are continuous, and many are transient. Household worship is very different because you live somewhere. Even if you are in a liminal space --- without a home, living in a dorm, or traveling --- there are Gods who preside over liminal spaces who can be deeply enriching to pray to.

The core element of household worship is making offerings to household Gods, often a hearth Goddess and one or more divinities who preside over things like property, storerooms, and abundance. Examples of these divinites are the Penates (Roman), house wights (various places), Agathos Daimon (Greek, meaning ``good spirit/intermediary''), Tykhe or Fortuna (Greek and Roman), the Kitchen God (Chinese folk religion), or a household-focused aspect of a God. Every God technically is a unique perspective on the entirety of everything, so technically speaking, you could worship any of them as your household God. However, the specific, preexisting associations people have made between a God and specific activities are powerful expressions of how that God operates in the world. They are also easiest to ``tap into'' for people who have never worshipped Gods. Some Gods with well-established household aspects, like Apollon, Zeus, and Hermes, have specific household functions --- Zeus of the storeroom, for example, is literally the God who keeps your cozy duvet inserts and tea safe; Apollon of the Streets protects public spaces and harmonizes the home with the exterior world through the threshold boundary. Hermes, as a liminal God, is highly relevant to boundaries. The Norse Goddess Frigg and the Goddesses associated with her (her Handmaidens) have associations with many domestic tasks.

Household worship can involve considerations of one's ancestors of affinity or heritage. Depending on your ancestry and your relationship with said ancestors, this will likely cause mixing and matching. For example, I pray to several hearth Goddesses right now: Frigg (Nordic), Nantosuelta and Brigando (both Celtic-Gaulish), and Hestia (Greek). I primarily worship Gods of the Greek pantheon despite not being Greek, and years after I started worshipping Hellenic Gods, I started exploring what it would be like to pray to deities related to my roots. Frigg was a certainty; less is known about the religious spheres of Gaulish Gods, so Nantosuelta and Brigando are more like an ongoing fact-finding mission based on things I've seen others do. It's okay to do fact-finding or to not be sure when one is just starting out --- or even to revisit long-held practices that one never worked through systematically before.

For reflection:

\begin{enumerate}
\def\labelenumi{\arabic{enumi}.}
\tightlist
\item
  Do you already know of any hearth deities? What about home divinities?
\item
  Who interests you the most? Look them up. Alternatively, \href{https://en.wikipedia.org/wiki/Household_deity\#List}{look at this list of hearth deities from around the world} --- I'm using a Wikipedia link because it's most accessible, but there are definitely cultures that are not represented in this incomplete list.
\end{enumerate}

\hypertarget{home-during-college}{%
\subsection{Home during college}\label{home-during-college}}

Household worship, from a practical standpoint, can be very simple. I have maintained a sacred space in my living area since I was twelve or so. In eighth grade, I made a bench in shop class that became my small floor-level shrine until I went to college, when I lived in the dorms. Incense and candles were forbidden for fire safety reasons, but I often had a small space on my bookshelf that was marked as ``sacred'' --- even if I often neglected it. As most young adult students do, I moved a lot: dorm room to dorm room, apartment to apartment, between my family's home and transient student spaces. You can \emph{still} have hearth deities during this period of your life and keep a shrine, even if it is very simple. Libations and electric candles go a long way. You may, however, prefer to focus on deities related to education, especially if you are living in a dorm setting.

\hypertarget{worship-without-being-creepy-about-heritage}{%
\subsection{Worship Without Being Creepy About Heritage}\label{worship-without-being-creepy-about-heritage}}

In Rumi's ``\href{https://www.scottishpoetrylibrary.org.uk/poem/guest-house/}{The Guest House},'' the poet frames our lives as transitory places:

\begin{quote}
This being human is a guest house.\\
Every morning a new arrival.

A joy, a depression, a meanness,\\
some momentary awareness comes\\
as an unexpected visitor.
\end{quote}

Even before I encountered that poem, I had been contemplating the idea of our lives as temporary places. As someone who reads Plato and who holds to Platonic teachings about reincarnation, I know that our families, lineages, and personal contexts change from lifetime to lifetime. There is no steadiness in them. Pardoxically, in Plato and the Platonists, there is a strong emphasis on meeting the context of your current life where it is and paying your respect to Gods and ancestors. However, this is ideally like being a guest in someone's house: the family may last a long time, and the people, but you will not be there forever. You need to leave the family better than you found it, and you need to be mindful that the family is not \emph{you} --- you are an incarnating soul.

When I started to embark on incorporating ancestral practices, it was important to me to do it without being creepy --- it's contextual to this lifetime. The ultimate goal for anyone in incorporating such practices is to ensure that one is not neglecting important divinities and Gods, and reaching out to ancestral deities is also an act of repair for those of us whose ancestors ruptured ties to Gods by converting to an evangelizing, one-true-way religion. In the popular TV series \emph{The Good Place}, interpersonal obligations were built up through the phrase ``what do we owe each other?'' --- also the title of a philosophical book that presumably is also working through what our web of connections and obligations mean. The way it played out on the TV show involved characters learning etiquette and mutual respect. If we want to set boundaries between ourselves and absolutism and intolerance, changing one's worldview to repair such rifts can work wonders. Obviously, there are many times when the answer to ``should I be worshipping you, Gods who may have been worshipped by my great-great-great-great-great grandparents?'' is no, and that's fine.

Because we are worshipping Gods and divinities who are individuals and not archetypes, an analogy can be drawn to reaching out to family after an estrangement you inherited from your parents --- it may be awkward, it may fail, but it's worth it to try. For example, if one's ancestors lived in areas where they likely had a Lararium (a household shrine for the Lares) eighteen hundred years ago, trying that out again may be a useful first step. If your family has lived in an area where people had Lararia eighteen hundred years ago, but you ultimately trace your roots elsewhere, you should also feel empowered to explore having one --- you live there, and it's part of your cultural history, too.

There are other ways our current incarnation can impact us. In the \emph{Phaedrus} commentary taken down by Hermias during one of the Platonist Syrianus' lectures fifteen hundred years ago, the text comments that someone will often reincarnate into a family to resolve injustices from generations ago. When we incarnate, we pick what is best for our soul given our options. This can include repairing such ruptures, all the while knowing that we may not even be in the group that benefits from these activities next lifetime.

Reaching for ancestral traditions is healthy when it comes out of a desire to do what is just and to repair what was broken or cast aside: We repair our relationship with the Gods and we commit to transmitting it to those who come after us. We should ask ourselves \emph{how, why, and whom} as prompts to deepen our practice without succumbing to hate, division, and false senses of superiority. Questions like these can also heal a disorienting sense of unplacing when they are approached from a place of compassion, fierce honesty, and care, and while the answers may create some very personalized practices, it will be a personalized ritual practice that is solidly grounded in piety according to what is most just in each person or family's specific case.

Admittedly, in some cases, Gods were transmitted to us so many generations ago that they feel like ours, and we need to be mindful about recognizing that the living descendants of the ancient cultures they come from have their own histories and relationships to the same Gods --- it's part of being culturally and globally aware. In America, most of us, regardless of our background, trace our cultural history to what was going on in Britain four hundred to two hundred years ago when they were an imperial power taking everything from everybody. Those of us (non-Greek) Americans who worship Greek Gods do so because British people really liked them during the Renaissance and Early Modern period, and fluency in Greek literature and its motifs proved someone was cultured even after the texts started to be translated into English. Greece is 1800 miles away from England. It is a very different culture. The same goes for Britain's Egypt craze and its impact on the accessibility of information about Egyptian Gods.

\hypertarget{today-is-not-yesterday}{%
\subsection{Today Is Not Yesterday}\label{today-is-not-yesterday}}

Iamblichus wrote in V.25 of \emph{On the Mysteries} that the traditions are not ``just a matter of human customs'' driven by convention. Rather than originating from us, the ``God is the initiator of these things, he who is called `the god who presides over sacrifices,' and there is also a great multitude of gods and angels in attendance upon him'' and every holy place, people, and sacrifice have divine overseers such that ``when we perform our sacrifices to the gods with the backing of gods as supervisors and executives of the sacrificial procedure, we should we should on the one hand pay due reverence to the regulation of the sanctity of divine sacrifice, but on the other we may have due confidence in ourselves'' because we are approaching the Gods as the God intended us to. I will add that, while the initiator of these rites is a God, human beings can never perfectly receive what is intended by the Gods.

Importantly, specific traditions' social inequality structures are not things that the Gods are responsible for; in the Platonic tradition, evils are a phantasmic occurrence caused by the extreme spatio-temporal partiality of the material world, and as parts of the material world, traditions are no exception. The generations that created these traditions may have had extremely poignant insights about the Gods. As our predecessors were human beings, just as we are human, it was just as easy for them to ignore heinous injustices they were habituated to accept from youth as it is for \emph{us}.

The Gods do not discriminate based on your race, ethnicity, social class, disability status, sexual orientation, gender, caste, and so on. When we pray, we are praying as souls reaching out to a God. All of us belong. Plotinus encouraged each of us to never stop working on our statue --- to always be willing to be better. Traditions for worshipping Gods are also like that statue. Especially in a private context, or in a small group of people with similar values, we have the opportunity to correct what is not good \emph{yet}. When we keep the God at the center of any corrections we make on our end, we ensure that the core keeps its integrity even if some peripheral elements change. And? Your practice is your practice. What you do at your shrine, as long as you are being a respectful and decent person, is just as good as anyone else's effort. Sometimes, people who love their Gods, but who face harm from other devotees, think that they need to choose between remaining in a specific community and receiving abuse and leaving and losing the Gods. No matter how hard it is, you can always leave and bring the Gods you love with you.

\hypertarget{practice}{%
\section{Practice}\label{practice}}

In the introduction, I recommended clearing off a shelf or small space to use for a household shrine. During Chapter One's practice, you will make use of this space.

Sometimes, no matter what, we run into challenges setting aside space. If you live in a college town, May is a great time to find tables and shelves cheaply (or for free), as students are leaving, and many of them are on tight deadlines. FreeCycle and other no-buy groups are other options. If this is not possible, or if you are waiting for the right time to find a place for your shrine, make space on your floor or wall that you can dedicate to the Gods. I found \href{https://web.archive.org/web/20220103123537/https://religionnews.com/2021/10/13/millennial-hindus-get-creative-with-home-temples-in-cramped-apartments/?utm_source=pocket_mylist}{this post about young South Asians coming to the United States and how they set up their sacred areas to be particularly inspiring for thinking about what one can do with limited space}. IKEA recently came out with a cutting board with legs. It is inexpensive and is perfect for elevating an offering space ever-so-slightly, whether you are placing it on the ground or on part of a table or desk.

If you are unable to establish a permanent shrine space, create a folder on your smartphone with images of the hearth Goddesses and divinities of interest to you. Turn on the phone feature that keeps the phone screen on when you're looking at it. Put the phone on a stand on a clean surface, full-screen the image after airplaning your phone, and offer prayers in front of the image.

Again, you need:

\begin{itemize}
\tightlist
\item
  Something for pouring liquid.
\item
  Something to pour liquid into.
\item
  The name of the deity you're worshipping --- either a post-it note or something fancier. (I often use popsicle sticks, and I will use colorful markers to make them look nice.)
\end{itemize}

That's it.

Often, especially when we're looking at ritual space stock photography or the spaces of people who have committed disposable income to pricier items, we see beautiful statuary and intricate spaces. This helps, but it's not necessary -- and, given the climate crisis, it makes sense to be deliberate about purchases instead of contributing to overconsumption. Once you have a good idea that you want to worship someone and have maintained the habit for long enough, go ahead and think about images.

The bare minimum you need for worshipping someone is their name. In antiquity, when people were pouring libations and making sacrifices on altars, the altar was marked with the name of the God or Gods. It is like having someone's address in the \emph{to:} field when you send an email.

For a hearth deity, I also recommend a small candle or flame representation. You can use a traditional lighter or a rechargeable electric lighter. Some people even use electric candles! There are phone apps that simulate fires and candles, too.

For a libation, any drink is fine --- even water. I buy loose dried tulsi and make a concentrate that I store in the refrigerator. Whenever I want to pray, I pour some into the libation jar I use and add water. I also offer a flavored water concentrate that I like, especially to my ancestral Gods --- it feels very intimate to share beverages I actually drink. Some people offer wine or other alcohols. I find that I don't go through the bottles quickly enough to use up all of the alcohol before it goes off, and since I rarely drink, it feels odd to give a God something that isn't part of my diet unless the deity has a historical relationship to alcohol.

Other types of offerings include incenses, flowers, and food. If you use incense, be careful of resins --- do not burn them indoors on charcoal disks, as there is a carbon monoxide risk. Burning resins outdoors with a fire-safe setup is fine. Stick and cone incenses are great for daily indoor use, and some companies have low-smoke incenses. If you use essential oils in an oil diffuser in your offering space, please check that your oil is pet-safe if you have animals --- vets have been more public in recent years about what to choose and avoid. Never use citrus oils around your cat.

Disposing of offerings can be done in a clean sink (for libations) or in your household trash. If you offer food or flowers, the most respectful way to put them in your waste receptacle is to wrap them in paper (like a used grocery bag from the time you forgot your reusable bags, padding paper left over from a package, or similar) and add them to the trash just before you put it out curbside. If you can compost, that is preferred, but don't worry if you can't --- I, too, live in an apartment without that option. If you offer non-fresh goods like dried flowers or wreaths, rotating them out after six months to a year is auspicious --- especially if you do it at about the time of the winter or summer solstice.

Sometimes, people eat the food offerings shortly after they are offered --- the food is now blessed by the God(s), and eating it is the completion of the reciprocal relationship. Traditionally, some practices (i.e., for Gods worshipped in Egypt) have done this by default. In other areas, it has been common according to custom for the Gods' portion to be separated from what everyone will eat. Do what seems right to you, \protect\hyperlink{underworld}{except for underworld Gods}.

If you are setting up a permanent space, be respectful. Never use it as a landing space for random objects and clutter. Encourage family members and/or roommates to avoid putting anything there that isn't an offering.

\hypertarget{household-prayer}{%
\subsection{Household Prayer}\label{household-prayer}}

Hopefully, you now have a shelf or another location where you can create a pop-up prayer space. If you have purchased or gathered any items, now is the time to place them there. Clean them, and then arrange them in an ergonomically useful way.

Before praying, I recommend washing your hands. If you are praying after your daily shower, congratulations, you are now habit-stacking.

If you have a candle or other flame representation, light it or turn it on. Keep your elbows at your sides and hold out your forearms with your hands palm-up. Standing up is the common practice, although one can sit if necessary.

Say something like:

\begin{quote}
I honor and acknowledge the household Gods, {[}insert any specific names{]}. I give you this offering of {[}whatever you're offering{]}.
\end{quote}

Make the offering. If you are doing a libation, this means pouring whatever you have chosen to offer into the bowl. Sometimes, after offering most of something, a person will take a sip of the remainder.

Take a pause and a few deep breaths.

If you are honoring specific Gods and spirits and know anything about them, this is the time to call to mind those aspects. For Nantosuelta, for example, one might think of her connection to beehives, fire, and Earth --- far from just being a Goddess of the home, she has many areas of affinity, many of which are not related to the home in the slightest.

Sometimes, people will recite poems they've found about the deities and divinities they're worshipping, play music, or speak from the heart about what they have going on in their lives. You could do that now if you like.

When done, say:

\begin{quote}
Thank you, {[}insert any specific names or just say ``household God(s)''{]}. May you bless me, my family, my friends, and my communities with whatever is most good, fitting, and appropriate.
\end{quote}

Dispose of the offering in a few hours. I like to clean out my offering bowls after I've done dishes in the evening.

\hypertarget{meditation}{%
\subsection{Meditation}\label{meditation}}

This meditation is based on an exercise from one of Plotinus' \emph{Enneads} and on something I've done since I was a teenager.

The passage from Plotinus' \emph{Enneads} is at 5.8.9, and here's how it is translated by Gerson (note that the pronoun ``he'' is used for the God, but this could be any deity regardless of gender):

\begin{quote}
So, let us grasp by discursive thinking this cosmos all together as one, each of its parts remaining what it is and not jumbled together, if possible, so that if any one of these should occur to us -- for example, the sphere outside the periphery of the cosmos -- an image of the sun follows immediately and together with it all the other stars, and earth and sea and all the living beings are seen, as if all these were in reality to be seen in a transparent sphere. Let there be formed in your soul, then, the image of a luminous sphere having all things in it, whether moving or stable, or some moving and some stable.

Keeping this image, take another for yourself by abstracting the mass from it. Abstract, too, places and the semblance of the matter you have in yourself. Don't try to take another sphere smaller than it in mass, but call on the god who made that of which you have a semblance, and pray for him to come. And he might come bearing his cosmos with all of the gods in it, being one and all of them, and each is all coming together as one, each with different powers, though all are one by that multiple single power. Rather, it is that one god who is all. For he lacks nothing, if all those gods should become what they are. They are all together and each is separate, again, in indivisible rest, having no sensible shape -- for if they had, one would be in one place, and one in another, and each would not have all in himself. Nor do they have different parts in different places, nor all in the identical place, nor is each whole like a power fragmented, being quantifiable, like measured parts. It is rather all power, extending without limit, being unlimited in power. And in this way, the god is great, as the parts of it are all unlimited. For where could one say that he is not already present?
\end{quote}

For this prayer, place a chair or cushion near your shrine to use as a meditation seat. Make a short prayer:

\begin{quote}
I pray to the Gods and give you this offering before my meditation. Be well disposed.
\end{quote}

If you have an offering, make it now, and then be seated.

For this meditation, start off by doing a body scan, eyes closed or open. Start the scan at the top of your head, working through parts of the body to see what is comfortable or uncomfortable. Notice the pressure of gravity where you are seated.

Once you have finished the body scan, turn your attention to the boundary between your skin and the air. Envision your awareness expanding from your body and into the room around you. Feel the furniture and become aware of the life within your home. From there, expand your awareness steadily to every room in your home or apartment. If you are in an apartment, expand your awareness to the entire building.

From your building, steadily come to encompass the neighborhood, then the city, and finally your region. Allow your awareness to expand in gradients until you are holding the world in your mind. Examine the differences in climate and ground cover, sunlight levels and depth of night. Rest here for a moment.

When you are ready, expand it outward from the Earth to the Earth-Moon system. Grow through the void to reach the planets of the inner solar system and the Sun, then the asteroid belt and the outer planets, and finally the mass of smaller bodies that make up the frozen fringes of the solar system. Expand your awareness to the nearest star systems, then to the galaxy. Know that we are in a network of galaxies, and expand into that. Eventually, your awareness will encompass all of the honeycombed web of the universe. It is vast in its voids and light oases.

All of this has evolved in time, and it came from a point of indescribably high density. The visible universe was once about the size of a grapefruit. You are containing at least all of this within your awarness.

Now, think of all that you are holding within your mind, and call to mind what it would be without mass. Call to mind all that is, and then imagine it spaceless. Contract all that you have become aware of without giving it mass, shape, or extent in time. Beyond the spatial extent, beyond time, and beyond bodies, there is an inflection point of divinity. As Plotinus wrote so long ago, ask for the God to come.

Whatever comes to your awareness, rest with it until you judge the meditation to be finished. Thank the Gods and put back your chair or cushion.

This exercise is a bit different from what is happening in Plotinus' Ennead 5.8.9, as it fuses the abstraction of his words and their beauty with an expansion into everything visible around us. However, it can still be a powerful visualization practice. If visualization isn't right for you, I encourage you to replace this with another meditation technique.

\hypertarget{time-blocking-exercise}{%
\section{Time Blocking Exercise}\label{time-blocking-exercise}}

In the above practices, you have done a ritual prayer and embarked on the opening of the way to your soul's inner statues --- a beautiful thing to be celebrated. It may feel new, different, and give you a sense of calm (at least, once the awkward feeling of newness fades) --- but chances are, you figured out a time to do these exercises that fits into your daily routine.

In Chapter 3, which focuses on purification, I recommend doing prayers right after you do your shower (or whichever freshening up activity you do each day). Chapter 3 will also give you an opportunity to create backup scenarios for when life gets hectic.

Now, though, I want you to think about your morning, afternoon, or evening routine --- whichever time you think it is best to pray. And I want you to time block it.

Time blocking could involve post-it notes, notecards, or a piece of paper. You could also use a digital post-it app like Mural, a workflow diagram generator, or anything you are comfortable with.

Think of what your routine is like for the time of day you have chosen. Write down each step, including the amount of time it takes you. This is a private activity, so please do not treat yourself as your aspirational self. If you make your coffee/tea and stand there by your French press entranced by your Twitter feed for twenty minutes, note that down.

Once you have a realistic image of your routine, you can think about what you want to adjust. The rituals we are discussing in \emph{The Soul's Inner Statues} can be as swift as five minutes. Can you wake up five minutes earlier, or is there time you can adjust? Are there other aspects of your routine that you're unhappy with?

Work through your routine until you find something that is both satisfying and actionable. Transfer the information to a physical sheet of paper (one or two is ideal) and put it where you will see it at that time of day.

Try not to change too much --- for habit-building, the most lasting change is incremental. Once you have made good progress on this for a few weeks, go back to your notes about your routine and pick one other thing to shift. For example, if you want to read more, ensure that there's a book by your coffee/tea setup instead of your phone.

As someone who prays in the morning, avoiding social media is what works for me so I can have that extra prayer time. Because I pray for longer than five minutes, I also wake up earlier --- at 6:15 AM instead of 6:45 AM. I like having time to give prayers to a variety of Gods, including Gods of the sacred day in the calendar system I learned in my early 20s, and to sit in contemplation for at least a few minutes. However, I do have that quick ritual routine in my back pocket if I oversleep. Sometimes, I do prayer beads after I brush my teeth and before I go to bed, which lasts only a few minutes unless an idea for a poem comes to me. The timing of your prayer is up to you.

\hypertarget{gods}{%
\chapter{Gods}\label{gods}}

We devoted much of the last chapter to discussing many Gods, and we ended with some essentials: household worship and taking time for contemplation. In this chapter, we will build on that and pick one or two more Gods to worship --- and, for now, no more than that.

It is not necessary to worship every single God every day (that would be impossible), just as one cannot be friends with every human being in the world or every sentient being in the cosmos. Beyond worshipping the household Gods and divinities, and perhaps ancestor worship, many people only have the bandwidth for a handful more before their attention is too scattered to have a deep and meaningful practice --- and \emph{The Soul's Inner Statues} is focused on breaking down the steps to work towards such a practice. Sometimes, a person may decide to pray to a collective of Gods: the Muses, for example, or the Fates, Matronae, Attendants of Frigg, Gods of Civic Discourse, Storm Gods, and so on. In my personal practice, I usually pray to the Muses as a collective. Others may devote prayers to all of the Gods in general --- often using names and terminology relevant to a specific cultural grouping of Gods, such as the twelve Olympians in Plato's \emph{Phaedrus} or a set of major Roman, Nordic, Celtic, Gaulish, or other deities.

Keeping the number of Gods limited at first helps you settle into a concise prayer routine, and it allows for mindful expansion of your practice if you judge that expanding it is important to you. Billions of people around the world pray for only a few minutes a day. The most important thing, and the hardest, is showing up.

As you read this chapter, take some notes on what comes up for you. The exercise at the end will ask you to look up one or two Gods and to pray to them.

\hypertarget{how-do-you-pick-a-god}{%
\section{How Do You Pick a God?}\label{how-do-you-pick-a-god}}

The factors involved in choosing Gods to worship are many. In this section, we will discuss a few major ones.

\textbf{Culture and region.} For many people, this is what is done: You worship the God whose temple is down the street (or whose ruins are in your area) or Gods who are widely present in your cultural substrate, whether or not your culture writ large worships them. In the United States, specific regions, cities, and states may use the images of specific Classical Gods or Goddesses, and adding a reverential aspect is often very straightforward.

\textbf{Following what a group does.} People raised in a specific religious group, or who join a group as an adult that honors specific deities, often do what everyone else is doing.

\textbf{Heritage.} Maintaining a connection to where we come from is important to many people, especially in the United States and other colonies. Whether someone's ancestors arrived in a place voluntarily or via human trafficking and abuse, it can sometimes be important for a person to explore their origins when selecting who to worship. My only word of caution is to ensure that you approach this mindfully. Especially for those of us with various European ancestries and ethnic backgrounds, a ``return to the roots'' can be coopted by the far-right. Anyone can worship the Gods. Resist far-right rhetoric. You are here for your reasons, and others have their own --- everyone has a perspective to offer. Developing a practice for the Gods, being committed to learning, and approaching the practice with reverence are what matters. Let your own thing be about repair without making it creepy.

\textbf{Activity.} This is exactly what it sounds like. You know what you like, what you do for work and hobbies, and what you want to be. You find Gods who have well-documented aspects in one or more of those areas and decide to worship them. Most people who worship Gods use this as a criterion for selecting someone to worship.

\textbf{Astrology.} In astrological systems, specific Gods are associated with specific signs of the Zodiac. These deities may shift slightly based on the system a writer is using. Sometimes, the position of the rising sign or which sign the moon is in matter just as much as one's sun sign. You can explore this with some Google searches on Gods for specific signs. Keep in mind that ``astrology'' is a very general term, so you will see results for Western, Vedic, and other forms of astrology. Speaking Platonically, the soul accumulates various traits (or garments) as it descends into the world of becoming, and the positions of the stars at birth have been used as signals of which Gods will be particularly prominent for a person that lifetime. Some astrologers believe that the planets are the Gods, but even if a planet is associated with a God, the God themself transcends that planetary association. Still, the symbolism is a way to connect to a God.

\textbf{Divination-based.} Sometimes, especially in the African Diaspora religions, the God(s) someone should worship closely are determined in divination. Divination protocols for such rituals are usually very intricate, and they require study and care for the practitioner to do them well. This book is not about those traditions, and if you are curious about them, I encourage you to do research on the ones that interest you, ideally from actual practitioners and leaders in those traditions. Outside of those traditions, others may also consult with divination specialists about deities. Note here that ``divination'' is a broad category that encompasses far more than tarot and oracle cards, and it's not just about predicting the future! Often, people use divination to seek advice on something that has been bothering them or gain insights into circumstances they are facing so they can make good choices.

\textbf{Experience.} In Proclus' surviving hymns to the Gods, he makes frequent references to holy words of the wise that excite the soul towards the Gods. This sometimes literally happens to people. Very rarely, you're reading something, you have a dream, or a freaky chance occurrence hits, and it sends chills down your spine. It could even send the world topsy-turvy for a few hours or days. Something feels \emph{right} about it. You may know the name of the God, or you may just know vague things about them. It's a gift when this happens. As a word of caution, keep an eye on your mental health --- brain conditions can mimic spiritual experiences, so practice self-awareness and be sure to follow up with a doctor if necessary. (Generally speaking, ``you are the chosen one'' is something that only happens in Hollywood movies and messianic fringe groups.) Intense spiritual experiences don't happen every day. Praying for a dream or an unambiguous sign is something people have done for millennia in pursuit of some milder spiritual experiences.

\hypertarget{an-example-of-choice}{%
\subsection{An Example of Choice}\label{an-example-of-choice}}

When I was a teenager, I decided on Apollon and the Muses because I felt that poetry was the only thing I was good at. In the American cultural reception of Apollon, he is often taken to be synonymous with music and poetry. Little did I know that, when I made that choice, things would change and expand out from there --- Apollon is a harmonizing God, first revealed in the truth-bearing triad in Platonism before weaving his way through various levels of emanation from the One. I was worried when I started studying library science that I still thought of him as my God when my work had nothing to do with him, and I struggled to find ways to relate my life to what I knew about him from his epithets, what scholars had written, and the things I saw people post online about worshipping him.

For several years, I strayed away and primarily worshipped Hermes, with Athene secondary, because they are very professionally relevant to my daily work. This was naïve of me given what I know now, and as soon as I was ready to know more about Apollon, specific things in my life fell into place that landed me back worshipping him --- everything I needed to know was rooted in the Platonists. I worship Hermes much less now, but still in a professional capacity. We're human. How we pray, and who we pray to, evolves over time.

\hypertarget{what-about-specific-traditions}{%
\subsection{What About Specific Traditions?}\label{what-about-specific-traditions}}

\emph{The Soul's Inner Statues} is a book about worshipping Gods at home, not about any specific religious tradition. Many of my examples draw on the Hellenic Gods because I have worshipped them for so long (in my own non-Greek context; if you're interested in them, I encourage you to connect with Greeks), and as a follower of Plato, it's important for avoiding cultural appropriation that I engage with the Gods of the Platonists. Specific religious traditions often have guidelines that are used to help devotees select a deity, if the tradition doesn't just choose the God for them. What happens in those traditions usually falls into one of the categories that I have named above.

Some marginalized groups are very protective of their ancestral practices, usually due to outsiders exploiting and selling their practices for the outsiders' own financial and social gain and due to centuries of oppression and persecution from missionaries and state-sanctioned religious violence. Be mindful of this and seek to understand where others are coming from. Worshipping a God privately at home does not grant you automatic acceptance by them. Showing respect for their communities and being willing to learn may go a long way, but then again, it may not. The decision is theirs.

\hypertarget{what-about-being-chosen}{%
\subsection{What About Being Chosen?}\label{what-about-being-chosen}}

While I do believe in spiritual experiences, I \emph{don't} believe that chosenness is possible. What I \emph{do} believe is that many (or is it just us Americans?) use vocabulary from pop culture fantasy novels and Christianity because they don't quite know how to put what they have experienced into words.

When people are saying this genuinely and not deceptively, I believe this means that they experienced the spiritual version of ``friendship at first sight'' (which is a real thing in research). They saw a depiction of a God or started to learn about them and just \emph{knew} they needed to worship that God. It can be a very magnetic pull and generate intense, difficult-to-describe emotions. The person making this claim is still the agent. They are the one who decides to offer the libation or light the incense.

Sometimes, people will describe their spiritual life as a \emph{calling}, a term that is highly associated with Christianity in America, but that has since been secularized. It's often used in America to describe people who give everything they have to their jobs because the job gives them meaning. \emph{Calling} or \emph{called} are just a bizarre, culturally-specific words to describe that feeling of being excited about something, empowered to grow, and connected to something greater than oneself. I do not use this language, and in our multicultural society, it can lead to misunderstandings when someone doesn't have the tacit cultural knowledge of the term's origins.

\hypertarget{what-about-transformative-experiences}{%
\subsubsection{What About Transformative Experiences?}\label{what-about-transformative-experiences}}

Transformative experiences of the Gods are beautiful things. These can come in the form of dreams, experiences in journey meditations, sudden illumination from wisdom tradition texts, or even just seeing the way the sun strikes newly-opened spring flowers. In ritual, it is not unheard of for someone to be overcome with strong feelings of connection and love when praying to a God for the first time, as if a floodgate has opened. Prayers can lead to steady streams of coincidences. Divination outcomes can teach us about the parts of ourselves we need to grow or let go of, and the techniques themselves put us in contact with the God(s) to whom we have consecrated our tools.

In Platonism, the Gods have something called \emph{providential love} (eros pronoetikos) for us --- a type of care flowing without cease and without boundaries to all who are open to receiving it. Experiencing bliss or a sense of deep connection in prayer can sometimes be mistaken for chosenness or some kind of special status. Having a religious experience, cleaning oneself up, and committing to excellence can also sometimes lean on the idea of chosenness as a self-soothing mechanism --- we like to believe that someone else is rooting for us \emph{specifically}, and it's useful to unpack why we may believe that is the case. Maybe we actually just chose a life (go us!) where we will do a specific set of activities, and a spiritual experience was the jolt that motivated us to finally get done what we had already committed to doing.

If something like this happens, approach it with humility. Decompress however you need, but an experience does not indicate ``chosenness'' (although it may indicate your \emph{closeness} to a God, especially if that is what you have been asking about while praying). Continue to pray and explore your spiritual practice. Again, the practice evolves over time, as we are embodied souls moving through time.

\hypertarget{an-important-caveat-about-gods-functional-roles}{%
\subsection{An Important Caveat About Gods' Functional Roles}\label{an-important-caveat-about-gods-functional-roles}}

Many of us are initially drawn to a deity because they have a specific functional role. It establishes rapport between you and the God before you have first prayed to them, much like how reading a social media profile and learning that you have something in common with someone will prime you to treat them charitably, at least at first.

Often, someone may ask, ``so, if I have \emph{x} issue, which deity should I go to?'' after they already have a foundational Gods-oriented outlook and ritual shrine. We are conditioned by the media to oversimplify what divine epithets and spheres of influence of a deity mean. (An epithet is like ``Apollon the Far-Shooter'' or ``Apollon the Harmonizer of All'' or ``Apollon of Delphi'' --- it marks some specific aspect of a God, and people will often use epithets to praise a deity when starting out a prayer.) Aphrodite is \emph{only} a Goddess of Love, for example, or Eir is only a Goddess of Healing, according to that mindset.

Once we get to know a deity, however, these functional areas start to break down. Apollon is most associated with poetry, music, light, and harmony, and yet I might pray to him for assistance with a family issue. I have worshipped Apollon for decades and can easily access that space within myself, and within the ritual space, that is resonant with the God in a way that facilitates steadiness even when moving into uncharted zones.

In the Platonic tradition, every God is the totality of everything (``all in all'') while preserving their own unique individuality and oneness. Gods worshipped in a specific part of the world ``flow'' into conversation with cultures and ethnic groups to produce unique fingerprints of how Gods operate in specific places. A group may also be united by a similar outlook when they live very far apart as long as they are united by common narratives and philosophical outlooks. These relationships between groups and the Gods evolve, progressing forward in time. We are now living in an era when much of the world is alienated from Gods.

Syrianus, in critiquing Aristotle's \emph{Metaphysics}, Books 13-14, wrote that

\begin{quote}
{[}T{]}he whole placing of the stars involves much supposition on our part. Thus it is that the fixed stars are arranged in one way according to the Egyptians, but according to the Chaldaeans or the Greeks in another way.
(191,23-25)
\end{quote}

While Gods do not have positions in the sky, when we create traditions surrounding the Gods, and especially when those traditions become cultural forces, humans are pattern-finders. We arrange the Gods in different ways based on our contexts. We may divide up the sky differently, omit some types of stories and deities in favor of others, but the same \emph{potentiality} of divine multitude underlies all of this.

Sometimes, the functional role a deity has traditionally occupied is very important to us long after we've solidified who we worship the most. I have a soft spot for solar and light-related deities, and that comes out in my practice.

\hypertarget{a-chain-unbroken}{%
\section{A Chain, Unbroken}\label{a-chain-unbroken}}

In Platonism, there is one other type of divine relationship --- something that is not a choice and could never be a choice. In Platonic doctrine, every soul is in the \emph{series} of a specific God. A \emph{series} is a chain that goes from the God through the hierarchies of intermediary spirits, ultimately ending with each of us. Because this is an innate property of every soul, it is not special; because we are each individuals who express that individuality differently, every soul in a series has its unique, particular way in which it is linked to the God that seeds it. This doctrine calls those of us who hold it to probe the paradox of something as ordinary as grass being so individually important.

The following long quotation comes from Proclus during a discussion of people incarnating into lives that are like and unlike the divine series one is suspended from:

\begin{quote}
Even amid matters that seem difficult to understand or puzzling, the person who simply \emph{knows} takes the easy path to divine understanding (\emph{gnôsis}) --- retracing {[}a path that runs via{]} the divinely inspired cognition (\emph{entheos noêsis}) through which things become clear and familiar (\emph{gnôrimos}), for all things are in the gods. The one who has antecedently comprehended all things is able to fill others with his own understanding. This is precisely what Timaeus has done here when he refers us to the authority of the Theologians and the generation of the gods celebrated by them.

Who, then, are these people and what is the understanding (\emph{gnôsis}) that belongs to them? Well, in the first place, they are `offspring of the gods' and `clearly know their own parents.' They are offspring and children of the gods in as much as they conserve the form of the god who presides over them through their current way of life, for Apollonian souls are called `offspring and children of Apollo' when they choose a life that is prophetic or dedicated to mystic rites (\emph{telestikos bios}). These souls are called `children' of Apollo to the extent that they belong to this god in particular and are adapted to that series down here. By contrast, they are called offspring of Apollo because their present lifestyle displays them as such. All souls are therefore children of god, but not all of them have \emph{recognised} the gods whose children they are. Those who recognise {[}their leading gods{]} and choose a similar life are called `children of gods.' This is why Plato added the words `as they say,' for these souls {[}sc. those of the people to whose authority Timaeus proposes to defer{]} reveal the order from which they come --- as in the case of the Sibyl who delivered oracles from the moment of her birth or Heracles who appeared at his birth together with Demiurgic symbols. When souls of this sort revert upon their parents, they are filled by them with divinely inspired cognition \emph{(entheos noêsis}). Their understanding (\emph{gnôsis}) is a matter of divine possession since they are connected to the god through the divine light and {[}this sort of understanding{]} transcends all other {[}kinds of{]} understanding --- both that achieved through {[}reasoning through{]} what is likely (\emph{di' eikotôn}), as well as that which is demonstrative (\emph{apodeiktikos}). The former deals with nature and the universals that are in the particulars, while the latter deals with incorporeal essence (\emph{ousia}) and things that are objects of knowledge. But divinely inspired understanding alone is connected to the gods themselves.

Proclus, Commentary on Plato's Timaeus Book 4 (confusingly, Volume 5), trans. Baltzly, 159.14-160.12 --- all of the things in parentheses and brackets are from Baltzly
\end{quote}

As incarnating souls, we ``dip down'' into various types of lives, from rational lives (think humans, orcas, cetaceans) to nonrational lives (think incarnating as a cat). Above us are heroes, who have a dipping point of rational lives (so, they can't become cats), above them daimones and angels (who do not have dipping points into material bodies), and finally, the Gods themselves. Every God has a chain of these intermediaries. At the risk of sounding too spatial for something that is not demarcated by space, the God is the ``trunk'', the angels are the ``boughs'', and the various levels of daimones are the branches that we (leaves) are suspended from. Each daimon has a cluster of souls it supervises. In addition, specific lives we choose have presiding daimones.

When it comes to embodiment, we have a big challenge: our forgetfulness and our ability to get extremely preoccupied with what happens in specific incarnations or series of incarnations. We sometimes pick life-patterns that align with our leader-God, and sometimes we don't. This is one of the reasons I called each of us unique --- we make traces in the realm of coming-to-be like fingerprints. Some of our journeys express the tragic beauty of our leader-Gods, or the violent and horrific ones, or the depths of disappointment; others skim the surface of generation, like a bird of prey diving into the sea to come back with good nourishment. The story of Cassandra, for instance, is metaphorically about being given gifts and then turning away from a God who cannot be run from because he is already the core of her; there are just as many people given the same prophetic gifts and who are not believed, yet never turn from the source of their good. In this context, part of the maxim γνῶθι σεαυτόν (know thyself) is that we must strive to uncover this knowledge of who we truly are when we cast away all of those garments and to perform our current role in the most just way possible. This may not be easy regardless of how we make that approach.

\emph{Series} has both the technical meaning just described and a secondary, slippery one, and in its slippery sense, it refers to the God or Gods who preside over our current lifetimes, who may or may not be the same as the God we are suspended from, and Gods who are highly related to specific talents and skills that we have in this lifetime. In other words, it can be tough to figure out who our leader-God is if we approach the problem just based on what we're good at or the specific aspects of our daily grind.

\begin{quote}
But we should ask which of the aforementioned six types of essential daimons they say is allotted to each person. Well then, they say that those who live according to their own essence (kat'ousian) -- that is, as they were born to live (pephukasi) -- have the divine daimon allotted to them, and for this reason we can see that these people are held in high esteem in whatever walk of life they pursue (epitêdeuein). Now {[}to live{]} `according to essence' is to choose the life that befits the chain from which one is suspended: for example, {[}to live{]} the military life, if {[}one is suspended{]} from the {[}chain{]} of Ares; or the life of words and ideas (logikos), if from that of Hermes; or the healing or prophetic life, if from that of Apollo; or quite simply, as was said earlier, to live just as one was born to live.

But if someone sets before himself a life that is not according to his essence, but some other life that differs from this, and focuses in his undertakings on someone else's work -- they say that the intellective (noêros) {[}daimon{]} is allotted to this person, and for this reason, because he is doing someone else's work, he fails to hit the mark in some {[}instances{]}.

Olympiodorus, writing on our allotted daimon, at (20.4-15) of the \emph{Alcibiades I} commentary
\end{quote}

This confusion is not even a bad thing. In Ancient Greek art, there are beautiful scenes of Gods pouring libations to other Gods. Water is, among other things, a symbol of generation. We could view our incarnations that are lived away from our leader-Gods as libations, or gifts, for the other Gods --- it is a system of mutual honor. A God pouring a libation for another God, or doing sacrifice in general for another God, is a gift of participation in a portion of their divinity to another, mediated by this weird realm of generation where so many things are possible. We are each, at our core, a one, and the flower at the core of us is always connected to the God we are suspended from; we are, at our periphery, dazzled by and participating in the amalgam of divine delights, a reflection in water of the feast that the Gods share in the \emph{Phaedrus}.

In generation, the beautiful divine patterns --- each driven by a specific God --- ``freeze'' out. Like a frustrated spin glass, there is no configuration of the whole system that can be utterly stable. (Spin glass is made of charged particles, with north and south poles. As with other magnets, they want to align north with south poles, but the configuration of the system prevents them from attaining that --- shifting positions keeps putting them in conflict with yet another neighbor. Proclus describes this frustration in a different way in an essay on evils.) These conditions, and these phases of imperfect stability, evolve with time. We are locked in one position only to shift to another. We have that stability of who we are, but also that instability inherent in this environment. Some types of closeness with a God or Gods are driven by where we are in our life or lives; others are metastable over long periods; and only one relationship is absolutely intrinsic.

For those of us who hold that we belong to a God, we can add so much insecurity to this innate wonder of existence. We can also fall prey to our egos, propping up specific fragmentary experiences above the experience's actual worth. How many people have had a vision of a deity and propped themselves up as a New Age cult leader? How many have had a delusion, or the triggering of a psychological episode in need of treatment, that never gets treated because it is mistaken for something divine? How many have laminated and displayed a transformative spiritual experience in a way that has made them miserable due to their reliance on it as an object of reverence instead of taking it for the transient beauty it truly was? How many have mistaken a divine gift of insight into their own nature, a sacred unveiling of their soul through the power of the God's providential love, for being chosen by either the God at their core or the God ruling their present life or a God of affinity? How many have used personal experiences like this to engage in appropriation, spiritual materialism, and spiritual bypassing that take the wonder of what happened and turn it into something awful?

Always focus on the God, and be willing to evolve in how you relate to them. When we focus on the basics of prayer, ritual, and piety, we embrace the changing rhythms of our lives, which is in our nature as souls who are living a life in the material world. We allow ourselves to explore what our relationships to specific Gods mean without grasping tightly to something that will eventually fall away like sand through our fingertips.

In the last chapter, when I quoted from Plotinus before presenting the meditation exercise to you, there is a moment when Plotinus instructs us to ask for the God to come. This is the kind of contemplative practice that can guide you to who that leader-God is, if this practical Platonic section piques your interest. Repeat the meditation, or something similar to it, regularly over time, in conjunction with prayer. If you come into a state of frustration, wanting and desiring a definitive outcome or sign, sit with that feeling. Frustration often makes what we are frustrated about harder to accomplish.

If this does interest you, I end this section with these words. Feel free to contemplate them in meditation or to try that semi-Plotinian cosmic meditation again. Close your eyes, find that stillness, and the God who is intrinsic to you is there. This is not something available to the privileged few, but directly accessible to all. Every human soul, and I do mean each and every one of us, is a prayer \emph{continuously} without you even realizing it, just as a heliotrope follows the sun.

\hypertarget{appropriation-appreciation-and-cultural-reception}{%
\section{Appropriation, appreciation, and cultural reception}\label{appropriation-appreciation-and-cultural-reception}}

This chapter requires a section on appropriation, appreciation, and cultural reception (wow, that's a lot, isn't it?) because, in the 21st century, these are all important issues. When we worship a God, we want to know that we're doing it in a way that is ethical and grounded in a sensible response to our desire to worship them. We do not want to worship with a spirit of cultural voyeurism or appropriation.

The term ``cultural appropriation'' may already have your stomach tightening, but don't worry: I am not going to tell you that you cannot worship a deity or set of deities. What I \emph{am} going to say is that we all need to be aware of what, for lack of a better way of putting it, has been an extremely f---ed up past several hundred years. Appropriation is a way of discussing that power differential and the ways in which dehumanization and prejudice impact relations among people to this very day, especially where these relationships intersect with cultural ``objects'' and practices. Talking about appropriation doesn't always mean that you have to give something up --- sometimes, it means adding to what you do so you can ensure it is more sensitive, that it meets your needs while being respectful of others.

I have done cultural appropriation in the past. In my 20s, when worshipping Hellenic Gods, I participated in non-Greek communities that saw themselves as cultural continuations of Ancient Greek religious practices. It's embarrassing to admit in writing, especially now that I realize how antagonistic many of those groups were to actual modern Greek organizations doing the same revivalist practices for many Gods. The truth is that I am an American without Greek ancestry, and while the Greek Gods and motifs inspired by them have been part of American literature, art, and other cultural objects for centuries, many things are not the same. Nor do they have to be. We can look to what inspired creatives in the English-speaking world about specific Gods, myths, and stories, and we can do that without fetishizing a past that isn't ours to take. (The same goes for Americans interested in Vikings or ancient Germanic tribes.) Some aspects of a transmission can be awful. To continue with this example, bad aspects of the cultural transmission from Greece include the pilfering of Greek cultural artifacts after the Greek liberation from the Ottoman Empire, the way in which Greek heritage was taken to prop up white supremacy while denigrating modern Greeks as too ``ethnic'' because they failed to fit that narrative, and the callous humanitarian tragedy that happened when Western European powers failed to intervene in the Greek Genocide because we were hungry for trade with the Ottoman Empire's successor state. I can worship the Gods, but I need to ensure that I am speaking from my own context and that I am doing what I can about the bad things --- activities like supporting repatriation efforts and ensuring that I am mindful of modern Hellenic continuity with their past.

This is where we move into ``appreciation'': Behaving respectfully and aligning as much as possible with what must be done in it. Currently, South Asian Americans in yoga like Anusha Wijeyakumar are advocating for more cultural sensitivity and awareness --- one element of that is actually diving \emph{deeper} into yoga philosophy and learning about its Gods, like Shiva. If you do yoga, learning more about important Gods (and ways to honor them) from South Asian teachers is a great way to move from appropriation to appreciation. You can even start learning about the philosophy! I promise, as a follower of Plato, I won't judge you for it! Stoicism, which is appropriated by modern society as a self-help tool, also has Gods at its core, and it should be approached in a holistic and respectful way, too. Platonism is the same: It has many Gods, and only recently did scholars start to take that seriously.

At its most basic, be respectful of Gods and spiritual practices. Those of us who are learning about something as outsiders need to understand and embrace our outsider status for what it is. Two things I've found helpful to read are \href{https://www.theatlantic.com/entertainment/archive/2015/10/the-dos-and-donts-of-cultural-appropriation/411292/}{this article from \emph{the Atlantic}} and \href{https://abeautifulresistance.org/site/2021/6/03/plague-of-gods}{this piece on cultural appropriation when worshipping many Gods and how that is related to the treatment of everything as if it were a commodity item}.

Those of us who worship Gods of a culture we're not from likely came across those deities through cultural reception. Cultural reception, the way I am defining it, refers to how something that originated from elsewhere is received and interpreted by a culture, and especially to how the receiving culture processes its relationship to that thing. For example, the cultural reception of ancient Mediterranean literature and philosophy in the English-speaking world has historically been tied to elite education, especially to white landholders and old money families. There is a further cultural reception at play nowadays in America where people who have historically had \emph{less} access to the Classics are working with the material (as it was a privileged cultural corpus) and putting it in dialogue with other aspects of American culture. All of us in America, colonizers and colonized, are swimming in that unique amalgam. Something similar is happening with Latin American countries' interest in their own histories of Classical reception, although I don't know the details beyond a press release for an initiative that came across my social media feed a year ago.

People living in the Mediterranean may have very little knowledge about the role their ancient literature plays in our cultures, depending on how frequently (and deeply) they interact with us. Due to globalization and the Internet, there are many opportunities for this to create clashes --- it's one thing to see a news story from halfway around the world about a new art installation of Medusa and quite another thing to suddenly have a crash course in American politics in the Reddit comments about said installation. In a spiritual context, the way Wiccans relate to the Goddess Hekate is very different from how a Greek American community may worship her in Astoria, New York.

\hypertarget{syncretism-and-eclecticism}{%
\subsection{Syncretism and Eclecticism}\label{syncretism-and-eclecticism}}

Syncretism is what happens when you combine belief systems --- and sometimes Gods --- in a systematic way. To use a linguistics analogy, syncretism is a lot like what happens when speakers of multiple languages convene in a single location --- at first, there are communication issues, but a way to communicate quickly starts. Eventually, a creole language is produced, with its own vocabulary, grammar, and syntax drawn from all of the components that fed into it. It's very different from speaking multiple languages (the analogy for practicing multiple ways of worship at the same time, but keeping them separate). At some point, the new language is formalized and becomes an institution of its own.

Syncretism is often contrasted with eclecticism. Many eclectics value the relationships with each specific God and are not worried about connecting the dots in a systematic way like syncretics are. Eclectics may be characterized as people who don't care about systems at all, but this is rarely the case --- people who respect Gods put care into how they are worshipped, but that care may take different forms depending on what someone prioritizes. Eclectics are willing to tolerate ambiguity and juggle different systems; syncretics want to systematize everything and unite it into a whole. Many people are honestly somewhere in the middle.

Syncretic and eclectic practices are what most of us in America will end up doing --- and, if not us, our children. We come from many places, with a vast array of cultural inputs, and America is a melting pot. In addition, almost everything is Americanized (to some extent) eventually. What matters most is that we stay curious about the Gods and our practice. The world is full of Gods, as Thales said.

An example of a syncretic (and slightly eclectic) practice would be if someone were to map hearth Goddesses from their heritage onto physical directions and pray to them according to direction. Returning to what I said about worshipping four hearth Goddesses, I could hypothetically set Frigg as East/Air, Hestia as South/Fire, Brigando as West/Water, and Nantosuelta as North/Earth. This would be a nesting doll of syncretism. First, I am looking at Goddesses from four different contexts; second, I am combining them with directional quarters that are used in modern popular Wicca; and third, I am praying to all of the Goddesses at the same time using that directional context. The unifying elements here would be that these Goddesses represent both hearth Goddesses of my maternal and paternal families, plus the Hellenic Gods that I pray to out of fondness. It would efficiently get around questions of who to pray to (and when) by praying to them all at once.

Another example of a syncretic practice would be the fusion of Belesama with Minerva in Roman Gaul. Belesama is a solar Goddess, with strong water associations; she is wisdom-filled, good with words, and radiant. During that historical period, she was seen as the local version of Minerva by Roman authorities, and the worship of the two Goddesses fused together. Belesama and Minerva are distinct Goddesses. I would, based on personal contemplative experience, place Belesama in a similar conceptual space to Apollon, Aletheia, and Helios. Sulis, another Goddess who was syncretized by Romans to Minerva, was also worshipped as Minerva during the Roman occupation. She has more in common with Hygeia and Salus.

\hypertarget{exercise-do-some-research-then-pray}{%
\section{Exercise: Do Some Research, Then Pray}\label{exercise-do-some-research-then-pray}}

Choose one (or more) of the methods listed above for selecting deities. Use resources available to you to do some searching. It's OK to go on Wikipedia --- check the Talk page of whichever deity's page you land on if you see anything you're really unsure about. The Talk page can be accessed at a link just above the page title, and it shows you what editors have been arguing about while maintaining the page. Other options include looking up terms like ``Norse pantheon list of Gods'' in Google.

Narrow down to one or two deities. Do another search, but just for them. You might use ``-tumblr -twitter -pinterest'' to remove things from the results if you're using Google.

Here are a few questions to guide you: What kinds of offerings are typically given to this deity? What are their main symbols? Were they worshipped in any specific regions? What kinds of appearances does this God make in pop culture today? Are there any surviving ancient prayers that you could give them?

Once you have done your research, write down the God's name on a piece of paper and go to your small shrine space. After praying to the hearth Gods, pray to one of the Gods you have chosen. Say hello in whatever way comes most naturally to you and give an offering. The offering could be the same one that you gave to the hearth Gods or a different one.

If you know extemporaneous speaking isn't for you, you're welcome to prepare something written to read for the God during this first introductory prayer. You might give your name, the reason you are praying to this deity out of all of your other options, and provide some important information. What would you tell another person you've just met?

After praying, take a few moments to pause.

The information you've learned about the God is also useful for contemplative practices.

\hypertarget{purification}{%
\chapter{Purification}\label{purification}}

I pray after showering. For me, those habits are linked. Whenever I shower later in the day --- for instance, if I've returned from the gym --- I immediately think of the prayer beads at one of my shrines.

When I was just starting out with prayer, this is how my mornings would go:

\begin{enumerate}
\def\labelenumi{\arabic{enumi}.}
\tightlist
\item
  Wake up, feed cat, shower
\item
  Set the kettle on
\item
  Pray to the hearth Goddess and a God of the day
\item
  When the kettle went off, I'd end the prayer and get ready for work
\end{enumerate}

It was a simple routine --- the kind of thing a young professional juggling many obligations and getting adjusted to adulthood could manage.

But why did I do it after showering?

Honestly, it was to streamline the purification protocols.

\hypertarget{physical-purity}{%
\section{Physical Purity}\label{physical-purity}}

\emph{Ritual purity} can sound like a dour topic. The Gods are present everywhere and to everything, so why do we need to wash up? The answer has several components.

First, when we acknowledge the presence of the Gods, especially ones we hold dear, we are welcoming fond associates into our awareness. Like inviting someone over, someone might shower, put on clean clothes, and do some basic tidying. Another person could freshen up by washing hands, putting on fresh deodorant, and tidying one's hair. Writing after the pandemic, when so many more people are experiencing fatigue and related issues as Long COVID side effects, it's important to stress that you do what you can do. That said, you \emph{should} shower before a formal shrine prayer if you have just had sex, if you've returned from a funeral or memorial observance, or if you are getting over an illness that left you bedridden.

Second, as discussed in earlier chapters, when we pray to the Gods, we are engaging with both them and a plethora of symbols and signs that point to them. Symbolically, doing a purification is a ``stripping away'' of material garments and the excesses that harm our focus. We approach the Gods clean, with a clear ritual of separation between ourselves and all that we have bouncing around in our head.

Third, traditions that worship many Gods call for purifications in a variety of circumstances. \emph{The Soul's Inner Statues} is providing a generic practice, albeit one informed by the Platonists due to my bias. The principles in this chapter should give you a solid baseline for your household practice, regardless of who you call your hearth Goddess and which Gods you chose to get to know in the last chapter.

During any purification, prayers or intentions may be voiced (e.g., ``let me be pure'') to mark the action as sacred, but that is not required. Here are some general methods. This list is not exhaustive, so I encourage you to look up ritual purification and cleansing or detailed instructions on how to do one of these things if you need to.

If you decide to shift into a specific traditional practice due to fondness and inclination, you will doubtless learn techniques and protocols specific to whatever you investigate.

\begin{itemize}
\tightlist
\item
  \textbf{Smoke cleansing.} This is common in the United States nowadays. To do smoke cleansing, you light a bundle of herbs, extinguish it to embers, and wave the smoke around the area or person(s) you want to cleanse. (Remember fire safety.) Sage bundles are often available in stores; however, rosemary and other herbs can also be bundled and used for this purpose. Smoke cleansing may go by other names, like \emph{recels}, in specific cultural traditions. Please note that smoke cleansing and smudging are different. Smudging, while it comes from a term used in English for any thick purifying smoke (often, according to the Oxford English Dictionary, to ward off insects), via lexical drift, it has come to be the preferred term for the burning of dried substances in several Native American cultures' ceremonies.
\item
  \textbf{Taking a shower, doing a ritual bath, or washing up.} This could be as simple as the morning shower you do before getting ready for work, a washing of the hands and face in the evening, or a fully-drawn bath. In some traditions, ritually sanctified water is poured over someone.
\item
  \textbf{Salt water.} In this method, a dried fragrant plant (often rosemary or bay) is lit on fire and extinguished in saltwater immediately. This is then sprinkled on the person(s) participating in the prayer.
\end{itemize}

Remember that it is possible to modify any of the above purification techniques if need be. Do what you normally do to feel presentable.

\hypertarget{underworld}{%
\subsection{A Brief Word on the Underworld Gods and Ancestors}\label{underworld}}

Depending on context, deities related to the underworld --- and ancestors --- may need to be worshipped separately. This is strictly the case in contexts with the Hellenic Gods, when it's important to \emph{never} share a drink or food with underworld deities like Hades --- they get all of it --- and where the offering place to the underworld Gods is separate.

It's a best practice to Google how an underworld deity is meant to be worshipped before you do it. If, during your research, you learn that the deity you are worshipping has \emph{strict} separation (like in the Hellenic example above), I recommend using separate libation vessels and receptacles. Some traditions will allow ancestors' photos to be placed at or near the shrine for non-underworld Gods; in others, it's best to keep any ancestor shrines separate. A few years ago, when researching whether I could have my ancestors at the same shrine as my ancestral household Gods, I learned that in many ritual liturgies for Norse Gods, ancestors are toasted in the same ritual as Gods, and many practitioners put them together on shrines.

``Separation'' is relative. Use a low shelf or a place on the floor that you put up and take down as needed, removing the items to a box when not in use. One could use a wood yoga block as a floor table to hold the ritual items, a cutting board with raised feet (IKEA has one), or even the storage box (if it's sturdy enough).

At my main household shrine, which is not for the Hellenic Gods, I do not separate out ancestors --- but I use different incense burners and candles. I pray to houeshold Goddesses and ancestors during the final few days of the lunar calendar. At a small meditation table that is low to the ground, I have a shrine for underworld Gods: I worship the Erinyes, Hades, Persephone, Hel, and Erecura. If I didn't have a history of worshipping the Erinyes, I would just do a pop-up shrine for specific underworld Gods as needed, so don't take me as a model on this.

The waning moon into the dark moon is an excellent time to connect with underworld Gods and ancestors.

\hypertarget{mental-purity}{%
\section{Mental Purity}\label{mental-purity}}

When approaching a shrine space, mental purity is just as important. Here, we are defining mental purity as \emph{being able to focus on the ritual and on the God(s)} --- nothing more, nothing less. Iamblichus, when reporting on Pythagorean precepts in the \emph{Life of Pythagoras}, wrote, ``In going to a temple, it is not proper to turn out of the way; for divinity should not be worshipped in a careless manner.''

Sometimes, mental purity is very easy. Sometimes, calming one's mind is like trying to calm spooked horses.

There are times when I've imprudently checked email before my morning prayers and have been slammed with every awful thing in the world or a to-do list item that sets me into a panic. It can be a rainy day, or a day with horrible weather waiting for me, and I am already tense just looking outside and thinking of how on Earth I am going to get to work without slipping on ice. Rarely, arguments or bad life experiences can linger for days, even weeks. We all have things going on in our lives.

What I find useful during those charged days is to slow down. I try --- often unsuccessfully, but we're all works in progress --- to avoid checking social media before praying so I don't see things that make my stomach churn. I do a two-minute mindfulness meditation with an app to wind myself down.

It can be useful to have backup plans for when mental disquiet happens to you: Slowing down, you will not have as much time, which could stress you out even more if you haven't planned ahead. We all know that things will disrupt our schedules sometimes. Plan for the inevitable --- pick a mindfulness technique to use in times of duress, and plan out different lengths of your rituals.

Some options for mindfulness techniques were listed in Chapter One \textgreater{} Contemplative Practices \textgreater{} Foundations.

Non-meditation mental purification techniques I recommend include:

\begin{itemize}
\tightlist
\item
  \textbf{Take a social media break.} Pick one or two days each week to uninstall and block all of your social media apps --- it works best if the days are always the same. Ensure that anyone who absolutely needs to speak to you has your contact information, and plan on what you will do while disconnected. If you are very online, this will not feel pleasant due to the dopamine withdrawal, so plan a few more self-care activities than you think you need. If you have a tough time following through, I found great success in changing my password and putting the only record of the new password beneath an icon of Athene.
\item
  \textbf{Write out your to-do lists or do a thought dump.} This is especially useful for anyone who wakes up with a sudden awareness of everything that needs to happen that day. Use paper or your favorite app.
\item
  \textbf{Have some self-care techniques handy.} These can be small things like doing a mud mask, a sequence of stretches, or diving into a folder you have of heartwarming letters from friends. You could also listen to chapters from a self-care audiobook. One book I've appreciated so far this year is \emph{Pause, Rest, Be} by Octavia F. Raheem.
\item
  \textbf{Listen to music on an upbeat playlist, preferably without lyrics.} It's harder to get earworms when you're listening to instrumental music (and, speaking from experience, it can be very distracting to pray with a song stuck in one's head!), and music has such power to restore mood. You can find happy and upbeat playlists that feature jazz, classical, ambient, or whichever genre you love.
\item
  \textbf{Get enough sleep.} Sleep power-washes your brain and makes you ready for the next day. Do what you need to do to get to bed earlier.
\end{itemize}

\hypertarget{exercise-what-are-your-backup-rituals}{%
\subsection{Exercise: What are your backup rituals?}\label{exercise-what-are-your-backup-rituals}}

This exercise builds on the idea of habit-bundling from previous chapters. Most of you are now worshipping the Gods in the morning or evening, with prayer added onto your other habits --- like praying after you shower or after you brush your teeth.

Take out some post-it notes or pen and paper or open a digital whiteboard app like Mural. Depending on the time of day when you want to pray, write down what you tend to do at each step of your morning or evening for five scenarios:

\begin{enumerate}
\def\labelenumi{\arabic{enumi}.}
\tightlist
\item
  You wake up late (or, for evening people, when you get home late).
\item
  You have lower energy than usual and need something simple.
\item
  You're traveling and are in a hotel or a guest in someone else's home.
\item
  People are visiting and you have to juggle hosting responsibilities.
\item
  You are experiencing high levels of stress or anxiety and need to focus on contemplative and de-stress techniques.
\end{enumerate}

The ritual you've already been working on is your core ritual --- the practice you can come back to as your ``home base'' when praying to Gods, divinities, and ancestors. Creating your backup rituals is an opportunity to embrace realism while maintaining connection.

Decide how much time you have for taking pause with the Gods in each of these scenarios. 30 seconds? Two minutes? Five? Ten? Write down exactly what you will do and make sure your decisions are written out on notecards (or in an easily-accessed digital note-taking app) so you have them handy.

For the next four days, try each of the rituals. Make note of any adjustments.

\begin{itemize}
\tightlist
\item
  What feels satisfying?
\item
  What is dissatisfying?
\item
  Are you being realistic about the amount of time you have, the interruptions you may face, and elements of your life that always derail you when you're under pressure?
\item
  If you're sleepy or groggy, should you really do that meditation?
\item
  What is the bare minimum physical purification you can commit to doing?
\end{itemize}

If you travel frequently, this is also an opportunity to set up a travel kit. Mine is simple: I found a small jar and bowl at Goodwill. I bring prayer beads for several Gods, and I have wood bookmarks of Athene and Apollon that I put in place card holders (the wood name card holders that you may have seen at weddings on the table). If I'm in a hotel, I use that. If not in a hotel, I spend a few minutes with prayer beads. One time, while staying with an acquaintance in a small apartment that had very little privacy, I prayed in the brief few minutes they were in the shower.

\hypertarget{the-kind-of-consecration-you-dont-want}{%
\section{The Kind of Consecration You Don't Want}\label{the-kind-of-consecration-you-dont-want}}

There is a concept for something in Ancient Greek Religion, \emph{agos} and \emph{enagēs}, that is worth mentioning here --- while the concepts are culturally specific, they describe something that I would argue is universal, and especially poignant for those of us living in former colonies.

Conceptually, this term usually refers to someone doing something so wrong that it \emph{consecrates them to a deity or divinity}, but in the worst way possible. The Gods are wholly Good, so they do not punish us. Behaviors that we engage in can, however, lead us down roads that will burn us because that is what is statistically likely and necessary in those circumstances. This term also has a \emph{positive} sense in cases where it refers to the effect of swearing an oath --- one swears by the God(s), and the action is consecrated to whichever God(s) were mentioned in the oath. This sets a boundary of necessity on one's behavior.

In the negative sense, the behavior that consecrates one to a God --- and badly --- is the inverse of what we will cover in the chapter on virtue. I once had a coworker who told me that she was a terrible person and that I wouldn't believe the things she'd done. I asked her if she had ever murdered or raped anyone, she said no, and that was the end of that. We often experience social shame and similar states as totalizing, catastrophic things. It is a growth process to learn to react appropriately to our own flaws. Most of us have not \emph{personally} done truly terrible things.

I associate the negative necessity-driven consecration most strongly with the hostility of land spirits and the dead in places that have been the sites of genocide and similar blood crimes. For example, the Sullivan Campaign in Upstate New York massacred Native Americans to make more land available to Revolutionary War heroes, and the heinous act cannot be washed away. Among the Hellenic Gods, it's usually Goddesses called the Erinyes (Alekto, Tisiphone, and Megaira), translated as Furies, who demand appeasement once blood crimes occur. The effects of this state cannot be washed away with a simple purification; instead, cleansing demands work. Those of us taking actions to make things better --- even if what we can do is small --- are far better off than people embracing the outcomes of unjust acts.

\begin{quote}
But how is it reasonable for descendants to pay the penalty for {[}the sins of{]} their forebears? Well, chiefly {[}because{]} they have inherited their estates and their gold and silver, often acquired by wrongful means, which is enough for them to incur a penalty. And then too the souls of the forbears suffer along with those of the descendants that are having a difficult time {[}here below{]}. And they {[}\emph{sc.} the descendants{]} do not suffer {[}these{]} punishments undeservedly, for the person who deserves to suffer such things is led into that kind of family, since providence and the divine nature and the gods who are the guides of fate transcendentally weave all things together in order and in accordance with justice.

\emph{Hermias: On Plato Phaedrus 227A-245E}, trans. Baltzly \& Share, 2018, 101,10-20
\end{quote}

Essentially, due to reincarnation, souls will choose the kinds of remedies they need to improve in their next life. Everyone who inherited difficult immoral legacies has the opportunity to improve things now.

Do what you can, and make sure you're not taking the spotlight for simply being a decent person. There are many platforms trying to lure us to warp our efforts to help into content for our ``audiences'' to show what great people we are, all the while ad revenue is rolling into the platform due to our engagement. It is equally easy to overextend ourselves due to the magnitude of human suffering and horrible behavior around us.

\hypertarget{guidelines-for-purification-in-the-everyday}{%
\section{Guidelines for Purification in the Everyday}\label{guidelines-for-purification-in-the-everyday}}

The guidelines below are ones that I am providing as useful suggestions for anyone praying to Gods. If you decide to pursue learning a specific tradition, you might find it has different protocols, and that's okay --- the point is having them.

\begin{itemize}
\tightlist
\item
  \textbf{Mourning:} wait until the new moon to pray again, and cover the shrine. You can still pray away from it and pray to ancestors. If a close relative dies, especially in the final few days of the lunar cycle, I recommend waiting a full lunar cycle and beginning again at the new moon after that. The night before you resume your ordinary prayer routine, put a photo of your new ancestor on the shrine where you keep ancestor-related items and do something meaningful for them.
\item
  \textbf{Birth/adoption/fostering:} Everyone showers or washes up. Put on nice clothes and introduce the new member of the family to the household Gods. Eat whatever foods you find special and offer some of them to the Gods.
\item
  \textbf{Menstruation:} Be respectful if you are in a space where there are menstruant taboos (e.g., Kemetism, Hinduism), but otherwise, change your products and do not approach the shrine if you are experiencing debilitating pain. If you are in that much pain, you need to rest.
\item
  \textbf{Sex:} Clean yourself up before you pray.
\item
  \textbf{Sickness:}

  \begin{itemize}
  \tightlist
  \item
    If you are having an acute sickness, do not pray if you cannot get up to do ordinary household things.
  \item
    If you have a chronic illness and are having a flare-up, either refrain from praying or implement one of your backup rituals. I recommend thinking about your backup rituals in terms of having a low-energy prayer routine, medium-energy prayer routine, and high-energy prayer routine so your practice can fit the rhythm of the condition you are managing. It is OK if your lowest-energy routine is you in bed murmuring a mantra under your breath.
  \item
    If you are asymptomatic or mildly symptomatic and can do normal activities, try your full or low-energy backup ritual. If something doesn't feel right, focus on recovering. As mentioned in the bullet point above, there are ways to bring moments of reverence into your day without overextending yourself.
  \item
    During any sickness, you may pray to healing deities, either formally or informally, or ask that someone else do so on your behalf.
  \end{itemize}
\item
  \textbf{Extreme stress:} Focus on your backup ritual. Make sure you do whatever is in your emotional first aid kit.
\end{itemize}

\hypertarget{exercise-pull-it-all-together}{%
\section{Exercise: Pull It All Together}\label{exercise-pull-it-all-together}}

Do a brief meditation or grounding and centering exercise, then pray to a household God, ancestors, or one of the Gods you have picked after the previous chapter. See how taking pause impacted you. The next day, try a different type of purification exercise.

Do this for a few days while being mindful of the purification's impact on you. Pick the one or two options that seem to work best and stick with them. You can always revisit this if need be.

\hypertarget{ritual-mechanics}{%
\chapter{The Mechanics of Ritual}\label{ritual-mechanics}}

This chapter is about the mechanics of devotional activity --- or, building on the spiritual habits you have already been engaging in to set yourself up for long-term success. Human lives are not the same from season to season, year to year, and decade to decade. The stars and planets trace out their orbits. We change jobs. We marry. We achieve some things and fail at others. We die.

Within our human lifetimes, there are times when we all pray for things we need in our lives: success at work, a new job, the satisfactory resolution of difficult interpersonal interactions, a better apartment, the ability to get to a flight on time.

We also pray according to seasons: when new herbs we are growing from seed first send up their shoots, when the ice and snow come and recede, for a gentle landfall of the hurricanes pummelling our shores during hurricane season. We pray for good harvests, happy new years, and abundant gift-giving seasons.

This chapter discusses how to incorporate these occasions into your practice. We will begin discussing the lunar and solar cycles, then get into specific types of cycles that happen in our lifetime --- age milestones, deaths, and the inevitabilities of our lives. First, though, let's do a framing exercise.

\hypertarget{do-what-is-achievable}{%
\section{Do What Is Achievable}\label{do-what-is-achievable}}

Throughout this chapter, you may have ideas. These could be about ways to honor sacred rhythms of time, be it through decorating your home, cooking special foods, or simply special types of rituals you want to do.

Some things to keep in mind while reading are:

\begin{enumerate}
\def\labelenumi{\arabic{enumi}.}
\tightlist
\item
  Are other members of my household or roommate situation going to participate? Do I have anyone locally who does similar things and with whom I could collaborate on lunar holidays?
\item
  How much time do I have? The civic calendar does not give us these days off. We spent much of Chapter 3 designing our plans for success even in adverse situations, and the same thing applies here.
\item
  How am I going to keep track of when to get things ready? If you want to have flowers at the solstices or switch out wreaths, factor in shipping times when you set up your reminder system. I use Google Keep for reminders because the notifications on my phone contain useful information without too many irrelevant elements, and my solstice reminders are set for November 30 and May 31 (about three weeks in advance). You can also set reminders in a system like Google Keep for your grocery list with a prompt to check whether any special days are coming up while you are planning your shopping.
\item
  Who do I want to worship? There are many deities who are related to the Sun, Moon, and seasonal cycles. The same ``selecting deities'' criteria apply here that applied to picking your first Gods to worship back in Chapter 2.
\item
  What is my ``special occasions'' religious budget? We all have different life situations, so (a) be realistic about what you can afford and (b) it's better to have less for special occasions, but have it come from a more ethical source, than to buy a large quantity of things from a less ethical one. The Gods do not care how much is in our paycheck or how many assets we have. They care about what is offered from the heart.
\end{enumerate}

You can react to these questions in a notebook, a note on your phone, or something in your audio memo app. Practical work like this may seem like I am dragging sublime, transcendent activities through deep mud, but plans make all of the difference when you are having a hectic week at work, a bad few nights' sleep, \&c.~Speaking from personal experience, it means the difference between decision paralysis (when we often end up doing nothing and feel disappointed in ourselves) and actually meeting the demands of our messy lives (when we feel relieved that we know ourselves well enough to have cut ourselves some slack).

With your own life context now in mind, keep your note-taking tool handy and think of what makes sense for you.

➡️ This is also a perfect time to revisit the time-blocking activity you did in Chapter 1 (or start it if you haven't done it). Is it still working out for you? After reading this chapter, are there things you want to shift? ⬅️

\hypertarget{following-the-sun-and-moon}{%
\section{Following the Sun and Moon}\label{following-the-sun-and-moon}}

In Proclus' \emph{Timaeus} commentary, the philosopher points at the divine presence within things as seemingly intangible as time durations and motions of the heavens:

\begin{quote}
There is, of course, a parallel with the sacred tradition which worships the former invisible {[}numbers{]} that are the causes of these {[}visible ones{]} by naming Night and Day as gods, as well as by delivering those things that commend one to the month and the year, the invocations and self-manifestations. These things are considered not as things to be totted up on one's fingers, but rather as among the things that have divine subsistence --- things which the sacred laws of those who serve as priests command us to worship and honour by means of statues and sacrifices. The oracles of Apollo also confirm this, as the stories say, and when these things were honoured, the benefits that result from the periods belonged to human beings, both the benefits of the seasons and those of other {[}periods{]} similarly. However, when these things were neglected a condition contrary to nature was the result for everything around the Earth. Not only that, but Plato himself in the Laws (X 899b2) positively shouts out that all these things are gods: seasons, months and years --- just like the stars and the Sun. We are introducing no sort of innovation when we say that it is worthwhile to conceive of the invisible powers that are prior to these visible things {[}as gods{]}. So much for these matters.

Proclus, Commentary on Plato's Timaeus, Book 4 (volume 5), 89.15-37, trans. Baltzly
\end{quote}

We can use this as our starting point for discussing how to build veneration of the moon and sun into your practice. A noteworthy omission is the astrological rhythms --- some people who worship Gods focus on when the Sun, Moon, and other celestial bodies are in different astrological positions.

\hypertarget{venerating-the-moon}{%
\subsection{Venerating the Moon}\label{venerating-the-moon}}

This section focuses on four main types of moon veneration: the rhythms of waxing and waning, moonrise, moonset, and eclipses. It is a rare person who venerates the moon in all of these ways --- people who have that kind of time are, generally speaking, devoted to spiritual lives to an extent beyond what is attainable for most people. I do not follow moonrise and moonset, for example, nor do I generally follow the eclipses unless they happen when I am awake.

The moon has, in many places, been seen as a feminine symbol related to fertility and the process of generation. Writers have long pointed out the relationship between its length and the length of menstrual cycles --- as one of the few species of animal that has these, we are tied to swift cycles in a way that few other animals are. In other cultures, the moon and its God have been seen as masculine --- hence the ``man in the moon'' motif common in Anglosphere cultures, the God Mani of the Nordic tradition, the God Suen of Harran, and so on. In Platonism, everything below the moon's orbit is called the ``sublunary realm'' --- while, in 2022, we no longer profess a geocentric worldview, we can take this as a symbol similar to what Carl Sagan says when discussing the ``pale blue dot'' and what others say about the slim sliver of atmosphere separating us from the void beyond. The moon is our closest companion, responsible for tides and an assistant to animals from afar. While not falling prey to the worst of the realm of generation, and achieving union with the One and ascending into the place beyond generation are goals within Platonism, only a few (I'm thinking of Plotinus) are on the record as shying away from lunar ceremonies. Marinus, in the \emph{Life of Proclus}, communicates how the divine successor of Plato, on arriving at the Platonic school in Athens, honored the moon on seeing her in the sky. This action convinced the school's leaders of his piety when they had previously not known what to make of the newcomer.

Within the lunar cycle, the definitions I am about to use are slightly different from what you will see when you look at a wall calendar. In many calendars form the ancient world, including the one I have used for most of my activities, the ritual month begins the day after the lightless moon --- that first slim sliver is called the new moon. On your wall calendar, or with whichever moon phase extension you've added to Google Calendar or Outlook, the calendar will refer to the dark/lightless moon as the new moon, following modern astronomy's conventions. Twenty years ago, most Earth-based spirituality books I read in my teens also equated the dark and new moon, usually to refer to the lightless/dark moon. The dark moon, new moon, and full moon are the major times at which someone might venerate this cycle.

\textbf{The dark moon} is a perfect time to pray to one's ancestors. If you are living near where your family is buried, visiting the gravesite to offer flowers and clean up the graves would be a lovely gesture. It is a time to pray to apotropaic Goddesses. Many such Goddesses preside over the paradoxes of goods and evils we find in the world around us. Hekate, for example, is both the Goddess who rules the spirits of the material worlds and the salvation from them; Eris, who is most associated with the apple incident that started the Trojan War, is both a Goddess of discord as well as a Goddess who disrupts difficult situations so they can come to a useful resolution. You may also pray to underworld deities at this time. The dark moon, or in fact any time within the final few days of a lunar cycle, is a great time to clean one's home and donate those boxes you've been meaning to take care of.

\textbf{The new moon} is a time of beginnings. Its first sliver is a day to mark as a special occasion for your household Gods, the lunar deities, and any special Gods of affinity you may have. Dust the shrine, give an offering slightly fancier than usual, and cook something nice for yourself (and your family). If you make a dessert, offer some to the Gods. Debrief, via freewriting, how your last month has gone and what you could do in the month to come. Talk to the Gods about your plans and pray that you can take the lessons of the past month forward in the best way possible.

Specific traditions may mark the new lunar month (either on the dark or new moon) with special rituals for specific Gods. If you seek out training and guidance in any of these traditions, expect that your lunar observances may change.

\textbf{The full moon}, depending on one's tradition, may or may not be emphasized. Many current and historical holidays for Gods will fall in the days leading up to or just after the full moon --- but then again, there are often holidays at other times, too. As with the other two times of the month, consider how \emph{you} want to honor the lunar rhythm.

I give offerings to lunar deities, do a star/celestial grounding and centering practice, and drink water that has symbolically been imbued with moonlight. Someone I know (who is spiritual but not interested in praying to deities) hosted monthly full moon circles with floor cushions, soft music, tea, and the recitation of creative works at one of the local Unitarian Societies.

\textbf{Moonrise} and \textbf{moonset} are not often emphasized outside of niche ceremonial practices. Usually, you will need to know these times for determining when the moon is in the sky during your rituals. For example, I know that if I do a full moon ritual past 8:30 PM or so, it will not be visible through my east-facing window when I pray. I also know that, despite the beautiful ambiance of candles at night, that the new moon rises and sets with the sun. The \href{https://www.timeanddate.com/moon/}{Time and Date website} is the best tool to use for this. When I notice the moon in the sky, day or night, I often take a brief pause and say hi.

If a lunar or solar \textbf{eclipse} is visible in your area, it will happen on the full moon (the lunar eclipse) or the dark moon (the solar eclipse). The swallowing of the sun and darkening of the moon are natural effects of the dynamics of the Earth, Moon, and Sun. While I try to see eclipses when they're visible, I don't incorporate them into my rituals. Some religious traditions encourage fasting and purification activities during eclipses.

\hypertarget{venerating-the-sun}{%
\subsection{Venerating the Sun}\label{venerating-the-sun}}

Both the Sun and Moon have Gods and Goddesses, but masculine solar deities are more well-known to people --- Gods like Helios, the Unconquerable Sun, Surya, or Ra. However, many cultures have had (or still do honor) a solar Goddess --- Belesama, Sul, Sunna, Amaterasu, and others. In my estimation, whether a culture adopts a solar Goddess or God is correlated (with some exceptions) with latitude on Earth. The closer to the equator, or the more blisteringly dry an area's summers, the harsher the sunlight; the farther from it, the more marked its softer impact is on fertility and fecundity. If we think back to what was written in Chapter 2, ``An Important Caveat About Gods' Functional Roles,'' these distinctions likely evolved on the cultural level because the opening for that specific deity to express a relationship in solar terms was available locally.

We will focus on three types of veneration: the solstices and equinoxes, sunrise, and sunset. We already discussed eclipses when treating the moon.

The \textbf{solstices} only happen twice every calendar year. If you are looking for a replacement for other joyous, complicated holidays, celestial mechanics has gifted us with two important days six months apart. The time of greatest darkness and the time of greatest light have different kinds of cultural associations. One of them welcomes in the season of winter, and the other welcomes summer. The beginning of winter is the start of a season of rest and incubation, prefacing the abundant outburst of spring. The beginning of summer starts off the heat and its calamities, but also prefaces the upcoming harvest season. Praying to the sun at dawn or sunset on these days --- or having a complicated meal with your family --- is also accessible to diverse and complicated families with a variety of spiritual beliefs.

For the solistices, consider rotating out any wreaths, dried floral décor, reed diffuser sticks, and other ambient things you have around the home. Going into the solstice with a clean, refreshed home is a beautiful thing. I dedicate a new laurel wreath to Apollon and hang it near my entryway on the winter solstice. Seasonal decorations can be taken out of storage and hung to celebrate. Ambiently, many Americans are used to doing this in December, but less accustomed to it in June; it may take some time for you to acclimate to a summer-season mindset.

The winter solstice is an excellent time to pray to a solar deity, your patrons, and household Gods for good fortune in the season to come. If you can, keep a light-vigil (but be mindful of fire safety if you're using a flame) on the darkest night of the year and welcome the new sun as it rises, heralding the season of renewal. In many traditions, this is a time of gift-giving.

On the summer solstice, pray to the solar deity, your patrons, and household Gods. Summer-solstice prayers can be more apotropaic (meaning, averting ill fortune and calamity). The height of the sun's energy heralds the beginning, or increase, of stormy seasons, as there is a delay between heating and environmental effects. You can also start a gift-giving tradition on the summer solstice.

The \textbf{equinoxes} in the spring and autumn are dedicated to the renewal of life and harvest season/dying season respectively. In the spring, you can pray to your household and patron deities for success on your new projects; in autumn, you give gratitude for what the world has brought, be it fruit or life lessons. Of course, there are some exceptions to the ``death and the autumnal equinox'' thing. Some cultures celebrate the dead during the dog days of Summer or in the latter part of winter. For those of us in the United States, Halloween and All Soul's Day happen just over a month after the autumnal equinox.

\textbf{Sunrise} is most famous for being the time when Pythagoreans greeted the Sun and did their prayers in antiquity. It is an auspicious time for prayer in other traditions, too. During fall and winter, sunrise happens at a time when I'm awake, so I take a few moments to pause and breathe with my eyes closed while facing it once I notice that it has cleared the building across the street, at least on days when the sky is clear. Even this brief pause makes me feel connected to the solar Gods. Sunrise is a good time for reciting chants, giving simple libations, or contemplating a passage from a myth, philosophical text, or poem.

\textbf{Sunset} is trickier. For many of us, it happens during our commute. Sunset, or prayers after sunset when you are preparing for bed, are good times to wind down for the night. This is a time that has traditionally been used for journaling and freewriting, so it's useful for doing that, too.

\hypertarget{life-milestones}{%
\section{Life Milestones}\label{life-milestones}}

At least in the United States, the ``milestones'' by which we measure adulthood's seasons have changed dramatically over the past three generations. Rather than progressing from childhood to school to marriage to parenthood to retirement, the length of educational training and the prevailing circumstnaces around us mean that we might miss, or need to reinterpret, many adult milestones that were once certainties.

\textbf{Achieving a work or career milestones.} Celebrate with an offering to your professional God(s). Do something special that expresses gratitude for your success. You can also pray to the Gods when you finalize your job goals for the next year.

\textbf{Bringing new family into your home.} If you have given birth or have adopted a child, or if you have a young relative who needs you to be their guardian, taking a few moments to introduce them to the Gods at the household shrine can be a meaningful way of instilling a feeling of belonging. This can include a purificatory aspect, as per the last chapter, to invite good luck and ward off harm.

\textbf{Bringing in a new pet.} While I consider my cat to be my fur baby, pets are not the same as caring for a developing young human being. You can pray to the household Gods and to your patron deity or you can incorporate any deities who have historically had a role in animal husbandry.

\textbf{Naming ceremonies.} If you have named a child, or if someone in your family has taken a new name for any reason, a naming ceremony where Gods and ancestors are invited to give their blessings is a great way to formally mark the occasion.

\textbf{Management of acute and chronic illnesses.} Whether you are sick for a few weeks or have a long-term illness with flare-ups, marking the times when you get better with a special ritual can be a great ``welcome back'' event. I was sick with a bad flu in February 2020 and, after being in bed for 10 days, the first day I was able to light candles at my shrine and pray felt so meaningful. It didn't matter that I had to sit down to gather my strength halfway through. I was there, and I had made it.

\textbf{Weddings and commitment ceremonies.} Most people making commitments to a life partner do so in the context of marriage, but that is not the only option --- many do commitments instead because marriage comes with negative financial impacts in the form of welfare cliffs and poorly-designed benefits infrastructure. Introducing your partner to the household Gods, sharing food with them, and purifying your living space can be incorporated into the ceremony, privately or as part of the larger ritual event.

\textbf{Pregnancy and childbirth.} You can pray for those who are pregnant or giving birth. If you are pregnant, you may consider altering your prayer ritual to incorporate offerings for the health and wellness of the future human you are growing in your lower abdomen. If you are actively trying to become pregnant or impregnate someone, incorporate fertility prayers and vent to the Gods about any hopes and fears.

\textbf{Funerals.} The purpose of a funeral is to give the body the rites that properly ``send off'' the deceased. They are an opportunity to show respect and care --- and they should always be done according to the belief system that the person had in life, not what we wish it would have been. I grew up in a conservative area where it was not uncommon for funerals to include a spontaneous plea for religious conversion to the ``correct'' sect of the prevailing faith, or ones held by people who wanted to bring the deceased back into their faith after apostasy. The deceased was not placed at the center of the ritual, which was horrible. Likewise, when visiting burial plots, stick to making offerings of food, drinks, and flowers that the deceased may have enjoyed.

\textbf{Death.} Unlike funerals, honoring a death can be done alone. When one of my coworkers died unexpectedly in late 2021, I lit a candle for her the evening I heard and said a few words. As someone who cries easily, I was too afraid of not holding it together to say anything at her vigil, but privately, with a candle burning and my words directed at her, the tears were not as disruptive. I included her in the names of the deceased when I performed an annual ritual for the Chthonic Gods that February, about two months after her death. Anyone can be honored in this way.

\textbf{Days of the week.} In English, Monday, Tuesday, Wednesday, Thursday, Friday, Saturday, and Sunday \href{https://www.merriam-webster.com/words-at-play/saturday-special-word-history}{are related to specific Gods}. In the Romance languages, the days of the week also come from Gods. Honoring the God associated with the etymology is a great way to tune into the rhythms of the year.

\textbf{Celestial clockwork.} Stars rise, reach their highest point, and set. Planets do, too --- and they appear to go through a specific set of constellations along the ecliptic, the Sun's apparent path. Some are fond of tracing the motions of a specific constellation.

It is important to approach prayer as a way of imbuing everything we do with what is sacred --- of being active agents in our experience of spirituality and completing the chain that begins with the Gods and emanates into matter. All things pray except the First, as Proclus wrote; we are all praying in some capacity continuously, and continuous with our actions. What matters is becoming mindfully aware and using our rational faculties to steer our own ships under the guidance of the Gods.

\hypertarget{the-mechanics-of-a-prayer}{%
\section{The Mechanics of a Prayer}\label{the-mechanics-of-a-prayer}}

In previous chapters, we have focused more on getting you started with prayer than on the mechanics. This chapter has opened the floodgates of possibility, linking that private practice in your home with everything from the rhythms of your life to the cycles of the sun and moon above. If you feel overwhelmed, that is OK: the world is full of Gods, and every moment is teeming with deities, too.

It is so easy to get carried away, so let's take a step back. Prayer is a rhythmic practice through which we connect to the Gods. That active connection and engagement with them is both ordinary and transcendent. If we try to force an ``end goal'' instead of focusing on the rhythm of our actions and words for the God, we can quickly get in our own way and create stress and friction within the body, mind, and soul. Sometimes, people communicate excitedly about dreams filled with divine potency, guided meditations that lead to new and meaningful information, or Gods experienced, but their experiences are not the everyday. Life involves an awful lot of lines and waiting rooms. We can also let our expectations get out of hand.

A prayer includes five basic elements:

\begin{enumerate}
\def\labelenumi{\arabic{enumi}.}
\tightlist
\item
  An invocation. You greet the God by their name, adding any epithets or titles.
\item
  An offering. Offerings anchor the prayer in the world around us, opening space for connection to the God.
\item
  Say some words, chant, read a hymn, or behold sacred images and stories about the God.
\item
  Be silent with the God. Focus on your breathing for a few moments, and allow that to ground you into an open space. Then, focus on the God.
\item
  Thank the God.
\end{enumerate}

If you are pressed for time and need to add something to your preexisting ritual practice, the above takes as little as 2-3 minutes once you are familiar with the rhythm of the words and actions. You can add it onto your morning ritual if you have the time to do it --- I recommend placing it right after the invocation of the hearth Goddess and before you pray to any patron deities.

For more complex celebrations, we can build out from this prayer template to encompass other types of activities. For instance, if you are seeking blessings in spring for the herb garden that you are about to spend hours of care on, consider processing through your growing area after the offering or bringing your seedlings to the shrine to seek blessings. Perhaps you decide to honor the God at a meal for the summer solstice. You make sacred space at the park where you do your cookout, offer the God a portion of the meal, perform the remainder of the ritual, and eat in the God's presence when you're done. This is limited only by your imagination, common sense, and decency --- always be respectful of the Gods in a ritual space, don't do anything wildly dangerous, and embrace the time you have to consciously focus on your connection to the sacred.

\hypertarget{praying-for-what-is-possible}{%
\subsection{Praying for What Is Possible}\label{praying-for-what-is-possible}}

The best time to start praying for something is at the beginning --- weeks before you have that exam or performance evaluation when you start to feel the pressure. There is a fable from Aesop called ``The Shipwrecked Man and Athene'' in which a shipwrecked sailor prays to Athene (his city's patron Goddess) to be saved from the ocean. Another man, swimming by, says, ``Pray to the Goddess for success, but move your arms!'' --- the Gods can open up pathways for us, but we need to commit effort.

Prayers at the last moment may sometimes be answered, but there is nothing strange about this --- as souls who inhabit material bodies, and who are under the care of the Gods, it is in our nature to be able to \emph{act in the material world} with these bodies. That is why we descend into generation.

One way to make this part of your life is to say a short prayer to major Gods you already pray to whenever you're about to start planning a project or writing a complex, multi-part to-do list. It doesn't need to be fancy, just a few words --- and at the end of the project's completion, thank the Gods you prayed to with an offering that is slightly fancier than usual. It's up to you how to interpret \emph{fancy}.

\hypertarget{praying-for-what-is-inevitable}{%
\subsection{Praying for What Is Inevitable}\label{praying-for-what-is-inevitable}}

Our sun, now a main sequence star, continues to eat her fuel, leaving just a little bit less to use an instant from now. Eventually, she will puff up as a red giant star and swallow many (or all) of her inner planets before her death as a nova billions of years from now. Earth may only be habitable for another billion years maximum --- the sun's heat increases with its age, and Earth's core will one day no longer produce the magnetic field that protects Earth's surface from the impact of the solar wind. Any day now for the next 100,000 years, the star Betelgeuse in Orion could go supernova, and the explosion could be as bright as the half-moon, lasting for several months before fading.

Our incarnations a billion years from now will be on strange planets orbiting suns just blossoming into the caretakers of living beings. The universe is teeming with life.

When I was in my early twenties, I prayed to Hermes in his role as the God who conveys souls to the afterlife to ease my maternal grandfather's suffering as he passed. He had fallen, the doctor botched the anesthesia, and he was aspirating food into his lungs. He couldn't speak and clearly did not want any of us to see him in the state he was in. It was decided by the doctors, my mother, and my uncle that according to their father's wishes, given his prognosis, it was time to start pallative care. They removed the feeding tube. It could have taken him weeks to die. I prayed for a swift death, and he was gone within two days.

At some point in our lives, we will all need pallative care. The sun will rise, for us and others, until it, too, dies.

\hypertarget{sample-ritual-outlines}{%
\section{Sample Ritual Outlines}\label{sample-ritual-outlines}}

These are rituals that I do for the new and full moons, so they are personalized to Gods I worship and my own situation --- which is slightly eclectic, albeit in a systematized way.

There isn't information here about what is offered, but I do a brief purification and light the hearth candle at the beginning with a prayer to the hearth Goddesses. Typically, I offer a nice incense to the lunar deities during the first moon prayer. I give a libation of milk to the stars (yes, I am not above inserting humor about the Milky Way) and a libation of water to the lunar deities.

When I do full moon rituals with my mother, who is Wiccan, we do something slightly different --- in her initiatory tradition, sacred space is set up through marking out a circle and visualizing a separation between the ritual space and the exterior world in the form of a sphere. Then, the elements are honored, and finally, the lunar deities are honored. She is a devotee of Hekate, who is associated with the moon and the sublunary world, so Hekate is a part of that ritual.

Note that the line breaks may lead one to believe this is poetry. While I did approach the ritual liturgy in a poetic style, it is usually not quite poetry. I find that adding line breaks also helps for reading during ritual, at least until I've memorized something.

\hypertarget{new-moon}{%
\subsection{New Moon}\label{new-moon}}

\hypertarget{opening-ground-and-center}{%
\subsubsection{Opening: Ground and Center}\label{opening-ground-and-center}}

Imagine yourself growing branches into the Earth, and feel the energy that you gather from her and the way your leaves are tied to her seasons and cycles. Imagine the sky up ahead --- the sun, the vault of stars, the inky blackness, the planets, the arms of our galaxy, the cosmic web --- and imagine that you are growing roots to nourish yourself on the blessings from above. Feel the energy from the Earth and sky combine within you, connecting you to the cosmos above and below.

(Note: I learned the basics of this approach in \href{https://www.google.com/books/edition/Stellar_Magic/h6lsPgAACAAJ?hl=en}{\emph{Stellar Magic: A Practical Guide to the Rites of the Moon, Planets, Stars and Constellations}} by Payam Nabarz and adapted it because I am very into cosmology --- I include more types of celestial bodies, and the above roots up instead of down because I've read too much Proclus. Nabarz has written a useful book for anyone interested in doing devotions for celestial bodies. Because I'm not interested in magic, I ignore the magical workings in that book or reframe them around sacred activities.)

\hypertarget{prayer-to-the-moon}{%
\subsubsection{Prayer to the Moon}\label{prayer-to-the-moon}}

I pray to the luminous new moon,
beginning of new growth,
auspicious and divine,
male and female, feminine and masculine,
God of the soothing northern moon,
Goddess of the healing southern moon,
God of the east-rising moon,
Goddess of the west-setting moon.
May you, Gods of the Moon, accept this offering:
Mani, Sirona, Selene, Luna, whichever Gods are
enthroned, exalted, upon the Moon's holy body.
May you, O Moon, and the Gods seated upon you
bless and protect us, guiding us to our Good.

\hypertarget{prayer-to-apollon-noumenios}{%
\subsubsection{Prayer to Apollon Noumenios}\label{prayer-to-apollon-noumenios}}

\href{http://www.labrys.gr/en/text_noumenia.html}{I use a prayer from Labrys, which I am not reprinting here for copyright reasons, but it is on their website.}

\hypertarget{moon-salutations}{%
\subsubsection{Moon Salutations}\label{moon-salutations}}

If one knows moon salutations in yoga, this is an opportunity to mindfully do two. Chant \emph{Selene, Mani, Ueronadā, Sironā, Luna,} each time a salutation is begun.

Follow this with Chandra Bhedana, a closure of the right nostil for 20 inhalations while thinking of clear white light coming in and out, purifying. Afterward, come into stillness and imagine being connected to the whole of generation, humming with the Gods, while firmly situated at one's center. This meditation can last as long as one likes.

\hypertarget{gratitude-prayer}{%
\subsubsection{Gratitude Prayer}\label{gratitude-prayer}}

We thank the Gods for witnessing this new moon ritual:
Goddesses of the hearth, first and last,
esteemed and gentle Mani,
God of the northern moon,
keeping day and night in solemn compassion;
beloved Sirona, adorned with stars,
Goddess of the night and the holy moon;
lush-braided Selene, luminous and enchanting,
Goddess who smiles down on the changeable world.
Apollon Noumenios, holy God of the new month,
accompanied by serpents, marking the boundary
between old and new, adorned with laurel.

\hypertarget{closing-prayer}{%
\subsubsection{Closing Prayer}\label{closing-prayer}}

Thank you, O Gods.
May you protect my family and my girlfriend's family.
May you be our ever-present companions, O holy ones.
May you drive back sickness, evil, and despair,
and may you bring us what is good and best.
The ritual is concluded.

\hypertarget{full-moon-aspirational}{%
\subsection{Full Moon: Aspirational}\label{full-moon-aspirational}}

\hypertarget{opening-ground-and-center-1}{%
\subsubsection{Opening: Ground and Center}\label{opening-ground-and-center-1}}

\hypertarget{orphic-hymn-to-the-stars}{%
\subsubsection{Orphic Hymn to the Stars}\label{orphic-hymn-to-the-stars}}

\href{https://www.press.jhu.edu/books/title/9661/orphic-hymns}{I use the Athanassakis translation}, but \href{https://www.theoi.com/Text/OrphicHymns1.html\#6}{the one by Thomas Taylor is freely available online}.

\hypertarget{prayer-to-the-moon-1}{%
\subsubsection{Prayer to the Moon}\label{prayer-to-the-moon-1}}

I pray to the luminous full moon,
swelling with abundance,
auspicious and divine,
male and female, feminine and masculine,
God of the soothing northern moon,
Goddess of the healing southern moon,
God of the east-rising moon,
Goddess of the west-setting moon.

I pray to you, luminous Moon, seat of many Gods,
Mani, Sirona, Selene, and many others call you theirs.
At the fullness of your beauty, shining upon all,
reflecting the luminous sun, sublunary ruler,
give my soul sustenance from your holy light.
Do not let me become entranced by illusion,
but guide me to the luminousness of Intellect
seated beyond the spheres, beyond the subjective,
vehicled in noetic and noeric fire, resplendent.

May you, O Moon, accept this offering.
May you, O Moon, and the Gods seated upon you
bless and protect those of us in embodiment,
guiding us to our Good, guiding us home.

\hypertarget{lunar-envesselment}{%
\subsubsection{Lunar Envesselment}\label{lunar-envesselment}}

Ideally, hold the bowl of water in a way that puts it line-of-sight with the moon outside. If this is not possible, close your eyes and envision moonlight filling the bowl. Chant \emph{Aiglê, Pasiphae, Eileithyia, Selene, Mani, Ueronadā, Sironā, Luna} in a musical way. Pour out some of the water for Selene and the Stars, then drink the rest while envisioning the power of the celestial vault and the moon filling you.

Follow this with Chandra Bhedana, a closure of the right nostil for 20 inhalations while thinking of clear white light coming in and out, purifying. Afterward, come into stillness and imagine being connected to the whole of generation, humming with the Gods, while firmly situated at one's center. This meditation can last as long as one likes.

\hypertarget{gratitude-prayer-1}{%
\subsubsection{Gratitude Prayer}\label{gratitude-prayer-1}}

We thank the Gods for witnessing this full moon ritual:
Goddesses of the hearth, first and last,
the stars who circle in the night sky above,
boundless and bounded, far and yet so near,
to Selene of lush braids,
she who waxes and wanes,
who counts out the month in her dance
around our Earth as the sunlight shimmers upon her surface;
to esteemed and gentle Mani,
God of the northern moon,
keeping day and night in solemn compassion;
to beloved Sirona, adorned with stars,
Goddess of the night and the holy moon.

\hypertarget{closing-prayer-1}{%
\subsubsection{Closing Prayer}\label{closing-prayer-1}}

Thank you, O Gods.
May you protect my family and my girlfriend's family.
May you be our ever-present companions, O holy ones.
May you drive back sickness, evil, and despair,
and may you bring us what is good and best.
The ritual is concluded.

\hypertarget{full-moon-when-its-been-a-long-day}{%
\subsection{Full Moon: When It's Been a Long Day}\label{full-moon-when-its-been-a-long-day}}

\hypertarget{opening-ground-and-center-2}{%
\subsubsection{Opening: Ground and Center}\label{opening-ground-and-center-2}}

\hypertarget{prayer-to-the-moon-2}{%
\subsubsection{Prayer to the Moon}\label{prayer-to-the-moon-2}}

I pray to the luminous full moon,
swelling with abundance,
auspicious and divine,
male and female, feminine and masculine,
God of the soothing northern moon,
Goddess of the healing southern moon,
God of the east-rising moon,
Goddess of the west-setting moon.

I pray to you, luminous Moon, seat of many Gods,
Mani, Sirona, Selene, and many others call you theirs.
At the fullness of your beauty, shining upon all,
reflecting the luminous sun, sublunary ruler,
give my soul sustenance from your holy light.
Do not let me become entranced by illusion,
but guide me to the luminousness of Intellect
seated beyond the spheres, beyond the subjective,
vehicled in noetic and noeric fire, resplendent.

May you, O Moon, accept this offering.
May you, O Moon, and the Gods seated upon you
bless and protect those of us in embodiment,
guiding us to our Good, guiding us home.

\hypertarget{chant}{%
\subsubsection{Chant}\label{chant}}

I go to the window if the Moon is visible and chant the verses from Mike Oldfield's \emph{Incantations} Part One, but may vary the deity names according to the melody of that album. It's very meaningful because my mom played this album frequently when I was a young child, and she would also sometimes sing this chant to the moon. Sometimes, I do this part before the actual ritual because I see the moon rise from my apartment window, which always takes my breath away.

\hypertarget{gratitude-prayer-2}{%
\subsubsection{Gratitude Prayer}\label{gratitude-prayer-2}}

We thank the Gods for witnessing this full moon ritual:
Goddesses of the hearth, first and last,
the stars who circle in the night sky above,
boundless and bounded, far and yet so near,
to Selene of lush braids,
she who waxes and wanes,
who counts out the month in her dance
around our Earth as the sunlight shimmers upon her surface;
to esteemed and gentle Mani,
God of the northern moon,
keeping day and night in solemn compassion;
to beloved Sirona, adorned with stars,
Goddess of the night and the holy moon.

\hypertarget{closing-prayer-2}{%
\subsubsection{Closing Prayer}\label{closing-prayer-2}}

Thank you, O Gods.
May you protect my family and my girlfriend's family.
May you be our ever-present companions, O holy ones.
May you drive back sickness, evil, and despair,
and may you bring us what is good and best.
The ritual is concluded.

\hypertarget{ancestor-ritual}{%
\subsection{Ancestor Ritual}\label{ancestor-ritual}}

When I pray to the ancestors, I give an offering of Nippon Kodo Cafe Time incense (five-minute cones) and sake or mead, and I do this offering at night during the final days of the lunar cycle, before the new moon. After the prayers, I may give the ancestors life updates.

\hypertarget{for-ancestors-related-to-me}{%
\subsubsection{For Ancestors Related to Me}\label{for-ancestors-related-to-me}}

I make this offering to
my grandparents of my paternal line,
my maternal line, my great-grandparents
each twist and turn of marriage,
the many lives branching out and upward
into the past in root-thickets of consciousness,
making a spirit and vessel of those who have chosen
to be incarnated in my family.
I pray to my recent ancestors whose faces I know,
to my distant ancestors whose faces and names have receded.
I pray to the ancestors who kept the old rites for the Gods,
for the ones who had lost them unknowing.
Please accept this offering of mead and incense.
Come, ever-gentle, and grant blessings to us, your descendants.
Assist us along the paths that we must take
in this guest-house we call a lineage.
Each of us, like sparrows, has entered this hall only to fly out again.
Incarnating here among the many choices of life,
we have given and will give what we have.
I honor your memory and seek your favor.

\hypertarget{prayer-to-the-disir}{%
\subsubsection{Prayer to the Disir}\label{prayer-to-the-disir}}

I pray to the holy Disir,
protectors of your descendants,
Goddesses, spirits, and esteemed female dead,
I come to you in gratitude.
Thank you for your guidance and protection.
May you protect the women of my family.
May you bless the women of my family.
May you instruct the women of my family.
May you be present in our homes and our lives.
We welcome you, trusted divinities.
We welcome you, our great elders.

\hypertarget{prayer-to-ancestors-of-intellect}{%
\subsubsection{Prayer to Ancestors of Intellect}\label{prayer-to-ancestors-of-intellect}}

For my intellectual lineage,
bright minds who illumine the soul,
who train it ever upward ---
I make this offering to you, O illustrious ones,
pure souls shining in the words I read.
Please accept this prayer of gratitude.
You guide me through the pages.
You walk me along the upward path
even though the brambles sting and my mind tires,
so I may know the gifts of true beauty
hidden in passages strung like ivy and laurel,
uncovered by those known and unknown to me,
men and women of steady minds,
of hearts so filled with the love of the Gods
and the spaciousness beyond being that is no space,
intellectuals, philosophers, true theologians,
poets and creators of the most beautiful agalmata
holding the keys to the Gods who rule all things.

\hypertarget{closing}{%
\subsubsection{Closing}\label{closing}}

Thank you, ancestors, for your blessings. Accept this offering, O predecessors.

\hypertarget{exercise-do-a-special-ritual}{%
\section{Exercise: Do a Special Ritual}\label{exercise-do-a-special-ritual}}

At the beginning of this chapter, I encouraged you to jot down some notes with ideas you have about honoring the sacred. Choose \emph{one} thing coming up after your next payday (or grocery shopping haul). Is it the full or new moon? Are you starting or ending a project? Have you just adopted a cat? Is it the start of exam period?

Using the information here and your budding experience with prayer, create a ritual outline. After consulting your budget, add anything you need for that ritual to your grocery/supplies list. Hold time on your calendar and do what you need to do.

Why wait until your next pay period? Many of us base our activities around our pay cycles. If you have room in your current cycle's budget for something special, it's fine to adjust. I get paid monthly and set aside a certain amount of money for incidental expenses. Usually, by the time the fourth week of the month rolls around, between coworkers fundraising for their children's creative enrichment programs, impulse book purchases, and realizing all of my socks (somehow) once again have holes in them, I'm at the point where I am thinking carefully to avoid going overbudget --- and I'm sure many people experience similar ebbs and flows. While offerings are factored into my monthly spending and I plan ahead for higher-cost holidays like the solstices, they are probably not budget categories for you yet. In any case, over the next few months, track what you spend on offerings and other spiritual wellness items, and the average of that is what you should put down as your habitual spending.

If you don't have a pay period, remember to do what you can with what you have. Flowers gathered in a natural area, nicely-steeped tea, or a small portion of what you have on hand can all make lovely offerings.

\hypertarget{lifelong-learning}{%
\chapter{Lifelong Learning}\label{lifelong-learning}}

Throughout the first part of this book, we built up the foundations of ritual practice that you can bring forward into your life.

Ritual practices, like most routines, will iterate. You will go through seasons of your life when five minutes \emph{is} all you have and other seasons when you can, and want, to do more. There will be days when showing up at shrine will be the last thing on your mind --- maybe you're anxious over an unexpected bill, you've lost a job, or someone in your family is going through a hard time --- and other times when you feel the abundant presence of the universe all around you.

Practicing daily rituals and establishing presence with the Gods is slightly different from studying something like religion, theology, or philosophy. Many reading this book may already be thinking of ways to deepen a practice by engaging with learning material and/or joining formal or informal group(s). This is the first part of three, and here, we will focus on how to develop self-directed learning trajectories. If you are completely good with your five minutes, the suggestions in this chapter are also adequate for secular learning.

Let's start by thinking about what we want to study.

In your notes on this book, and in your practice thus far, you may have jotted down questions, or the same thoughts may be cycling through your head over and over. Here are some examples:

\begin{itemize}
\tightlist
\item
  Why is Kaye talking about Platonism? Didn't Plato just write political philosophy? How does this match what I learned in school?
\item
  While praying to Ptah, I came across hymns that used terms unknown to me. What do these mean in Egyptian traditional religion, and how can I build up my knowledge so I understand what I'm reading?
\item
  Stoicism talks about Zeus a lot. Is there a theistic way to approach Stoicism that is different from the Silicon Valley self-help?
\item
  That's a good point about needing to consider honoring Gods associated with yoga if I do yoga. What are ways to approach this that are respectful of South Asian culture?
\end{itemize}

Before I got really into a Platonizing practice, I had heard from someone that everything I needed to know about the Hellenic Gods is contained within Plato. The first dialogue I read after hearing that, Plato's \emph{Symposium}, was astonishingly different from what was assigned in school --- and I wanted to read more. Still, without a committed study habit and with a lot changing in my twenties as I finished school and started working, it was hard to read \emph{anything}, let alone Plato --- and the hours I spent on social media didn't help. It wasn't until my late twenties when I left Facebook that I started to rediscover my focus and mental energy.

\hypertarget{content-is-written-by-people}{%
\section{Content Is Written By People}\label{content-is-written-by-people}}

Generally speaking, there are two types of content you will encounter when you are trying to learn more: academic content and practitioner content. Some academics also worship many Gods (and sometimes the Gods they study), but due to pressures in academia, it can be hard to figure out who they are unless you start talking to people.

Academic content is most frequently released in the form of journal articles, books (jargon term: scholarly monographs), or gray literature (think PowerPoint slides, reports, and conference talk notes). Practitioner content may be in the form of social media posts, blogs and short manuals, and published books. Both academics and practitioners will post lectures to YouTube and other platforms.

One challenge with \emph{any} content in the present century is the politically-charged, polarized atmosphere. For example, content on Nordic Gods is divided between extreme right-wing groups with white supremacist associations and extremely progressive material. The right-wing materials put money into the pockets of people who are harming nonwhite American citizens (among others). The extremely progressive material may assume that you are only one awakening away from becoming a Marxist-anarchist, but it's a more ethically sound financial investment. I usually Google an author's name to see what comes up and avoid people and publishing houses that ring alarm bells. I do the same with Etsy purchases.

Another challenge is the tendency for people to be pressured to monetize content in order to get by and make rent, especially with today's wealth inequalities --- often before that person has enough mastery of the material to be ready to provide a spiritual service to others. The Internet is filled with people who got into something 10 months ago and are trying to make money off of it. We always need to put the spiritual wellness and integrity of the ones we are teaching \emph{first} in any learning environment, as being in an teacher/student relationship means that the recipient of a teaching has invested care of their soul's development in \emph{us}. It's not to be taken lightly or carelessly.

Experience itself is multifaceted --- someone may have a PhD in a topic related to the Gods or initiatory training in a specific tradition, and another person may have spent 20-30 years experimentally working through a non-initiatory practice on their own (although, truth be told, nothing happens in a vacuum). In library science, we call this ``authority is constructed and contextual'' (an information literacy frame as put out by American College and Research Libraries, also known as ACRL). If you are engaging with informal content (like a blog), be sure to check the person's About page and their blog's first few posts. Usually, bloggers will write an origin story at some point to talk about where they are starting from, and even if that origin story is a decade old on their Wordpress (meaning you don't know how it compares to who the person is now without reading their bio for their more recent activities), it gives you valuable information about them. I've changed how I present myself significantly over the past few years due to an increased awareness of and desire to respect modern Hellenes/Greeks. I have zero expertise in how a Greek person mediates their Hellenic identity, whichever God or Gods they worship, or what worshipping Gods means to them as an individual, and I am building up what a theistic cultural reception of the Hellenic Gods can look like in America (hopefully in a tactful way), so that's where I focus my attention. That's in my bio, but the first post on my blog is something I wrote in my mid-to-late 20s about my values and practice, which looked very different at the time.

\hypertarget{finding-book-reviews}{%
\section{Finding Book Reviews}\label{finding-book-reviews}}

If you Google an author's name or a book title with the phrase ``book review'' (with the quotes around ``book review''), you will find reviews that Google has indexed beyond just what's on Amazon. If you don't have ideas for a specific book, just put in a topic (such as divination, worshipping Kemetic Gods, Shiva in yoga, prayers for Gods).

Many of the books in the Google results will be worth reading, although it's good to know an author's perspective before you jump into their work.

\hypertarget{using-the-library}{%
\section{Using the Library}\label{using-the-library}}

Library access can be found at the local, state, and national levels. You may also have access to the library of the institution where you work or study. Your state library will license resources for the entire state to use remotely --- it's worth knowing what those resources are. The same goes for your local library.

When you visit your library's website, also check their policies on interlibrary loan and item delivery between libraries in the same state system. Interlibrary loan is often no-fee at academic institutions, but some local and state libraries will apply a fee to using it to offset the costs. Simialrly, item delivery between locations may either be free or have a nominal fee.

\hypertarget{do-you-have-alum-jstor-access}{%
\subsection{Do You Have Alum JSTOR Access?}\label{do-you-have-alum-jstor-access}}

At least in the United States, many people who have graduated from college have graduated from institutions that provide alum access to JSTOR. Look at your alum website, and if you can't find anything about that, contact the library.

\hypertarget{my-plan-for-plato}{%
\section{My Plan for Plato}\label{my-plan-for-plato}}

This is the advice that, after several years studying Platonism-related topics, I would give to someone embarking on an in-depth study. Notice how my own plan meandered and changed as I learned more. This is inevitable.

If one has heard that Platonists were spiritual sages in antiquity --- say, from Linda Johnsen's \emph{Lost Masters: Sages of Ancient Greece}, one may attempt to dive directly into Plotinus like I did when I was in my early 20s and have a disorienting and awful experience. No matter how exciting an author seems, starting with the fundamentals is crucial to understanding what is going on at all.

So. I bought one of those giant books with every work by Plato and started reading.

That was also a mistake.

The breakthrough happened when I learned that there was such a thing as \href{https://www.oxfordhandbooks.com/view/10.1093/oxfordhb/9780199935314.001.0001/oxfordhb-9780199935314-e-28\#:~:text=According\%20to\%20Iamblichus\%2C\%20for\%20example,and\%20finally\%20Timaeus\%20and\%20Parmenides.}{a Plato reading order}. There is a modern, which-one-did-he-write-when order (which I don't recommend), and there are several systematic treatments that were used in the ancient world when reading Plato --- one with a lot of dialogues and another with a selection. The selection of dialogues is the Iamblichean reading order, named for Iamblichus, a Syrian Platonist who was instrumental for reinvigorating traditional religious practice through the careful application of Platonic doctrine. He assigned twelve dialogues: \emph{Alcibiades I}, \emph{Gorgias}, \emph{Phaedo}, \emph{Cratylus}, \emph{Theaetetus}, \emph{Sophist}, \emph{Statesman}, \emph{Phaedrus}, \emph{Symposium}, \emph{Philebus}, \emph{Timaeus}, and \emph{Parmenides}.

At first, I read the dialogues on their own, and I reached the \emph{Symposium} with that method. I picked up one of the Platonic commentaries (the one by Hermias on Syrianus' \emph{Phaedrus} lectures) by happenstance and discovered that its exegesis of the material was extremely useful and groundbreaking. However, when I read the \emph{Parmenides} commentary, I felt like the rug had been pulled out from under me because my understanding of the rungs on the ladder from the One to Matter was almost nonexistent in any meaningful way. I continued reading Platonic texts and found some helpful modern scholars' work at good price points that helped me understand the metaphysics a bit more. I also started listening to the \emph{Secret History of Western Esotericism} podcast (despite not identifying as an esotericist) because its episodes on Plato were very helpful. I read the \emph{Parmenides} before the \emph{Timaeus} and only read the \emph{Timaeus} after the \emph{Republic} and \emph{Laws}, which I added due to never having read them. At about that time, I realized that there was no one correct reading order for the commentaries and that swallowing them down as simultaneously as possible would give me the best rough outline. The pandemic happened all of a sudden (when I was reading the \emph{Laws}); spiritual Platonists started posting content to YouTube --- Tim Addey, Mindy Mandell --- and I found other content from Pierre Grimes. The combination of reading, audio, and visual content was extremely helpful. I also found some virtual groups to be active in, all based on Zoom, and a few people I knew on social media were extremely helpful and generous with their time when I had questions. I continued --- and continue --- swallowing down commentaries and other Platonic writings.

What I would encourage, based on my experience, is to read \emph{Alcibiades I}, \emph{Gorgias}, and the \emph{Phaedo}. Then, stop reading those and pick up Radek Chlup's \emph{Proclus}, which introduces key elements of the system in a manageable way. Take note of the tables included in that book --- they are very useful. Several articles (see bibliography) by Danielle Layne contain useful tables for understanding how Platonists read dialogues, and \href{https://www.academia.edu/30296722/_Polycentric_Polytheism_pp._37-40_in_Witches_and_Pagans_32_June_2016}{Edward Butler wrote an article for a general audience on the polycentricity of the Gods} that can be used to help drive those concepts home. Deviate from this to read Iamblichus' \emph{On the Mysteries/De Mysteriis}, which will help ground all of the cereberal in practical prayer and attentiveness to the Gods. Ensuring that one maintains an active prayer practice while learning is absolutely crucial to gaining insights.

Read the \emph{Alcibiades I} again, this time in conjunction with one of the commentaries (remember: interlibrary loan) --- Proclus and Olympiodorus both wrote on it. Read Tim Addey's \emph{The Unfolding Wings}, Sallust's \emph{On the Gods and the World}, and Mindy Mandell's \emph{Discovering the Beauty of Wisdom}. For podcasts, listen to the \emph{Secret History of Western Esotericism} from the beginning, \emph{The History of Philosophy Without Any Gaps}' early episodes, and look up any podcast interviews with Gregory Shaw or Edward Butler. From there, read the Gorgias and its commentary, the Phaedo and its commentaries, and continue in that fashion for the dialogues that Iamblichus recommended until you reach the Parmenides. For the Parmenides, first read Aristotle's \emph{Metaphysics} and Syrianus' response to it, then read the Parmenides alongside Proclus' commentary. While reading the commentaries, cycle to treatises from Plotinus as they are mentioned in footnotes and endnotes. The \emph{Elements of Theology} and \emph{Platonic Theology} are useful to approach now, as are any other texts by Aristotle you may be interested in (or not). Read Plotinus' \emph{Enneads} and any of Proclus' essays or Damascius' other works. It's all an iterative process, and once one has the trunk fairly solid, the branches may grow where they take you closer to the light.

If you're just curious about the philosophical system and do not want the deep details, reading something like \emph{Discovering the Beauty of Wisdom} alongside the \emph{Alcibiades I}, \emph{Phaedo}, \emph{Apology}, and \emph{Phaedrus} will get you a decent idea. I recommend that anyone read Iamblichus' \emph{On the Mysteries}, though. To be perfectly frank, whenever I got discouraged writing this primer on worshipping Gods because it seemed daunting or I doubted my relevance, skill, or capacity, I just thought back to Iamblichus to anchor and guide me. Almost everything we need to know about honoring the Gods is in Iamblichus.

\hypertarget{creating-your-plan}{%
\section{Creating Your Plan}\label{creating-your-plan}}

The above may have been overwhelming. For one, Platonism involves reading a lot of scholarly monographs --- someone studying Norse concepts of the soul or divination or modern devotional poetry will skew their reading more towards practitioners' publications. If you are studying Stoicism, the modern popularization has made it so accessible that Stoics even have their own meditation app.

Here is what I hope you took away from the prior section on getting more familiar with Platonism:

\begin{itemize}
\tightlist
\item
  Choose multiple formats. Audio, video, and writing are all represented above.
\item
  While starting out with a survey book of a topic may be ideal, you do actually need a sense of the material itself. In the example above, my advice to read a few dialogues before getting into other types of content is based on that premise.
\item
  You don't need to have it figured out all at once. You can use the works cited by someone to build out and adjust your reading list.
\item
  Learning can be overwhelming, as it pushes us outside of our comfort zones. That's why it's learning. During the lowest points, do what you can to ease the discomfort and remember why you're doing this.
\end{itemize}

Above all, keep your plan sensible. If you have half an hour a day to study between when you put your kids to bed and when you want to spend time with your spouse --- or if you know that your only alone time at home will be when you wash the dishes with headphones in --- do not choose a plan that is unrealistic about the formats you need or the reading speed you have. If you have a condition that impacts your focus or stamina, put your health first and seek out advice from others with similar conditions about how they build time for learning and personal enrichment into their lives. For example, magazines like \emph{ADDitude} (for people with ADHD) or \emph{Momentum Magazine} (for people with multiple sclerosis) are good choices for getting the support you need, as they synthesize information from experts and community members. Typically, you can find such publications via Google or in resources lists on nonprofit websites.

If you frequently use social media, it's likely that your attention span will have deteriorated from when you were younger. Consider doing a social media fast for a few days while you start up your new study habit. If you frequently binge-watch shows, reducing your screen time to 1-2 episodes per day can improve you experience of the show and give you some much-needed reading time, as research shows binge-watching can reduce the amount of enjoyment we get out of savoring new episodes.

Put your audiobook and ebook apps on your phone's home screen and hide everything that isn't reading material. Ensure that books you need to read are located within arm's reach on the couch.

Here are some guidelines for getting the most out of your learning plan:

\begin{itemize}
\tightlist
\item
  While you read, take notes. Keep a notebook or write in the margins so you can roll around in the text and react to what you're reading. If you are using ebooks, export the notes when you are done so you can review them.
\item
  Never underestimate exercise. I've found that doing some exercise in the early evening helps me bounce back from mental fatigue --- something as simple as a walk or a few sun salutations.
\item
  Build on what you're learning. Start writing essays about things that interest you. You're not being graded.
\item
  Celebrate small wins. When you have finished something tricky, do something nice for yourself --- put on your favorite song, dance, give yourself a hug, or say something good about yourself in the mirror.
\end{itemize}

While you are studying, you will likely come across courses (free and paid), reading groups, and other ways to learn from experts. Take advantage of the opportunities that are affordable to you, and don't assume that a higher cost equals a better learning experience. Be wary about becoming a ``groupie'' for a specific expert --- direct your devotion at the Gods, not at human beings, and always be aware that the ones you are learning from are humans with merits and flaws. If someone is trying to induce the fear of missing out (FOMO) in their marketing materials, or if they're positioning themselves as the only safe harbor, be suspicious. They probably don't have the exclusive knowledge they're claiming to possess.

\hypertarget{excercise-create-a-content-list}{%
\section{Excercise: Create A Content List}\label{excercise-create-a-content-list}}

Brainstorm a topic that you want to know more about, ideally related to worshipping Gods. Identify content items that you can engage with to learn more about it. They could be books, videos, podcasts, courses --- you name it. Look up reviews for the books and learn more about the podcast, video, and book creators. Narrow the resources down to five.

Then, find learning time in your schedule --- one or two blocks of time per week. Use that time to engage with the materials you have selected.

\hypertarget{virtue}{%
\chapter{Virtue}\label{virtue}}

\begin{quote}
Virtue might be described as the perfection of the soul and proper balance of its life and as the highest and purest activity of reason and intellect and discursive intelligence. Let the acts of virtue be taken, above all, as being boniform, excellently fine, intellectual, noble, full of moderation, participant in appropriateness, promoting moral advancement, aiming at the best end, and graceful.

Iamblichus, Letter 16: To Sopater, ``On Virtue,'' trans. Dillon and Polleichtner
\end{quote}

Cultivating virtue and adhering to an ethical system are absolutely essential for engaging with other people. If we define virtue as the excellence by which we are perfectly realizing our soul's capacity, becoming more virtuous is also an offering that we can give to the Gods, an intangible method of connecting our own partial existence to their expansive, limitless oneness.

Many of us do not think about virtue, morality, or ethics constantly, at least until we are wronged by others or we need to make a hard decision between doing what is convenient and doing what is right. In Proclus' \emph{Ten Problems Concerning Providence}, he writes (in alignment, incidentally, with Platonic teachings found in texts like the \emph{Phaedrus} and \emph{Phaedo}) that virtue, and what we develop inwardly, remains with us through even the most extreme misfortunes. We are often thwarted from achieving external delights, he says, but we can always turn inward to develop our inner core. In \emph{On Providence}, Proclus expresses admiration for the teachings of Epictetus, especially the one letting go of what we cannot control.

We often think of virtue in terms of self-denial, with an image of a celibate monk or nun or other renunciate in our minds. The tension between embracing embodiment and fleeing from external pleasures has always existed, both in spiritual and secular philosophies and lifestyles. The guest-house metaphor I used for thinking about ancestry many chapters ago can also be used here as a bridge between this self-denying perspective and a perspective that embraces embodiment. Many virtue systems focus on correct behavior and action because we are acting out roles, and like actors in a play, taking too much of the role into ourselves can be destructive, especially if something happens that jars our most extreme negative emotions.

Broadening out from the system I have internalized, ethical guidance to follow may include things like the Delphic Maxims, Solon's Tenets, guidance within the Hávamál, Kemetic teachings on Ma'at, the Pythagorean Golden Verses, Stoic writings, Aristotle, the yamas and niyamas of yoga, and so on. There are often commentaries on these guidelines written by practitoners and academics. I encourage you to read as many systems as you need and to think about how they relate to your own life --- many date back to ancient times, society has changed, and it can be useful to examine what has remained the same and what is different. The Delphic Maxims, as one example, include one that reads ``rule your wife.'' Contextual to the culture at the time, men often married women much younger than them, most women did not enjoy vast personal freedoms, and the same stereotype existed about women being overindulgent spendthrifts that exists today in many parts of the world. The maxim is actually about ruling the parts of us that are less seasoned and more prone to desire, all bundled in a metaphor that many women have issues with when we read it.

Typically, people will choose a system that is related to Gods they worship, especially since we have been habituated by the prevailing circumstances in America to view ethics as something given by religious tenets. Once someone starts looking for other like-minded people, it is common for them to study a specific text together. In organizations, examining a specific text may even be part of the onboarding process. One can even examine secular ethical and virtue writings, alone or in community, to tease apart the best elements of a virtuous life.

\hypertarget{a-continuous-process}{%
\section{A Continuous Process}\label{a-continuous-process}}

Many people were taught in their childhood that missteps and incorrect behavior were shameful, often in ways that made them feel powerless to change how others had judged them. We will discuss virtue and ethics from a growth mindset --- we all start somewhere, and we all have aspects of our lives that we nail and other parts of our lives where we need to put in more work.

A growth mindset is adequately described in this passage from Plotinus:

\begin{quote}
If you do not yet see yourself as beautiful, then be like a sculptor who, making a statue that is supposed to be beautiful, removes a part here and polishes a part there so that he makes the latter smooth and the former just right until he has given the statue a beautiful face. In the same way, you should remove superfluities and straighten things that are crooked, work on the things that are dark, making them bright, and not stop `working on your statue' until the divine splendour of virtue shines in you, until you see 'Self-Control enthroned on the holy seat''

Plotinus, Ennead 1.6.9, trans. Gerson et al.
\end{quote}

In the above passage, we are encouraged to actually work on the parts of ourselves that seem daunting. From a practical standpoint, we can ask ourselves:

\begin{itemize}
\tightlist
\item
  What are my unhealthy habits?
\item
  What do I lie to myself about?
\item
  What have I been avoiding?
\item
  Last time I had a conflict with someone, what went wrong? Is there anything I need to learn for the next time I am in a similar situation?
\item
  In the past, what has worked for me when changing behavior? What hasn't worked?
\item
  What are my strengths?
\item
  How am I going to build self-care and self-compassion into my personal growth process?
\end{itemize}

The above questions may cause some distress, especially if we have been ignoring something for a long time. Look at pictures of cute animals, listen to relaxing music, and take care of yourself. Moral self-image stress can cause a lot of internal friction.

On a personal note, James Clear's \emph{Atomic Habits}, Cal Newport's \emph{Digital Minimalism}, the Fabulous App (see your device's app store), and the interpersonal trainings on LinkedIn Learning have all been immensely useful to me over the years. James Clear wrote, ``You do not rise to the level of your goals. You fall to the level of your systems.'' This has proven true for me, and it's one of the principles I used when describing how to start a prayer routine.

➡️ If you are experiencing mental health struggles, I encourage you to see a therapist. Talking through elements of our life with someone else can be a very valuable experience, especially for those of us who had rough childhoods and who developed toxic shame early in our lives. If your life feels overwhelming and you are considering an exit, please call the National Suicide Prevention Hotline at 800-273-8255. ⬅️

\hypertarget{platonic-virtue}{%
\subsection{Platonic Virtue}\label{platonic-virtue}}

On a functional level, one of the reasons we have so many issues --- at least in Platonism --- is due to the operation of our three-part soul. As Plato's Socrates describes in the \emph{Republic}, we have an appetitive soul (the seat of desire), described as many-headed; a spirited soul (the seat of emotional reactions), described as a lion; and the rational soul (the seat of our logical thinking), what makes us human. In embodiment, the soul's rational faculties are driven this way and that by the two layers of our irrational soul (appetitive and spirited). Many of our issues in life are due to not having a good handle on our impulses. The virtue that restrains the appetitive soul is temperance; the spirited soul, courage; and the rational soul, prudence. When everything is working together properly, the soul can manifest justice and its highest potential. Mishaps and brief falls are inevitable during our embodiment.

Platonism incorporates some elements of Aristotelian and Stoic teachings to classify levels of virtue according to nature, habitude (sometimes called ethical), civic society, purificatory, contemplative, paradigmatic, and hieratic. Most people do not aim for the contemplative, paradigmatic, and hieratic virtues in their lives; to be functional members of society, we need to have everything up to the civic virtues nailed on a routine basis. The purificatory virtues signal a turn from that outer life to our own interior, culminating in the hieratic, which backflows outward from us and towards others through ritual action and care for the community. Tim Addey's \emph{Unfolding Wings} discusses this in greater detail (Chapter 4), as does Mindy Mandell's \emph{Discovering the Beauty of Wisdom} (Chapter 4). A good overview from an ancient commentator is Proclus' Essay 7 on the Republic, which is soon to be released in an English translation. Proclus and Olympiodorus also focus on virtue in their commentaries on Alcibiades I; \href{https://www.academia.edu/44622434/The_Neoplatonic_Scale_of_Virtues_in_Olympiodorus_Body_and_Soul}{the translator of Olympidorus' commentary on \emph{Alcibiades I} did a video presentation on the Platonic ladder of virtue}.

Usually, the things we want to work on are bad habits. Occasionally, we have errors in our civic judgments --- our sense of what is politically right and wrong may be off. Reactively, we may seek out the first ideology that seems ``correct'' and perpetuate our cycle of ignorance. It is better to explore \emph{what} was wrong about our old beliefs and to critically interrogate every system we approach. Tim Addey wrote, strikingly, that ``conflicting temperaments are often harmonized and reconciled by communal forces {[}at the habitual level{]}, and {[}they{]} only reemerge during the breakdown of civil order'' (\emph{The Unfolding Wings}, 67). I often think about that when I look at the major political upheavals we experience today and how chaotically people react to them.

Virtue has a direct connection to our capacity to receive the Gods when we pray. Working on it calms the discord within our heads. It pulls the soul back from the many desires and passions. When we are calm inside, we can make our focus on prayer more or less complete. This is why, in the \emph{Laws}, Plato writes:

\begin{quote}
{[}T{]}he finest and truest of all principles, in my view --- which is that for the good person, in the natural order of things, sacrifice to the gods, contact with them by means of prayers and offerings, and religious observance of every kind is at all times finest and best, the most likely to result in a happy life, and far and away the most appropriate thing for {[}them{]}.

Laws, 716d, with some gender-neutral modifications
\end{quote}

\hypertarget{exercise-self-compassion-meditation}{%
\section{Exercise: Self-Compassion Meditation}\label{exercise-self-compassion-meditation}}

Self-compassion meditations are everywhere. As someone who does not teach meditation, this exercise has two components:

\begin{enumerate}
\def\labelenumi{\arabic{enumi}.}
\tightlist
\item
  Locate a self-compassion meditation.
\item
  Do the meditation.
\item
  If you like the meditation, do it for a few weeks --- pick the date, time, and place you will do this. If you don't like the meditation, try another kind of self-compassion meditation until you find one that works for you.
\end{enumerate}

When thinking about where we fall short of an ideal, all too often, we judge ourselves with more harshness than we deserve. Some of us may even shut down and think, ``Well, if I'm not perfect, or with the background I have, what is even the point?'' Cultivating self-compassion blunts the impact of these feelings and helps us remain grounded in our inner goodness and potential to be better people.

When you locate a compassion meditation, try to figure out what the meditation is asking you to do. Compassion meditations that ask me to think of a nice thing I have done for someone else make me confused and worried. The first time I tried it, I failed to think of anything, so I looked online to see what people meant. It turned out that many people classify behaviors as ``nice'' that I was taught were basic courtesy. Even at work, my job is focused on helping people find what they need, so giving a stranger my time and attention is part of a normal day. I have more success with the type of compassion meditation in which one imagines feelings of goodwill and lovingkindness towards others. Some meditations ask us to visualize receiving love from someone who cares about us. I usually think of a God during those segments.

If you use Headspace, there is a self-compassion course in the app by Dora Kamau; she also has a website and \href{https://insighttimer.com/dorakamau}{a presence on Insight Timer}. The Headspace course uses my favorite type of compassion exercise. Other meditation apps will show you what is available after doing a keyword search. Dr.~Kristin Neff, a researcher who does significant work on self-compassion, has \href{https://self-compassion.org/guided-self-compassion-meditations-mp3-2/}{free downloadable meditations on her website}.

\hypertarget{exercise-contemplate-ethical-guidance}{%
\section{Exercise: Contemplate Ethical Guidance}\label{exercise-contemplate-ethical-guidance}}

During this contemplation exercise, find some time when you will not be disturbed for a few minutes. Find a pen and paper to write down any thoughts that come up.

Begin this exercise with a prayer to the God(s) of your choice. Then, read aloud the following:

\begin{quote}
Put more trust in nobility of character than in an oath. Never tell a lie. Pursue worthy aims. Do not be rash to make friends and, when once they are made, do not drop them. Learn to obey before you command. In giving advice seek to help, not to please, your friend. Be led by reason. Shun evil company. Honour the gods, reverence parents.

\href{https://en.wikisource.org/wiki/Lives_of_the_Eminent_Philosophers/Book_I\#Tenets_of_Solon}{\emph{Lives of the Eminent Philosophers}}, Tenets of Solon, Diogenes Laertius
\end{quote}

Taking each sentence on its own, what comes up for you? Where in your life do you display the value expressed in this tenet? Where could you invest more work, and how?

Do this for each of the elements in turn. When done, thank the God(s) and use the document you have created to create actionable plans for self-development. Feel free to move on to investigate other ethical texts and repeat this exercise. Usually, you will want to reflect on a small chunk of the material.

\hypertarget{navigating-groups}{%
\chapter{Navigating Groups}\label{navigating-groups}}

A \emph{group} is any collection of people, whether it has convened informally or formally. It could be your family, a collection of people you've met online, or a few people in town with whom you get drinks and catch up. It could also be your team at work, the members of an organization, or people on the membership rolls of a specific religion. In this chapter, we will discuss groups, both in a general sense and in terms of what you need to check on to avoid cults and toxic people.

Jean-Paul Sartre wrote that \href{https://www.vox.com/2014/11/17/7229547/philosophy-quotes-misunderstood-wittgenstein-sartre-descartes}{Hell is other people}, and to a certain extent, that is true. In the article I just linked, the writer explains that it's not because people irritate us and we have painful interpersonal drama, but because we are judged and judge in social situations. We can be both the victims of cruelty and the ones who mete it out. Now, in the context of many Gods, there is no hell that directly corresponds to the Christian one --- for example, the closest in the belief system I follow would be Tartarus, which is where souls that have made many bad decisions in their most recent life are purified. (Imagine if you were put in a room with a firm, yet compassionate, therapist-judge and couldn't leave until you'd worked through all you had going on in your last life. To call that \emph{rough} is an understatement.) Other belief systems have similar states of being. The point, however, is that we can be awful to one another. Sometimes this awfulness arises out of ignorance. At other times, someone wants to gain status and is willing to lie to get it. We can also get so caught up in the storylines of our embodied lives that we become unwilling to budge.

Paradoxically, \emph{Homo sapiens sapiens} is a species that evolved to be most at ease in tribal bands of 50-200 people. We need other people for psychological safety, well-being, and a sense of belonging. We have trouble with groups larger than that because our brains struggle with depersonalizing others. Even the most introverted person wants to feel a sense of respect and care from people in their immediate community. In a spiritual sense, it can be good to have family and friends to decompress with about spiritual issues and with whom you can celebrate larger holidays.

\hypertarget{family}{%
\section{Family}\label{family}}

I am very lucky in that I grew up praying to mother Goddesses in the backwoods. I was with my family and in a very loose community. We went to a Unitarian Society many Sundays, although the religious education program for kids was often more miss than hit, but the true joys were when we convened every six weeks at a rural property and were able to process to a sacred place, do a ritual, and have a potluck in community afterward. We celebrated the seasons, the elements, a God, and a Goddess.

My parents never required that we participate, and we rarely did family rituals at home. I am a self-directed, ``okay, let's do this'' person when it comes to spirituality, and I always have been. The Christian sect we were in when I was a young child made no sense, and when we apostatized, I was relieved because the framework of many Gods felt right --- later on, when I became interested in theology, my feelings became grounded in something more. My youngest sister rebelled against the family by getting interested in very conservative Christianity as a teen, although that didn't last. My middle sister felt that she wasn't included in the family rituals and that it was ``our thing'' --- for our parents and me. I only learned this after we were all adults and when she discussed wanting to go deeper into the practice, and it made me feel awful that I hadn't been there for her.

Based on comparing our childhood experiences, I do recommend involving your kids --- if you can --- in revering many Gods. Let them \emph{know} that they're included, explain why you pray to the Gods you pray to and that they're welcome to pray to whomever they like, and be open to their participation in your daily rituals if they want that. If they want a sacred space in their rooms, help them. I started using candles in my room when I was twelve, at about the same age I started doing my family's laundry and much of my family's weeknight cooking. You know your children and their maturity level. Depending on your kids and their engagement with large media franchises, you may have to explain how worshipping Greek deities is different from Percy Jackson (the ``chosenness'' thing can be particularly tricky for kids; I remember encountering a bunch of teens once who thought they were demigod children of deities because they'd fallen into that extreme belief online) or how the Marvel characters named for Norse Gods are just fun interpretations of myth from people who are mostly atheists.

➡ Make sure that, to the extent possible, any shared family ritual time is a device-free zone. It's OK to use eReaders or phones to access texts, show sacred images, or use a playlist. ⬅

Sometimes, a person starts following many Gods after a period of significant trauma in their early religious life, such as those who left cults. Emphasizing participation in spirituality or even teaching a child ethics can bring up feelings about what happened when the parent was younger. Keep in mind that your child may have no idea, or only a vague idea, of what happened to you. Shield them from your experience by modeling the values that you want them to have when they grow up. Emphasize ethics, but be sure to focus on a growth mindset in which we as individuals are continuously learning to be better people, not one in which small mistakes are damning. If you can, consider seeing an individual or family therapist.

Finally, involving your kid comes with some important safety tips. To be blunt, your child will go online --- either with your permission or in secret. What you want is for your child to have a firm foundation in reality so they don't end up joining a cult or some fringe group that believes it is channeling new physics from aliens on a planet orbiting Zeta Reticuli. Critical thinking skills are an absolute must. Teach your kids how to distinguish among religion, mystical/esoteric practices, New Age, and the occult. New Age is where many of the fringe beliefs are, but fringe can spill over into the other communities I just mentioned --- and, when we think of modern conspiracy phenomena, it is very clear that no segment of the population is wholly immune. Developing a firm foundation can be challenging for children who have a score of four or higher on ACE (Adverse Child Experiences), which correlates with higer chances of self-soothing escapism and failure to thrive as adults. I have an ACE score of four, and I held some embarrassing beliefs in my late teens as a way of self-soothing. It wasn't until going to therapy in my late 20s and graduating from therapy after getting very into Platonism and learning growth-based frames for thinking about myself and my life experiences that I started to feel steady.

Even when something they encounter online is sound, there is so much overlap online among pagans, polytheists, and indigenous traditions, on the one hand, and the occult and witchcraft on the other. Online, we are all exposed to new ideas from people with backgrounds that are completely unfamiliar to us, good and bad. \emph{The Soul's Inner Statues} is emphasizing a theistic, many-Gods approach that is not overly tied to any of these things --- and if that approach is what you are most comfortable with, keep that identity at your heart when you and your kids go online.

Kids don't need to be watched every second they're connected to the web, but they do need structure, critical thinking, and parents who care about their spiritual lives. They need to be taught what boundaries are.

\hypertarget{partners-and-roommates}{%
\section{Partners and Roommates}\label{partners-and-roommates}}

If you have other adults in your living space, like a spouse or roommate, involving them in your religion could take several forms. If the other adult also worships many Gods (likely different ones), holidays in your respective traditions are opportunities to share food and stories and pray together. If your spouse or roommates are not religious, they may also be perfectly fine with sharing a special meal for holidays --- most nonreligious people and atheists are not antireligious even if YouTube comments and online trolls indicate otherwise. If your spouse or roommate is antireligious, you will likely experience substantial friction. Members of exclusivist monotheistic faiths who are progressive may also be willing to have meals with you, but more conservative ones will likely have issues with you and call you an idolater. I know someone whose spouse believes she is consorting with demons, and it's psychologically hard for her as someone who is drawn to worshipping other Gods to have her spouse be that hostile.

\hypertarget{finding-people}{%
\section{Finding People}\label{finding-people}}

This section relies on three words for types of relationships: parasociality, acquaintanceship, and friendship. In the United States, many people will not distinguish between acquaintances and friends, but I find that this is a crucial personal frame to help me calibrate my social obligations towards others, especially as someone who was not trained to have good empathic boundaries as a child.

\emph{Parasocial} is a term developed in the mid-1950s to describe the relationship between TV personalities and their audiences. In the online era, Influencers began using it to describe their relationships with the massive numbers of people who interact with them. More recently, it has begun to refer to interactions among people who really don't know each other online --- you follow their content, they follow yours, but you two do not interact as either acquaintances or friends. In a parasocial relationship, we typically have one-sided information about someone. Online, we tend to share carefully-curated details about our lives, so much so that others feel they know and trust us when they may have only a vague idea of who we even are, if any idea at all. It plays with the follower's emotions and may lead to feelings of devotion, allegiance, goodwill, betrayal, moral outrage, and social rejection, as mercurial as a stormy sea, based on what we post. While most of us do not have a huge online reach, Influencers and celebrities do, and they experience the downsides of parasocial relationships more intensely than the rest of us.

\emph{Acquaintanceship} is a mutual state in which you vaguely know someone, but may not be on the same page about who both of you are. More distant acquaintances are people who see each other and say hi at the gym or while catching transit. Close acquaintances are people who have decided that they care about each other in a vague social sense, enough to have genuine interest in how the other is doing and perhaps learn the names of one another's pets or children. The mutuality is key here --- whereas in a parasocial relationship, you are engaging with someone's broadcasted content, in an acquaintanceship, you are actually getting to know each other. Many acquaintances are perfectly happy not to have a deeper level of engagement. Aristotle, while he had a dubious understanding of the Platonic Forms and an irritating lack of respect for women, has a lot of really useful perspective in the \emph{Nichomachean Ethics} (chapters 8 and 9) for thinking about friendship. He describes something called a ``friendship based on utility'' that ``belongs to the marketplace'' (1158a, ln. 21-23). Utilitarian friendships, in my opinion, are professional, collegial, or logistical close acquaintanceships.

The pathway from acquaintanceship to friendship involves gradual steps, usually in the form of mutual experiences. People seeking to make friends will also start to disclose more information, gradually testing the waters to see if the feelings of affinity are reciprocated and if the other party can be trusted.

\emph{Friendship}, like acquaintanceship, comes in layers. Casual friends care about one another, take interest in one another's lives, and may have some shared values in common. They may also be members of a social cluster who are more peripheral to each other than to others in the group, or they may have become friends due to shared professional or personal experiences. Friendship of this nature is also friendship between those who are prepared to be mutual and who see one another at (roughly) the same level --- as equals --- even if there are differences of age, social class, or cultural background. The ``purification of correction'' (a concept I've read about in Platonists), or the mutual checks friends offer one another, is available at this level of interaction because friends generally want what is best for one another and care enough to allow the other to be imperfect.

Aristotle writes that ``a wish for friendship arises swiftly, but friendship itself does not'' (1156b, 30). This is true in both the transition of acquaintances to friends and about the transition from casual to close friends. While in a casual friendship, one might censor oneself and keep things private, \emph{close} friends generally share much more of themselves with one another. Aristotle calls this form of friendship \emph{complete friendship} (1156b, line 5)--- the two of you have similar gasps of virtue and enough similarity in outlook that you can know and accept one another's character. A person cannot sustain many of these relationships at once.

Friendship is where ethical imperatives towards one's associates appear in full force --- Solon's ethical tenets instruct us to be slow in making friends, but not to abandon them once we have solidified our friendship. That makes the most sense when applied to friends who are at middle to total levels of closeness. Aristotle provides guidance on this, too, in the \emph{Nicomachean Ethics}, saying that corruptions of virtue and outlook among our close friends must be approached with care for their character, not out of interest in their public persona or property or the utilitarian benefits of the friendship. Even when a friendship cannot be maintained, he writes that ``on account of their prior friendship, {[}they must{]} render something to those who were once friends, when its dissolution was not due to excessive corruption'' (1165b, with the quotation from line 35). Ending a close friendship is a last resort. And trust me: If one of your friends, medium-close or close, gets cancelled or doxxed by parasocials, you will feel the heat from this, and the parasocials will assume you have no moral backbone and are willing to just chuck your friend to the curb at the drop of a hat.

While in a parasocial relationship, the feelings of betrayal have no actual basis in mutual care, feelings of betrayal and hurt when something bad happens in a casual-to-close friendship are based on a real, intangible connection. They can often be resolved by actually talking it out with the friend. Unlike with children, irreversible fallouts rarely happen at the drop of a hat. You can be more of your whole self with close friends without fearing that they will turn on you or that their care is conditional.

I started this section with an overview of friendships for one key reason: \textbf{Just because others worship Gods, and even the same Gods as you, doesn't mean you'll all get along.} This is a painful life lesson. Our personalities are all unique, no matter how we try to bin them into categories with Myers-Briggs, astrology, or anything else. One might think that being in the series of the same God, if two people were to be blessed with confident insight into that, would make them instant friends. Even in that case, since a God is everything in a unique way, and as each of us is a particular soul that expresses a unique fingerprint of that uniqueness, devotees express the full range of what that God is.

Ideally, when looking for people to approach about your mutual interest in the Gods, your goal is to identify candidates for close acquaintances and casual friends. Rarely, you will develop new close friends.

\hypertarget{local-organizations}{%
\subsection{Local Organizations}\label{local-organizations}}

Depending on the size and demographics of your local area, there may be people nearby with preexisting groups that you can join. Shops that sell alternative religious materials (like incense, deity statues, and so on) often have good inventories of the groups that exist nearby. You can also use Meetup and check for groups on social media that are local. Many will have some public events even if their core activites are private and members-only.

If you are worshipping deities that have temples or shrines in your metro area, learn more about the organization and contact them to ask if it is possible for you to pray there. Generally speaking, they will have volunteers and other people who would be happy to show a respectful outsider how to engage with the Gods in a tactful manner.

\hypertarget{social-media}{%
\subsection{Social Media}\label{social-media}}

To find friends in the 2020s, most people start on social media, private Discord or Slack chats, and Internet forums. There are some important pitfalls, and I advise you to use social media strategically to identify people you actually want to talk to. Once you have that, get out. Especially in private Discords that are run by Influencer personalities, things can quickly go south into cult-land.

The reason I encourage you to have an exit game is that the psychological impacts of heavy social media use are similar to other high-risk addictive behaviors. Once you feel parasocially obligated to be online and post to a large audience, you will also likely be ``trapped'' on Twitter, in Facebook groups, or on Instagram or TikTok. Studies show that people who are primarily active on social media for their social interaction face higher rates of anxiety and depression, as social media doses us with dopamine using algorithms similar to casino slot machines. Parasocial validation cannot functionally replace getting to know people in person.

While you are on social media, be it TikTok, Facebook, Twitter, or the latest and greatest new platform, do not feel pressure to follow back or friend anyone. Your update feed is your space. Look at the person's profile and decide if you find their updates parasocially relevant. If so, follow. If not, don't. I rarely follow back immediately --- it may take me weeks, months, or years to make that decision. Most people I follow are those writing content that I have found useful. In private social media spaces, I only let people see my content if I feel comfortable having them see it. Occasionally, I follow people back when I already know them (distant to close acquaintances). One rarely knows why someone is following the people they follow. Knowing a stranger's habits that well is a bit creepy.

Facebook and Instagram groups about the Gods are also filled with what I like to call aesthetic posts --- people posing with animal skins, horns, period clothing, and other items. It is materialistic and annoying for people who are actually there for content that helps them grow spiritually. It can also trigger our acquisitive instincts due to the human instinct to conform to group norms. Even when people post shrine photos, it can prompt a materialistic impulse for people whose setups are simpler and less expensive. Unless you know someone, you have no idea if they've been practicing for 20 years and have built their shrine items up over that amount of time or if they went into debt buying all of those items two months ago.

Social media also rewards oversharing and a lack of personal filter/boundaries. For this reason, spiritual bypassing, out-of-the-norm experiences, and misinformation are all rewarded by the algorithms. Rather than sit with a private experience for a while, many people instantly share what they think happened online. Sharing personal experiences is best done first in a private journal, then among people one actually knows and trusts. Because many of us are lonely and lack close friends, social media oversharing becomes even more seductive. We are all more likely to post when we are feeling strong emotions, too, like when online misinformation pushes our buttons. We all slip up.

Also, remember that when fights happen on social media, most people in the conversations are fighting with shadows created by their own fears, not with you. This is why people can get so vindictive --- the depersonalization involved in online interactions combines with whatever is going on in their psyches to make a bad situation even worse. If you have learned techniques from difficult conversations trainings (my workplace has these all the time), now is the time to use them. Over the years, I have learned that it is impossible to deescalate everything.

\hypertarget{the-mechanics-of-group-ritual}{%
\section{The Mechanics of Group Ritual}\label{the-mechanics-of-group-ritual}}

Once you find one or more people to do ritual with, it's time to look at your longer ritual outlines. Typically, whether you do a ritual in-person or on Zoom, you will need to share the ritual outline ahead of time. If everyone involved worships many Gods, there may be edits and amendments based on elements of practice that others want to include. Google Docs or another multi-person editing tool is great for this.

Collaboratively assign people roles. When I was growing up, we knew that the hosts who owned the property would be managing the overall ritual, but they took volunteers for invoking the elements and the ritual's deities. This happened when people were gathering in the processional area about ten minutes before the ritual started. If you know who will attend your rituals beforehand, in Google Drive or another collaboration tool, let people pick which sections of the ritual they want to take charge of.

If running a ritual on Zoom or another videoconferencing platform, be sure that someone is taking the role of Zoom host --- this person can spotlight the person doing each part of the ritual. Chanting isn't as easy on Zoom as it is in person, so you may want to mute everyone except the chant leader. Be clear with everyone whether you are all doing the ritual simultaneously together (with similar offerings and space setups) or if one person will show the ritual setup and make the offerings. If it's the latter, the designated ritual space provider should be spotlighted even when others are reading.

Some groups will do a debrief of the entire ritual immediately before the ritual takes place, which can be tedious for those who came prepared. I recommend doing this offline and setting an expectation that everyone is reading through it in advance. If new people are present who are unfamiliar, do not assign them anything, but make sure you or another person welcomes them and is available to answer any questions after the ritual.

Most often, people will use their own homes for rituals and trust that guests will follow rules of hospitality. They may also secure permits to do rituals in public spaces like parks if the ritual is large enough; if it's only a few people, they might just go to the park, depending on local regulations. Less frequently, groups may rent a space (a Unitarian Universalist Church is great for that), which can help establish trust among people who don't know one another well enough to feel okay sharing their home address for future rituals. Within traditions that worship many Gods, there are sometimes physical locations that are supported by the community, either temples or land.

\hypertarget{joining-groups}{%
\section{Joining Groups}\label{joining-groups}}

So, let's say that you decide you like a group --- online or offline --- and want to be a joiner. Congratulations on having taken such a big step! You're going to learn a lot, and I hope that this primer has been helpful.

When you are making your decision, here are some things to look for:

\begin{enumerate}
\def\labelenumi{\arabic{enumi}.}
\tightlist
\item
  Does the group have a nondiscrimination policy? How is it enforced? Is training mandatory for officers? Many are quick to sign statements about inclusivity, but few are willing to put in the work to ensure that leaders are actually prepared to handle such situations.
\item
  Is the group centered around a single person? If it is centered around one person, is it because they have the highest level of training in the tradition? Who trained them? Is there an accountability structure that applies to people with their level of training at the regional, national, or international level? How are officers and leaders of the group chosen?
\item
  Has the group appeared in the news associated with any cult scandals? Have former members written any exposés? What did they say, and how did the group respond?
\item
  If the group is claiming to be a revival of a tradition from a region, are people who are descendants of the group(s) living in that region represented in leadership positions --- and not just in a tokenizing fashion? Thinking back to the cultural appropriation content in Chapter Two, is the group engaging in any red flags? What do members of the group say about the tradition's living descendants in general? Is the group open to growth and change, or does it fall under ``reenactment''?
\item
  If you have children, is the group child-friendly? Are there some rituals that are fine for the whole family and others that are restricted to teens and adults? Is there childcare?
\item
  What are the educational development opportunities for members?
\item
  Is the group legally incorporated? This isn't a dealbreaker, but groups that have legally incorporated are often more stable than ones that are not. Decentralized continuous traditions may not be incorporated, so this question does not apply to them.
\end{enumerate}

Using these questions to frame your research, you should be prepared to sift through the organization's website and ask questions of members of the group if you attend a few rituals and want to know more about them. I was once in a group where long-term members repeatedly described, in an elated way, how they thought outsiders would go after them and how they were prepared for the worst. Red flags like that will likely not be on the website --- you have to get to know people.

Once you have found a group, sign up for its listservs, forums, and other online presence locations. Try to make at least a few of their in-person or online meetups when possible. When I first joined one group, I lurked for a long time, but it meant that I didn't get a feel for what other people in the group were like until I was eligibile for leadership positions, and I came to know that we did not have compatible values.

\hypertarget{when-leaders-go-bad}{%
\subsection{When Leaders Go Bad}\label{when-leaders-go-bad}}

Throughout our youths, we come to know that the authority figures in our lives --- our parents, other adults, religious teachers, and so on --- are just as human and fallible as we are. The first time I saw someone I considered a spiritual leader behave in a toxic way online, I was in my early 20s. Nowadays, with the way social media is designed, I imagine people have this shock at a much younger age. It felt awful, but I was only \emph{witnessing} what was happening, and it hardly impacted me at all beyond having to process extreme disappointment.

Unfortunately, that experience is mild in comparison with what happens to many other people. Cults, sexual abuse, and similar horrors are rampant in interpersonal spaces, spiritual or secular. These things may be centered around a specific person, or they may take place within the context of a group's organizational structure.

When it comes to individuals, we can draw some understanding from the commentary that Edwin F. Bryant made on Patañjali's \emph{Yoga Sūtras}. In a chapter on meditative absorption, Bryant translates I.15 as, ``Dispassion is the controlled consciousness of one who is without craving for sense objects, whether these are actually perceived or described'' (p.~52). He cites two prior commentators on the same passage of the \emph{Yoga Sūtras}: dispassion is ``indifference to objects even when these are available'' (Vācaspati Miśra) and includes ``members of the opposite sex, food, drink, and power'' (Vyāsa). It dovetails aptly with things said in Platonism about control of the passions of the appetitive and spirited parts of the human soul, which --- when they are left unchecked because someone has not properly brought them into alignment --- lead to disasters great and small. ``The wise,'' Bryant continues, ``strive for detachment and the eternal experience of the soul rather than the never-ending pursuit of ephemeral pleasure'' (53).

It is that detachment that allows someone to lead a group without allowing the power to go to their head --- and, crucially, even when someone starts out holy, they can backslide. Spiritual teachers are honestly some of the most vulnerable people to lies, tyrannical downfalls, vice, and abuse, partially because people trust them, and we are less likely to hold people we admire accountable. The goal of being a guide for others is to positively impact others' minds and souls, and those others then open up and become vulnerable. Socrates gets into how devastating spiritual abuse can be at Phaedo 89d-e, as it hardens people against any spirituality or truth-seeking at all --- the wound leaves them jaded. A spiritual teacher needs a group of peers and friends to hold them accountable, people who are not afraid to say the hard things.

In the same commentary on the \emph{Yoga Sūtras}, at I.14 (``practice becomes firmly established when it has been cultivated uninterruptedly and with devotion over a prolonged period of time,'' 49), Bryant takes what Patañjali has written to comment on the recent stream of abuse scandals in yoga. According to him, the practice must be cultivated continuously, like a garden. Otherwise, disaster may happen:

\begin{quote}
As an aside, many Hindu gurus and yogīs have been embroiled in scandals that have brought disrepute to the transplantation of yoga and other Indic spiritual systems to the West. This sūtra provides a mechanism of interpreting such occurrences. If one reads the early hagiographies of many Hindu gurus whose integrity was later found compromised, one is struck by the intensity, devotedness, and accomplishments of their initial practices. Nonetheless, however accomplished a yogī may become, if he or she abandons the practice of yoga under the notion of being enlightened or of having arrived at a point beyond the need of practice, it may be only a matter of time before past saṁskāras, including those of past sensual indulgences, now unimpeded by practice, begin to surface. The result is scandal and traumatized disciples. There is no flower bed, however perfected, that can counteract the relentless emergence of weeds if left unattended. As Patañjali will discuss later in the text, as long as one is embodied, saṁskāras remain latent, and therefore potential, in the citta {[}mind{]}. Hence one can read this sūtra as indicating that since the practices of yoga must be uninterrupted, one would be wise to politely avoid yogīs or gurus who claim to have attained a state of enlightenment such that they have transcended the need for the practice and renunciation presented by Patañjali here.

Yoga Sūtras, trans. Bryant, p.~51
\end{quote}

This passage could apply to any spiritual community: The leaders cannot, for the sake of those following them, fall prey to power. Sometimes, people go into leadership positions because they feel a lack that they believe only power over others can provide, and they spend the remainder of their lives hiding skeletons in their closets and using their current followers as a replacement for real self-work. Leadership often ends badly for them. In any case, the fallout impacts the entire group. It is negligence at its best, calculated indifference to the welfare of others at its worst. All teachers need training and accountability. All teachers need to demonstrate care towards the souls of others, imitating the Gods, as the Gods are wholly good.

Beyond leaders and teachers, toxic groups scare people with what will happen to them. They ostracize someone who criticizes a group policy instead of inviting discussion, explanations, and (perhaps) change. They terrorize people if they make simple human mistakes. They judge and punish if a member likes the wrong books, buys food the group doesn't approve of, or displays negative emotions. Meanwhile, they puff up followers with the idea that they are members of a special, awakened elite. You cannot be compassionate or cultivate civic virtue if you internalize us/them exclusivity rhetoric.

Often, those drawn into a cult are people who are motivated to do good in the world --- just as Plato teaches, people always want to do what they think is best, even if they are in error. Their strong motivation to do good is manipulated through love-bombing and the fear of ostracism and hatred from the people they respect. Us/them thinking and ``we're the only good people and you must believe these specific doctrines, and these alone'' rhetoric is extremely prevalent in online New Religious Movement communities in the 2020s, on the right and the left. Do not trust people who try to cut you off from different ideas while threatening to ostracize you if you disobey.

Paradoxically, the language of ``we're a niche'' is also used in a less intense way in nontoxic systems. In Platonism, for example, theurgic and philosophical incarnation choices are prioritized for achieving henosis and a providential exit from the current cosmic cycle. But it encourages critical thinking at the same time --- you are supposed to question, to work through the dialogues, to understand that even the holiest people incarnated were human beings. Exegesis is a torch relay. When people try to misapply Platonism for spiritual bypassing or to gain power, they will often be overly literal in interpreting texts and/or fuse Platonism with a zeitgeist.

Be wary of leaders who try to twist your embodiment anxiety into something toxic to your mind. Be wary of leaders who plaster their faces everywhere --- while dead holy people are honored with libations and appropriate commemorative well-wishes, they should not be revered as if they are on the same status as the Gods --- and those who scare you into silence with the threat of ostracism if you question anything or have slightly different beliefs than they do.

The Gods are the wellsprings of human happiness, as per Iamblichus. Humans are an ocean of transient pleasures and pains. Follow the Gods, and they will lead you up the mountain to touch the sky. Follow humans uncritically, and they may very well lead you to the bottom of an ocean trench.

\hypertarget{exercise-know-what-you-want}{%
\section{Exercise: Know What You Want}\label{exercise-know-what-you-want}}

Research groups, either tradition-specific or not, based on who you worship and what you are looking for in a community. Use the questions above to narrow down to one or two (if possible) to learn more about. Try attending one of the group's open events if you can.

Meanwhile, identify what you want in online or in-person spiritual friends. What are your dealbreakers? Where do you plan to meet people? If you plan to use social media, what rules do you want to set around using the platform?

These questions are ideal for answering in a journal or text file, and you can refer back to them over the next few months. I recommend checking in about your social media habits at least monthly to ensure that you are not succumbing to overuse.

\hypertarget{conclusion}{%
\chapter{Conclusion}\label{conclusion}}

We started off this primer by talking about the Gods and introducing ourselves to them. In seven chapters, we went through a whirlwind of information about getting started with a ritual practice, all framed as an uncovering of our soul's inner statues and our relationship to the Gods who root us to them.

It goes without saying that all spiritual practices are iterative. There are things I do now that I did not do when I was twenty; there are things I will do at fifty that I do not do now in my mid-30s. The same is true for all of us.

I wish you all of the blessings, luck, and divine guidance that the Gods can give you. I will close with words from Iamblichus.

\begin{quote}
Extended practice of prayer nurtures our intellect, enlarges very greatly our soul's receptivity to the gods, reveals to {[}people{]} the life of the gods, accustoms their eyes to the brightness of divine light, and gradually brings to perfection the capacity of our faculties for contact with the gods, until it leads us up to the highest level of consciousness (of which we are capable); also, it elevates gently the dispositions of our minds, and communicates to us those of the gods, stimulates persuasion and communion and indissoluble friendship, augments divine love, kindles the divine element in the soul, scours away all contrary tendencies within it, casts out from the aetherial and luminous vehicle surrounding the soul everything that tends to generation, brings to perfection good hope and faith concerning the light; and, in a word, it renders those who employ prayers, if we may so express it, the familiar consorts of the gods.

Iamblichus, \emph{On the Mysteries}, V.26, trans. Clarke et al.
\end{quote}

\hypertarget{acknowledgments-and-further-reading}{%
\chapter{Acknowledgments and Further Reading}\label{acknowledgments-and-further-reading}}

\hypertarget{acknowledgments}{%
\section{Acknowledgments}\label{acknowledgments}}

When I started praying to Bast at the age of nine, and at ten when my parents first brought me to Circle, the world was very different. As a kid, my dad would take me to the mall to Waldenbooks, where I perused the spirituality section (all four shelves) and the discount books on display at the front of the store. My mother would take me to the Opened Book in Hannibal, MO, where I purchased ritual supplies and books. As most children do, I acquired most of those early items by pleading; when I was 15 and able to work, I used my pocket money. The community in the Hannibal, MO, area was the best thing that happened to me as a child, a bright star. I dearly hope that everyone who knew me as a child, teen, and young adult forgives my embarrassing flaws and accepts my gratitude, both in that community and in the community I joined when I went to college. I am extremely grateful for the communities of affinity I have fallen into over the past few years --- while many have lamented the pandemic for a lack of closeness, Zoom has brought me to so many amazing people. You know who you are if you are reading this. I also acknowledge my more recent interlocutors on the blogosphere and social media --- we are all taking a journey together, as parasocial as it may be, regardless of how well we know each other. Growth by fire is lasting.

Outside of family and community, I have the authors of the books I have devoured over the course of my life to thank. Most of these books are long gone, decluttered when I downsized to move to my current city. The items below are the best I have to offer by way of a reference list, as a supplement to the direct hyperlinks I made in the \emph{Soul's Inner Statues} itself. The works hyperlinked in the text or referenced below represent what was most on my mind while I was writing. The core scaffolding of the \emph{Soul's Inner Statues} came from blog posts published on KALLISTI and its predecessor (also named KALLISTI) between 2009 and 2022.

\hypertarget{some-references-of-interest}{%
\section{Some References of Interest}\label{some-references-of-interest}}

ACRL. (2015, February 9). \emph{Framework for Information Literacy for Higher Education} {[}Text{]}. Association of College \& Research Libraries (ACRL). \url{https://www.ala.org/acrl/standards/ilframework}

Addey, T. (2011). \emph{The Unfolding Wings: The Way of Perfection in the Platonic Tradition}.

Aristotle. (2012). \emph{Aristotle's Nicomachean Ethics}. University of Chicago Press.

Bryant, E. F. (2009). \emph{Yoga Sutras of Patañjali} (First edition). North Point Press.

Butler, E. P. (n.d.). Noēseis Archives. \emph{Polytheist.com: Noēsis}. Retrieved March 8, 2022, from \url{http://polytheist.com/noeseis/}

Clarke, E. C. (2003). \emph{Iamblichus: On the Mysteries} (Bilingual edition). Society of Biblical Literature.

Clear, J. (2018). \emph{Atomic Habits: An Easy \& Proven Way to Build Good Habits \& Break Bad Ones} (Illustrated edition). Avery.

Gaifman, M. (2018). \emph{The Art of Libation in Classical Athens} (Illustrated edition). Yale University Press.

Griffin, M. (2014). \emph{Olympiodorus: Life of Plato and On Plato First Alcibiades 1--9}. Bloomsbury Publishing.

Iamblichus. (2020). \emph{Iamblichus' Life of Pythagoras, or Pythagoric Life Accompanied by Fragments of the Ethical Writings of certain Pythagoreans in the Doric dialect; and a collection of Pythagoric Sentences from Stobaeus and others, which are omitted by Gale in his Opuscula Mythologica, and have not been noticed by any editor} (T. Taylor, Trans.). \url{https://www.gutenberg.org/ebooks/63300}

Johnsen, L. (2007). \emph{Lost Masters: Sages of Ancient Greece}. Himalayan Institute Press.

Layne, D. A. (2014). The Character of Socrates and the Good of Dialogue Form: Neoplatonic Hermeneutics. In H. Tarrant \& D. A. Layne (Eds.), \emph{The Neoplatonic Socrates} (pp.~80--96). University of Pennsylvania Press.

Layne, D. A. (2017). The Anonymous Prolegomena to Platonic Philosophy. \emph{Brill's Companion to the Reception of Plato in Antiquity}, 533--554. \url{https://doi.org/10.1163/9789004355385_031}

Mandell, M. (2020). \emph{Discovering the Beauty of Wisdom}. Prometheus Trust.

Moore, M. (2020). \emph{Hidden Zen: Practices for Sudden Awakening and Embodied Realization}. Shambhala.

Nabarz, P. (2009). \emph{Stellar Magic: A Practical Guide to the Rites of the Moon, Planets, Stars and Constellations}. Avalonia.

Newport, C. (2019). \emph{Digital Minimalism: Choosing a Focused Life in a Noisy World}. Portfolio.

Petrovic, A., \& Petrovic, I. (2016). \emph{Inner Purity and Pollution in Greek Religion: Volume I: Early Greek Religion}. Oxford University Press. \url{https://doi.org/10.1093/acprof:oso/9780198768043.001.0001}

Plotinus. (n.d.) \emph{Plotinus: The Enneads} (L. P. Gerson, G. Boys-Stones, J. M. Dillon, R. A. H. King, A. Smith, \& J. Wilberding, Trans.). (2018). Cambridge University Press.

Proclus. (n.d.). \emph{The Six Books of Proclus, the Platonic Successor, on the Theology of Plato} (T. Taylor, Trans.). Wikisource.

Proclus. (2003). \emph{On the Existence of Evils}. Cornell University Press.

Proclus. (2013). \emph{Proclus: Commentary on Plato's Timaeus: Volume 5, Book 4} (D. Baltzly, Trans.). Cambridge University Press.

Proclus, Opsomer, J., Steel, C. G., \& Bloomsbury (Firm). (2012). \emph{Ten problems concerning providence}. Bristol Classical Press. \url{https://doi.org/10.5040/9781472552143}

Proclus, \& Steel, C. G. (2007). \emph{On providence}. Cornell University Press. \url{http://books.google.com/books?id=ZV8EAQAAIAAJ}

Raheem, O. F. (2022). \emph{Pause, Rest, Be: Stillness Practices for Courage in Times of Change}. Shambhala.

Sallustius. (n.d.). \emph{Sallustius On the Gods and the World} (T. Taylor, Trans.). Wikisource.

Syrianus. (2007). \emph{Syrianus: On Aristotle Metaphysics 13-14} (R. Sorabji, Ed.; J. Dillon, Trans.). Bristol Classical Press.

Wijeyakumar, A. (2021). \emph{Meditation with Intention: Quick \& Easy Ways to Create Lasting Peace}. Llewellyn Publications.

\end{document}
